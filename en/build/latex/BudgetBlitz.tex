%% Generated by Sphinx.
\def\sphinxdocclass{report}
\documentclass[a4paper,10pt,english]{sphinxmanual}
\ifdefined\pdfpxdimen
   \let\sphinxpxdimen\pdfpxdimen\else\newdimen\sphinxpxdimen
\fi \sphinxpxdimen=.75bp\relax
\ifdefined\pdfimageresolution
    \pdfimageresolution= \numexpr \dimexpr1in\relax/\sphinxpxdimen\relax
\fi
%% let collapsible pdf bookmarks panel have high depth per default
\PassOptionsToPackage{bookmarksdepth=5}{hyperref}

\PassOptionsToPackage{warn}{textcomp}
\usepackage[utf8]{inputenc}
\ifdefined\DeclareUnicodeCharacter
% support both utf8 and utf8x syntaxes
  \ifdefined\DeclareUnicodeCharacterAsOptional
    \def\sphinxDUC#1{\DeclareUnicodeCharacter{"#1}}
  \else
    \let\sphinxDUC\DeclareUnicodeCharacter
  \fi
  \sphinxDUC{00A0}{\nobreakspace}
  \sphinxDUC{2500}{\sphinxunichar{2500}}
  \sphinxDUC{2502}{\sphinxunichar{2502}}
  \sphinxDUC{2514}{\sphinxunichar{2514}}
  \sphinxDUC{251C}{\sphinxunichar{251C}}
  \sphinxDUC{2572}{\textbackslash}
\fi
\usepackage{cmap}
\usepackage[T1]{fontenc}
\usepackage{amsmath,amssymb,amstext}
\usepackage{babel}



\usepackage{tgtermes}
\usepackage{tgheros}
\renewcommand{\ttdefault}{txtt}



\usepackage[Bjarne]{fncychap}
\usepackage{sphinx}

\fvset{fontsize=auto}
\usepackage{geometry}


% Include hyperref last.
\usepackage{hyperref}
% Fix anchor placement for figures with captions.
\usepackage{hypcap}% it must be loaded after hyperref.
% Set up styles of URL: it should be placed after hyperref.
\urlstyle{same}


\usepackage{sphinxmessages}
\setcounter{tocdepth}{1}

\usepackage{bbstyle}

\title{BudgetBlitz: Documentation}
\date{Sep 22, 2023}
\release{2.8}
\author{Basin Michael}
\newcommand{\sphinxlogo}{\vbox{}}
\renewcommand{\releasename}{Release}
\makeindex
\begin{document}

\ifdefined\shorthandoff
  \ifnum\catcode`\=\string=\active\shorthandoff{=}\fi
  \ifnum\catcode`\"=\active\shorthandoff{"}\fi
\fi

\pagestyle{empty}
\sphinxmaketitle
\pagestyle{plain}
\sphinxtableofcontents
\pagestyle{normal}
\phantomsection\label{\detokenize{index::doc}}


\sphinxstepscope


\chapter{Notations}
\label{\detokenize{notations:notations}}\label{\detokenize{notations::doc}}
\sphinxAtStartPar
Menu: \sphinxmenuselection{Actions \(\rightarrow\) Profiles}

\sphinxAtStartPar
Button: \DUrole{bbbutton}{SMS and Push tunes}

\sphinxAtStartPar
Spinner: \DUrole{bbspinner}{Report settings}

\sphinxAtStartPar
Directory, report, and chart: \DUrole{bbmeta}{SMS import tunes}

\sphinxAtStartPar
Directory item: \DUrole{bbitem}{Personal}

\sphinxAtStartPar
Section: \DUrole{bbsection}{View}

\sphinxAtStartPar
Property of directory item: \DUrole{bbproperty}{Name}

\sphinxAtStartPar
Value that manually edited:  \DUrole{bbvalue}{One account summary}

\sphinxAtStartPar
Variable: \DUrole{bbvar}{biz.interblitz.intent.CONVERT\_TEXT\_TO\_NEW\_TRANSACTION}

\sphinxstepscope


\chapter{Preface}
\label{\detokenize{preface:preface}}\label{\detokenize{preface::doc}}
\sphinxAtStartPar
It would be nice to draw your attention to the common issue of a Budget Blitz for Android user. It is the high entry level.
That means a user should have some time for studying the application. Therefore please schedule a time
and it will not be lost for nothing. Most users say they do not regret about time spent.


\section{Introduction}
\label{\detokenize{preface:introduction}}
\sphinxAtStartPar
User manual will help you to get started with the app Budget Blitz for Android. The guide is not exhaustive but it is constantly
supplemented and developed along with the application. Comments and suggestions are welcome,
see. {\hyperref[\detokenize{preface:feedback}]{\sphinxcrossref{Feedback}}} (\autopageref*{\detokenize{preface:feedback}}).


\section{Additional Sources}
\label{\detokenize{preface:additional-sources}}
\sphinxAtStartPar
Questions and answers in English: \sphinxurl{http://qa.bbmoney.biz/en/}

\sphinxAtStartPar
Questions and answers in Russian: \sphinxurl{http://qa.bbmoney.biz/ru/}

\sphinxAtStartPar
4PDA discussion: \sphinxurl{http://4pda.ru/forum/index.php?showtopic=658215}

\sphinxAtStartPar
Previous user manual: \sphinxurl{http://interblitz.biz/projects/blitz-0035/wiki/User\_manual}


\section{User Manual Editions}
\label{\detokenize{preface:user-manual-editions}}
\sphinxAtStartPar
HTML: \sphinxurl{http://bbmoney.biz/en/manual/index.html}

\sphinxAtStartPar
PDF: \sphinxurl{http://bbmoney.biz/en/assets/budgetblitz-user-manual.pdf}


\section{Feedback}
\label{\detokenize{preface:feedback}}\label{\detokenize{preface:section-feedback}}
\sphinxAtStartPar
Author: Basin Michael

\sphinxAtStartPar
Contacts: \sphinxhref{mailto:basin.michael@gmail.com}{basin.michael@gmail.com}

\sphinxstepscope


\chapter{About}
\label{\detokenize{about:about}}\label{\detokenize{about::doc}}
\sphinxAtStartPar
Budget Blitz for Android is used for accounting and planning:
\begin{itemize}
\item {} 
\sphinxAtStartPar
personal finances;

\item {} 
\sphinxAtStartPar
very small business finances;

\item {} 
\sphinxAtStartPar
small business finances.

\end{itemize}

\noindent\sphinxincludegraphics[width=0.250\linewidth]{{about-010-main-screen}.png}

\noindent\sphinxincludegraphics[width=0.250\linewidth]{{about-170-pie}.png}

\noindent\sphinxincludegraphics[width=0.250\linewidth]{{about-190-lines}.png}


\section{Key Features}
\label{\detokenize{about:key-features}}
\sphinxAtStartPar
Combining the accounting of personal finances and finances of a company.

\sphinxAtStartPar
Comprehensive accounting of finances, i.e. categories, payers and payees, persons, projects are supported.

\sphinxAtStartPar
Automatic recognition of SMS and push notifications coming from financial institutions. Amounts, commissions, categories, projects, persons,
payers and payees detection, automatic balance adjustment, 160+ banks of different countries supported,
see. {\hyperref[\detokenize{banks:chapter-supported-banks}]{\sphinxcrossref{\DUrole{std,std-ref}{Ready to use Financial Institutes}}}} (\autopageref*{\detokenize{banks:chapter-supported-banks}}).

\sphinxAtStartPar
Financial highlights on the main screen.

\sphinxAtStartPar
Smart default values for new transactions.

\sphinxAtStartPar
App widget acting as a customizable brief report.

\sphinxAtStartPar
Distributed teamwork with customizable access rights.

\sphinxAtStartPar
PC web browser driven client.

\sphinxAtStartPar
API for receiving transactions from other applications.

\sphinxAtStartPar
Various financial reports.

\sphinxAtStartPar
Reports driven notifications.


\section{Interesting Solutions Implemented}
\label{\detokenize{about:interesting-solutions-implemented}}
\sphinxAtStartPar
Banks SMS and push notifications detection subsystem:
\begin{itemize}
\item {} 
\sphinxAtStartPar
Automatic category, payer and payee, person, project recognition;

\item {} 
\sphinxAtStartPar
Convenient key phrases selection immediately from SMS and push notifications;

\item {} 
\sphinxAtStartPar
Automatic calculation of rates for foreign transactions;

\item {} 
\sphinxAtStartPar
Automatic detection of transfers between accounts;

\item {} 
\sphinxAtStartPar
Option to create custom SMS and push notifications import tune in order to support new banks.

\end{itemize}

\sphinxAtStartPar
Reports subsystem:
\begin{itemize}
\item {} 
\sphinxAtStartPar
Simplified OLAP technology is used for reports;

\item {} 
\sphinxAtStartPar
Drilldown feature supported;

\item {} 
\sphinxAtStartPar
App widgets are used for brief reports;

\item {} 
\sphinxAtStartPar
Shortcuts with predefined settings for reports.

\item {} 
\sphinxAtStartPar
Reports driven notifications, access to the prepared report from notification.

\end{itemize}

\sphinxAtStartPar
Teamwork subsystem:
\begin{itemize}
\item {} 
\sphinxAtStartPar
Data exchange is used, no registration, no common database, each team member has own database.

\item {} 
\sphinxAtStartPar
Flexible system of rights and data areas for exchange. You can synchronize transactions between team members for only one account, a project, and so on.

\item {} 
\sphinxAtStartPar
Unlimited number of team members.

\end{itemize}

\sphinxAtStartPar
PC access subsystem:
\begin{itemize}
\item {} 
\sphinxAtStartPar
Windows, Linux, Mac, etc are supported by PC client. All you need is modern browser. Internet Explorer 8+, Google Chrome, Apple Safari, Mozilla Firefox, Opera supported.

\end{itemize}

\sphinxstepscope


\chapter{How It Works}
\label{\detokenize{intro:how-it-works}}\label{\detokenize{intro::doc}}

\section{Funds Accounting}
\label{\detokenize{intro:funds-accounting}}
\sphinxAtStartPar
Budget Blitz for Android uses transactions to store all movement of funds. Each transaction has
four dimensions category, project, payer or payee, and person. Transactions can be
actual or planned and onetime or recurring. Recurring transactions have own customizable frequency. Generally
this ones are planned but it is possible to make actual recurring transactions too.

\sphinxAtStartPar
Any transaction can be split for details. These transactions are called split, see. {\hyperref[\detokenize{glossary:term-split}]{\sphinxtermref{\DUrole{xref,std,std-term}{split}}}}.


\section{Directories Structure}
\label{\detokenize{intro:directories-structure}}
\sphinxAtStartPar
Each transaction has own account. It can be account of a financial institution, or e\sphinxhyphen{}money account, or cash,
or something else. Each transaction has own currency that may be differ from account currency.

\sphinxAtStartPar
In its turn each account belongs to a portfolio. Each portfolio has own currency that may be differ from account currency also.

\sphinxAtStartPar
But that’s not all. Portfolios have a type. It may be considered as a kind of activity. You will have only one
personal type of portfolio for personal finances. But when you a have a business then personal and
business portfolio types will be in use. For some special cases more than two portfolio types can be used.

\sphinxAtStartPar
Category, project, payer or payee, and person connected to the portfolio type. It is not really
complex as you can see on the chart below:

\noindent{\hspace*{\fill}\sphinxincludegraphics[width=0.750\linewidth]{{directories_structure_en}.png}\hspace*{\fill}}

\begin{sphinxadmonition}{note}{Note:}
\sphinxAtStartPar
You can edit any directory. For example, add a currency, a category or something else. There are no restrictions!
\end{sphinxadmonition}


\section{Difference Between Contractors and Persons}
\label{\detokenize{intro:difference-between-contractors-and-persons}}
\sphinxAtStartPar
Contrary part of transaction is payer or payee. This is often called a contractor. Only transfer transactions
have no contractor. All other transactions have. For example your child is contractor
when you give him or her some money. So you should put him or her to the \DUrole{bbmeta}{Payers and payees} directory.

\sphinxAtStartPar
Persons are transaction details as well as categories and projects. For a example a shop is the contractor
and your child is the person when you buy clothes to your child.

\sphinxAtStartPar
There is an option to connect contractor and person. To do so just define a person in a card
of contractor. After that the person will be selected within the contractor. For instance, in order
to combine both examples above, you should consider the child as the contractor and the person at the same time.

\sphinxAtStartPar
As the result, you will see total child expenses using a person filter and total
money delivered to the child using a contractor filter.

\sphinxstepscope


\chapter{Getting Started}
\label{\detokenize{getting-started:getting-started}}\label{\detokenize{getting-started::doc}}
\sphinxAtStartPar
In this chapter it is suggested the certain sequence of actions for setting up the application.
If you don’t like that, you don’t have to.  Just remember, any option you like, you can change later.


\section{Basic Customization}
\label{\detokenize{getting-started:basic-customization}}
\sphinxAtStartPar
Edit basic settings after the first start of Budget Blitz for Android.

\noindent\sphinxincludegraphics[width=0.250\linewidth]{{gettingstarted-010-main-screen}.png}

\noindent\sphinxincludegraphics[width=0.250\linewidth]{{gettingstarted-020-click-on-actions-menu}.png}

\noindent\sphinxincludegraphics[width=0.250\linewidth]{{gettingstarted-030-settings}.png}

\sphinxAtStartPar
Here you can:
\begin{itemize}
\item {} 
\sphinxAtStartPar
set a secret key pattern to restrict app access;

\item {} 
\sphinxAtStartPar
set SMS and push notifications parser on or off;

\item {} 
\sphinxAtStartPar
set synchronization between devices on or off;

\item {} 
\sphinxAtStartPar
set default money amount sign for new transactions;

\item {} 
\sphinxAtStartPar
set main currency and a source for foreign exchange rates;

\item {} 
\sphinxAtStartPar
set up automatic backups;

\item {} 
\sphinxAtStartPar
set upcoming payment notifications on or off;

\item {} 
\sphinxAtStartPar
set ring tones when transactions created on SMS and push notifications coming.

\end{itemize}

\sphinxAtStartPar
While basic settings are ready you can go deeper. Next steps you can see on the main screen.


\section{Loading Bank Settings}
\label{\detokenize{getting-started:loading-bank-settings}}
\sphinxAtStartPar
This section is intended for users who plan to use automatic creation of transactions on SMS or push notifications arrived
from bank or payment systems or other installed apps.

\sphinxAtStartPar
Requirements for automatic transactions creation on SMS push:
\begin{enumerate}
\sphinxsetlistlabels{\arabic}{enumi}{enumii}{}{.}%
\item {} 
\sphinxAtStartPar
the option should be on, see menu \sphinxmenuselection{Actions \(\rightarrow\) Profiles};

\item {} 
\sphinxAtStartPar
the app should have access to SMS (Android 6+).

\end{enumerate}

\sphinxAtStartPar
Requirements for automatic transactions creation on notification push (use menu \sphinxmenuselection{Actions \(\rightarrow\) Profiles}):
\begin{enumerate}
\sphinxsetlistlabels{\arabic}{enumi}{enumii}{}{.}%
\item {} 
\sphinxAtStartPar
you should select packages to import push notifications:

\item {} 
\sphinxAtStartPar
you should grant access rights to the app.

\end{enumerate}

\sphinxAtStartPar
Press the button \DUrole{bbbutton}{SMS and Push tunes} to load setting for your bank or payment system. Hereafter
to do that use menu \sphinxmenuselection{Actions \(\rightarrow\) Import \(\rightarrow\) SMS and Push tunes} or directory
\DUrole{bbmeta}{SMS import tunes}.

\noindent\sphinxincludegraphics[width=0.250\linewidth]{{gettingstarted-050-click-on-import-tunes}.png}

\noindent\sphinxincludegraphics[width=0.250\linewidth]{{gettingstarted-060-available-sms-tunes}.png}


\section{Portfolio and Account Settings}
\label{\detokenize{getting-started:portfolio-and-account-settings}}
\sphinxAtStartPar
Once installed the app has three portfolio types \DUrole{bbitem}{Personal}, \DUrole{bbitem}{Small business}, \DUrole{bbitem}{Universal},
one personal portfolio \DUrole{bbitem}{Wallet}, two accounts \DUrole{bbitem}{Card}, \DUrole{bbitem}{Cash} and default list of categories.

\sphinxAtStartPar
Values to show depend on the portfolio type of the transaction.  For example,
one categories list is used for personal finances and another for a small business finances.
But nevertheless there are some common categories. Universal type of the portfolio is used
for this values. Universal values shown regardless portfolio type of the transaction.

\sphinxAtStartPar
A portfolio is like a group of accounts. On the main screen the app groups accounts by portfolios and
calculates financial highlights.

\sphinxAtStartPar
Make the required amount of portfolios and accounts. See {\hyperref[\detokenize{account-identities:chapter-account-identities}]{\sphinxcrossref{\DUrole{std,std-ref}{Account Settings for Notifications Import}}}} (\autopageref*{\detokenize{account-identities:chapter-account-identities}}) to use automatic transactions creation
via SMS or push notifications.

\sphinxAtStartPar
Now you can import SMS or import initial transactions from \sphinxhref{https://en.wikipedia.org/wiki/CSV}{CSV} and \sphinxhref{https://en.wikipedia.org/wiki/Open\_Financial\_Exchange}{OFX} or just enter initial balance.
Once you have at least one transaction the main screen will show a summary and buttons will disappear.


\section{Initial Data Import}
\label{\detokenize{getting-started:initial-data-import}}
\sphinxAtStartPar
First of all check account settings according chapter {\hyperref[\detokenize{account-identities:chapter-account-identities}]{\sphinxcrossref{\DUrole{std,std-ref}{Account Settings for Notifications Import}}}} (\autopageref*{\detokenize{account-identities:chapter-account-identities}}). Then press \DUrole{bbbutton}{SMS and Push notifications}
in the \DUrole{bbsection}{Import} section or select the menu \sphinxmenuselection{Actions \(\rightarrow\) Import \(\rightarrow\) SMS and Push notifications}, select an account
and then import notifications. See more details in the chapter {\hyperref[\detokenize{import:chapter-import}]{\sphinxcrossref{\DUrole{std,std-ref}{Data Import}}}} (\autopageref*{\detokenize{import:chapter-import}}) and \sphinxhref{http://qa.bbmoney.biz/en/index.php?qa=3\&qa\_1=manual-sms-import}{questions and answers}.

\sphinxAtStartPar
Also you can import initial transactions from \sphinxhref{https://en.wikipedia.org/wiki/CSV}{CSV} and \sphinxhref{https://en.wikipedia.org/wiki/Open\_Financial\_Exchange}{OFX} file. Check and edit CSV file before import according to chapter {\hyperref[\detokenize{import:chapter-import}]{\sphinxcrossref{\DUrole{std,std-ref}{Data Import}}}} (\autopageref*{\detokenize{import:chapter-import}}).
You have no need to modify OFX file before import.


\section{Entering Initial Balance and Credit Limit}
\label{\detokenize{getting-started:entering-initial-balance-and-credit-limit}}
\sphinxAtStartPar
You can enter initial account balance via transaction. The date of transaction can be arbitrary but it is
highly preferred to make transaction first in the transactions list.
Another important thing is to use category \DUrole{bbitem}{Initial balance loading} for that transaction.

\noindent\sphinxincludegraphics[width=0.250\linewidth]{{initialbalance-010-initial-transaction}.png}

\noindent\sphinxincludegraphics[width=0.250\linewidth]{{initialbalance-020-initial-budget-item}.png}

\sphinxAtStartPar
As well as initial balance you can enter a credit limit via transaction too. It is preferred that the transaction date coincide with the date of setting the limit by the bank.
Use \DUrole{bbitem}{Credit limit changing} category. Please pay attention that it is a technical category with
\DUrole{bbproperty}{Revenue} and \DUrole{bbproperty}{Expense} set off. You can see more details about such approach reasons
at \sphinxhref{http://qa.bbmoney.biz/en/index.php?qa=27\&qa\_1=how-to-setup-credit-limit-for-new-or-existed-account}{questions and answers (How to setup credit limit for new or existed account)}.

\noindent\sphinxincludegraphics[width=0.250\linewidth]{{initialbalance-040-credit-limit-transaction}.png}

\noindent\sphinxincludegraphics[width=0.250\linewidth]{{initialbalance-050-credit-limit--budget-item}.png}

\sphinxAtStartPar
It would be better to enter each debt or credit with two transactions. For example you have a loan of 1000 USD. So you should
\begin{enumerate}
\sphinxsetlistlabels{\arabic}{enumi}{enumii}{}{.}%
\item {} 
\sphinxAtStartPar
make a positive transaction with amount of 1000 USD using category \DUrole{bbitem}{Loans} and a real contractor or person.

\item {} 
\sphinxAtStartPar
make a negative transaction with amount of 1000 USD using category \DUrole{bbitem}{00 None} or real one if known.

\end{enumerate}

\sphinxAtStartPar
As result the balance is equal zero, but \DUrole{bbmeta}{Debts and credits} report will show the value of your loan.

\sphinxstepscope


\chapter{Main Screen}
\label{\detokenize{main-screen:main-screen}}\label{\detokenize{main-screen:chapter-main-screen}}\label{\detokenize{main-screen::doc}}

\section{Description}
\label{\detokenize{main-screen:description}}
\sphinxAtStartPar
Main screen of Budget Blitz for Android contains portfolios and accounts financial highlights. The app shows
as much highlights as you have portfolio types. Examples below have only one \DUrole{bbitem}{Personal} portfolio type.

\sphinxAtStartPar
Financial highlight has totals of each account and portfolio. Available amount is shown when account
has a credit limit.

\sphinxAtStartPar
Actual and planned amounts of revenues and expenses are placed below totals. Also a transfer amount is shown
if exists in the current time range.

\noindent\sphinxincludegraphics[width=0.250\linewidth]{{mainscreen-010-main-screen}.png}

\noindent\sphinxincludegraphics[width=0.250\linewidth]{{mainscreen-014-main-screen-swipe-left}.png}

\noindent\sphinxincludegraphics[width=0.250\linewidth]{{mainscreen-015-main-screen-transactions}.png}

\sphinxAtStartPar
The list of all transactions according the current time range is shown at the left of summary.


\section{Time Range Selection}
\label{\detokenize{main-screen:time-range-selection}}
\sphinxAtStartPar
Time range editor is placed at the top of screen. Editor supports swipe and select gestures.

\noindent\sphinxincludegraphics[width=0.250\linewidth]{{mainscreen-089-main-screen-dates-range-swipe-left}.png}

\noindent\sphinxincludegraphics[width=0.250\linewidth]{{mainscreen-090-main-screen-dates-range-swipe}.png}

\noindent\sphinxincludegraphics[width=0.250\linewidth]{{mainscreen-100-main-screen-dates-range-spinner}.png}


\section{List View Settings}
\label{\detokenize{main-screen:list-view-settings}}
\sphinxAtStartPar
List view settings is placed at the bottom of screen. You can change default grouping,
edit filter, and change time range. Using filter you can setup a portfolio, account,
currency restriction, and put plan on or off.

\sphinxAtStartPar
At the pictures below you can see how to set up account filter.

\noindent\sphinxincludegraphics[width=0.250\linewidth]{{mainscreen-020-main-screen-bottom-sheet-opening}.png}

\noindent\sphinxincludegraphics[width=0.250\linewidth]{{mainscreen-030-main-screen-bottom-sheet-open}.png}

\noindent\sphinxincludegraphics[width=0.250\linewidth]{{mainscreen-040-main-screen-filter}.png}

\noindent\sphinxincludegraphics[width=0.250\linewidth]{{mainscreen-050-main-screen-filter-account}.png}

\noindent\sphinxincludegraphics[width=0.250\linewidth]{{mainscreen-060-main-screen-filter-apply}.png}

\noindent\sphinxincludegraphics[width=0.250\linewidth]{{mainscreen-065-main-screen-filter-applied}.png}

\sphinxAtStartPar
Now the main screen has only one \DUrole{bbitem}{Cash} account.


\section{Saving List View Settings}
\label{\detokenize{main-screen:saving-list-view-settings}}
\sphinxAtStartPar
You can save modified list view settings in order to use them in the future. Select
\DUrole{bbspinner}{Report settings} and create new setting. Filters will be copied to the new setting.
So, you need to put a name of the setting, for example, \DUrole{bbvalue}{One account summary}, and
press \DUrole{bbbutton}{Save}.

\noindent\sphinxincludegraphics[width=0.250\linewidth]{{mainscreen-070-main-screen-select-new-setting}.png}

\noindent\sphinxincludegraphics[width=0.250\linewidth]{{mainscreen-080-main-screen-setting-save}.png}

\sphinxAtStartPar
It is possible to have several settings. The app will use last setting for the main screen
after run.

\sphinxstepscope


\chapter{Directories}
\label{\detokenize{directories:directories}}\label{\detokenize{directories:chapter-directories}}\label{\detokenize{directories::doc}}
\sphinxAtStartPar
Directories are available from the top menu or from the \sphinxmenuselection{Actions \(\rightarrow\) Directories} menu.
It depends on the current screen.

\noindent\sphinxincludegraphics[width=0.250\linewidth]{{directories-010-select-directories}.png}

\noindent\sphinxincludegraphics[width=0.250\linewidth]{{directories-020-menu-directories}.png}


\section{Portfolio Types}
\label{\detokenize{directories:portfolio-types}}
\sphinxAtStartPar
Portfolio types are used to separate dimensions between your activities. For example, one set of
categories is used for personal finance and another set is used for business finances. When you
edit a transaction its dimensions (category,  payer or payee, project, and person) correspond
to the certain type of portfolio.

\noindent\sphinxincludegraphics[width=0.250\linewidth]{{directories-030-types-of-portfolio}.png}

\noindent\sphinxincludegraphics[width=0.250\linewidth]{{directories-040-portfolios}.png}

\noindent\sphinxincludegraphics[width=0.250\linewidth]{{directories-050-accounts}.png}

\sphinxAtStartPar
It will be useful to pay attention to the \DUrole{bbitem}{Universal} type of portfolio. Its name is \DUrole{bbitem}{00 None}
for old versions. Dimensions of this type of portfolio are always available. For example \DUrole{bbitem}{Transfer}
category can be selected in any transaction regardless selected account and connected portfolio type.

\begin{sphinxadmonition}{note}{Note:}
\sphinxAtStartPar
Dimensions of the \DUrole{bbitem}{Universal} type of portfolio are always available
\end{sphinxadmonition}


\section{Portfolios}
\label{\detokenize{directories:portfolios}}
\sphinxAtStartPar
Portfolio is a group of accounts. Portfolio has own currency. The Budget Blitz for Android uses a portfolio currency
to show financial highlights. Use the \DUrole{bbmeta}{Currencies} directory to apply currency rates.


\section{Accounts}
\label{\detokenize{directories:accounts}}
\sphinxAtStartPar
Account is a bank account, debit or credit card, investment account, cash, or something else.
Account has own currency. It may be differ from a currency of portfolio.

\sphinxAtStartPar
Identity of an account is used for transactions import, see {\hyperref[\detokenize{import:chapter-import}]{\sphinxcrossref{\DUrole{std,std-ref}{Data Import}}}} (\autopageref*{\detokenize{import:chapter-import}}).
You can put several identities. Use the comma to separate one identity from another.
Usually card or account number is used for identity. Phone number, or SMS sander name,
or identity of push notifications package can be considered as identity as well.

\sphinxAtStartPar
Key phrases are also used for transactions import. For transfers source account is detected by identity
and target account is detected by key phrases. Key phrases are used for transfers only.

\sphinxAtStartPar
For example you have SMS from bank:

\begin{sphinxVerbatim}[commandchars=\\\{\}]
\PYG{n}{Visa2900} \PYG{n}{card}\PYG{o}{.} \PYG{n}{Cash} \PYG{n}{withdraw} \PYG{l+m+mf}{200.00} \PYG{n}{USD} \PYG{n}{ATM} \PYG{l+m+mf}{5412.} \PYG{n}{Balance}\PYG{p}{:} \PYG{l+m+mf}{274.26} \PYG{n}{USD}\PYG{o}{.} \PYG{l+m+mi}{25}\PYG{o}{/}\PYG{l+m+mi}{03}\PYG{o}{/}\PYG{l+m+mi}{14}\PYG{p}{,}\PYG{l+m+mi}{15}\PYG{p}{:}\PYG{l+m+mi}{00}\PYG{p}{:}\PYG{l+m+mf}{00.}
\end{sphinxVerbatim}

\sphinxAtStartPar
Visa2900 is identity of the \DUrole{bbitem}{Card} account, ATM is the key phrase for the \DUrole{bbitem}{Cash} account.
Since SMS pushed the app Budget Blitz for Android will create two transactions, debit transaction on \DUrole{bbitem}{Card} account
and credit transaction on \DUrole{bbitem}{Cash} account.

\sphinxAtStartPar
SMS import tune establishes the algorithm of transactions detection. See more details in the {\hyperref[\detokenize{notifications:chapter-notifications}]{\sphinxcrossref{\DUrole{std,std-ref}{Advanced Import SMS and Push Notifications Setting}}}} (\autopageref*{\detokenize{notifications:chapter-notifications}}).

\sphinxAtStartPar
Default values of payers and payees, projects and persons are used when you create transaction
and import transactions, Also the app uses this values for teamwork on data exchange.

\noindent\sphinxincludegraphics[width=0.250\linewidth]{{directories-055-accounts-continue}.png}

\noindent\sphinxincludegraphics[width=0.250\linewidth]{{directories-060-categories}.png}

\noindent\sphinxincludegraphics[width=0.250\linewidth]{{directories-070-contractors}.png}


\section{Categories}
\label{\detokenize{directories:categories}}
\sphinxAtStartPar
\DUrole{bbmeta}{Categories} directory plays the main role for the classification of transactions.
A category may a set of options,
\DUrole{bbproperty}{Revenue}, \DUrole{bbproperty}{Expense}, \DUrole{bbproperty}{Totaling (summary)},
technical, \DUrole{bbproperty}{Eliminable} and  \DUrole{bbproperty}{Archived}.

\sphinxAtStartPar
Categories sorting order under transaction editing depends on
\DUrole{bbproperty}{Revenue} and \DUrole{bbproperty}{Expense} options. For a revenue transaction
revenue categories are placed at the beginning and then expense ones placed and vice versa.

\sphinxAtStartPar
A category may be neither revenue nor expense. In that case the category is technical. For instance
technical category is used for credit limit changing. For that transaction there is no money turnovers
for a card owner but nevertheless balance is changed. See more details about credit limit at
\sphinxhref{http://qa.bbmoney.biz/en/index.php?qa=27\&qa\_1=how-to-setup-credit-limit-for-new-or-existed-account}{questions and answers (How to setup credit limit for new or existed account?)}.

\sphinxAtStartPar
Since a category has \DUrole{bbproperty}{Totaling (summary)} option you can use \DUrole{bbmeta}{Debts and credits} and \DUrole{bbmeta}{Plan implementation}
reports to get there balance.

\sphinxAtStartPar
Sometimes you need to eliminate transactions from revenues and expenses. Usually it is
transfer transactions. Use categories with \DUrole{bbproperty}{Eliminable} option for them.
The app has standalone totals at the main screen and transactions list for transfers and other transaction with
categories with \DUrole{bbproperty}{Eliminable} option.

\sphinxAtStartPar
The app uses key phrases to find an item when importing transactions. You can set several
comma separated key phrases.

\sphinxAtStartPar
It is possible to define several categories for a transaction.

\sphinxAtStartPar
Once the app installed the directory of categories has default items. It’s up to you edit, add, or delete them.


\section{Payers and Payees}
\label{\detokenize{directories:payers-and-payees}}
\sphinxAtStartPar
Contrary part of transaction is payer or payee. This is often called a contractor. Only transfer transactions
have no contractor. Bat all other transactions have. Transaction have only one contractor.


\section{Projects}
\label{\detokenize{directories:projects}}
\sphinxAtStartPar
You can use projects to account vacations, startups, housing projects and so on.
Transaction may have several projects.

\sphinxAtStartPar
The app uses key phrases to find an item when importing transactions. You can set several
comma separated key phrases.


\section{Persons}
\label{\detokenize{directories:persons}}
\sphinxAtStartPar
You can use persons to account family members, company staffers and so on.
Transaction may have several projects.

\sphinxAtStartPar
The app uses key phrases to find an item when importing transactions. You can set several
comma separated key phrases.

\noindent\sphinxincludegraphics[width=0.250\linewidth]{{directories-080-projects}.png}

\noindent\sphinxincludegraphics[width=0.250\linewidth]{{directories-090-persons}.png}

\noindent\sphinxincludegraphics[width=0.250\linewidth]{{directories-100-currencies}.png}


\section{Currencies}
\label{\detokenize{directories:currencies}}
\sphinxAtStartPar
Once the app installed it contains almost all world currencies. Of course you can add a new one.

\sphinxAtStartPar
Currency rates are used for calculate financial highlights. You can set rates manually
or load from internet resources. Available sources are European Central Bank,
Russian Central Bank (currencies and metals), Bank of Canada, National Bank of the Republic of Belarus,
National Bank of the Republic of Kazakhstan, Bank of Israel, BitPay (BTC rates), Poloniex (cryptocurrencies trading market).

\sphinxAtStartPar
Let author know if you need more, see {\hyperref[\detokenize{preface:section-feedback}]{\sphinxcrossref{\DUrole{std,std-ref}{Feedback}}}} (\autopageref*{\detokenize{preface:section-feedback}}).

\sphinxstepscope


\chapter{Transactions}
\label{\detokenize{transactions:transactions}}\label{\detokenize{transactions:chapter-transactions}}\label{\detokenize{transactions::doc}}

\section{Introduction}
\label{\detokenize{transactions:introduction}}
\sphinxAtStartPar
Transactions are used to account any changes in funds. So use a transaction when you need to
enter an initial balance, change a credit limit, store crediting or debiting funds,
store cash withdrawal at ATM, or something else. This approach is most versatile. The history of
all movements will be stored due to that approach and you will be able to recall any transaction
you need.

\noindent\sphinxincludegraphics[width=0.250\linewidth]{{transactions-010-transactions}.png}

\noindent\sphinxincludegraphics[width=0.250\linewidth]{{transactions-015-transactions-bottom-sheet-opening}.png}

\noindent\sphinxincludegraphics[width=0.250\linewidth]{{transactions-016-transactions-bottom-sheet-open}.png}

\sphinxAtStartPar
You can use filters and fast time range selection at the transactions list.

\noindent\sphinxincludegraphics[width=0.250\linewidth]{{transactions-017-transactions-dates-range-swipe}.png}

\noindent\sphinxincludegraphics[width=0.250\linewidth]{{transactions-018-transactions-dates-range-spinner}.png}

\noindent\sphinxincludegraphics[width=0.250\linewidth]{{transactions-020-transaction}.png}

\sphinxAtStartPar
Transaction has to be one of revenue or expense. There is no special option just put positive
or negative amount. For transfer use categories with \DUrole{bbproperty}{Eliminable} option.
Since installed the app contains \DUrole{bbitem}{Transfer} category you may apply to.

\sphinxAtStartPar
For a foreign transaction you should put a currency and it’s rate. This rate can be different from
a rate stored in the \DUrole{bbmeta}{Currencies} directory. The app by itself calculates currency and rate
for transactions imported from SMS and push notifications.

\sphinxAtStartPar
Use transaction dimensions, categories, payers, payees, projects, and persons to get a comprehensive
funds accounting.


\section{Splits}
\label{\detokenize{transactions:splits}}
\sphinxAtStartPar
You can divide transaction for details. It is often called as make a {\hyperref[\detokenize{glossary:term-split}]{\sphinxtermref{\DUrole{xref,std,std-term}{split}}}}.
When you have a check in a supermarket it is convenient to make a split to store
food costs, household goods costs, and so on. Of course it is far from the only case.

\sphinxAtStartPar
The app always calculates first part of a split by itself. Just put amounts of others.
Put zero amount to remove redundant part.

\noindent\sphinxincludegraphics[width=0.250\linewidth]{{transactionsplit-010-select-transaction}.png}

\noindent\sphinxincludegraphics[width=0.250\linewidth]{{transactionsplit-020-transaction-details}.png}

\noindent\sphinxincludegraphics[width=0.250\linewidth]{{transactionsplit-030-transaction-edit-detail}.png}

\noindent\sphinxincludegraphics[width=0.250\linewidth]{{transactionsplit-040-transaction-details-row-second}.png}

\noindent\sphinxincludegraphics[width=0.250\linewidth]{{transactionsplit-050-transaction-details-row-first}.png}


\section{Planned Transactions}
\label{\detokenize{transactions:planned-transactions}}
\sphinxAtStartPar
Transactions are one of actual or planned. The option \DUrole{bbproperty}{Planned} is used
for a planned transaction. The app takes into account planned transaction until they have expired.
A date and time of a transaction is a key for expiration. You can plan any funds movement,
expenses, revenues, debts, credits, and so on. Reports will help you to
compare actuals and plans.

\noindent\sphinxincludegraphics[width=0.250\linewidth]{{transactionplan-010-transaction-set-plan}.png}


\section{Manual Transfers}
\label{\detokenize{transactions:manual-transfers}}
\sphinxAtStartPar
The app stores transfer within two transactions. There is a fast and convenient way to make
a transfer from a transaction card.
\begin{enumerate}
\sphinxsetlistlabels{\arabic}{enumi}{enumii}{}{.}%
\item {} 
\sphinxAtStartPar
Make a new transaction and put an amount.

\item {} 
\sphinxAtStartPar
Select the \DUrole{bbbutton}{Transfer} button near the source account.

\item {} 
\sphinxAtStartPar
Select target account and the app will make the rest.

\item {} 
\sphinxAtStartPar
Edit other options if you want to.

\item {} 
\sphinxAtStartPar
Save the target transaction.

\item {} 
\sphinxAtStartPar
You will see transfers amount at the main screen.

\end{enumerate}

\noindent\sphinxincludegraphics[width=0.250\linewidth]{{transactionstransfer-010-create-transaction}.png}

\noindent\sphinxincludegraphics[width=0.250\linewidth]{{transactionstransfer-020-transaction-edit}.png}

\noindent\sphinxincludegraphics[width=0.250\linewidth]{{transactionstransfer-030-transaction-select-transfer}.png}

\noindent\sphinxincludegraphics[width=0.250\linewidth]{{transactionstransfer-040-transaction-select-transfer-account}.png}

\noindent\sphinxincludegraphics[width=0.250\linewidth]{{transactionstransfer-045-transaction-select-transfer-account}.png}

\noindent\sphinxincludegraphics[width=0.250\linewidth]{{transactionstransfer-050-transaction-edit-save}.png}

\noindent\sphinxincludegraphics[width=0.250\linewidth]{{transactionstransfer-070-transfer-result}.png}

\sphinxAtStartPar
At this moment source and target transactions are not connected to each other.
Do not forget to edit both ones in future.


\section{Recurring Transactions}
\label{\detokenize{transactions:recurring-transactions}}
\sphinxAtStartPar
Many transactions happen with some frequency. Usually recurring transactions
are planned but sometimes actual too.

\sphinxAtStartPar
You can establish a custom frequency for recurring transactions.

\noindent\sphinxincludegraphics[width=0.250\linewidth]{{recurringtransactions-010-select-directories}.png}

\noindent\sphinxincludegraphics[width=0.250\linewidth]{{recurringtransactions-020-select-recurring-transactions}.png}

\noindent\sphinxincludegraphics[width=0.250\linewidth]{{recurringtransactions-030-select-transaction}.png}

\noindent\sphinxincludegraphics[width=0.250\linewidth]{{recurringtransactions-040-reccuring-transaction}.png}

\sphinxstepscope


\chapter{Account Settings for Notifications Import}
\label{\detokenize{account-identities:account-settings-for-notifications-import}}\label{\detokenize{account-identities:chapter-account-identities}}\label{\detokenize{account-identities::doc}}

\section{Identity Choosing}
\label{\detokenize{account-identities:identity-choosing}}
\sphinxAtStartPar
You have to put an identity at the account card before import SMS or push notifications.
This will ensure detection of an account for a transaction. Usually financial institutions
put last four digits of a card number to a notification. So use them as a card identity.

\sphinxAtStartPar
For a example, for the SMS

\begin{sphinxVerbatim}[commandchars=\\\{\}]
\PYG{n}{VISA1234}\PYG{p}{:} \PYG{l+m+mf}{08.08}\PYG{l+m+mf}{.13} \PYG{l+m+mi}{14}\PYG{p}{:}\PYG{l+m+mi}{05} \PYG{n}{payment} \PYG{l+m+mf}{500.00} \PYG{n}{USD}\PYG{o}{.} \PYG{n}{balance} \PYG{l+m+mf}{1000.00} \PYG{n}{USD}\PYG{o}{.}
\end{sphinxVerbatim}

\sphinxAtStartPar
you should choose VISA1234 as a card identity. Some financial institutions do not put
digits of an account or card number into notifications. For example, in the SMS

\begin{sphinxVerbatim}[commandchars=\\\{\}]
\PYG{n}{Transaction} \PYG{o}{\PYGZgt{}\PYGZgt{}} \PYG{o}{\PYGZhy{}}\PYG{l+m+mi}{600} \PYG{n}{USD}\PYG{o}{.}       \PYG{n}{Atm}\PYG{o}{\PYGZhy{}}\PYG{n}{nyc}\PYG{o}{\PYGZhy{}}\PYG{l+m+mi}{001}
\end{sphinxVerbatim}

\sphinxAtStartPar
there is no way to find out an identity. Well, in that case you should use
sender name or number. For example, short number for Sberbank is 900.
For push notifications sender is a package identity. For example,
ru.rocketbank.r2d2 is the package identity for RocketBank.

\sphinxAtStartPar
Open the card of an account in order to setup identity. Press \DUrole{bbspinner}{Account number or card ID} and
select identity from a financial institution message. Put the identity by hands
if you want to use sender or package identity.

\sphinxAtStartPar
Also do not forget to select an import tune for your financial institution.

\noindent\sphinxincludegraphics[width=0.250\linewidth]{{accountidenties-005-select-references}.png}

\noindent\sphinxincludegraphics[width=0.250\linewidth]{{accountidenties-010-select-accounts}.png}

\noindent\sphinxincludegraphics[width=0.250\linewidth]{{accountidenties-020-open-card-account}.png}

\noindent\sphinxincludegraphics[width=0.250\linewidth]{{accountidenties-030-scroll-to-identity}.png}

\noindent\sphinxincludegraphics[width=0.250\linewidth]{{accountidenties-035-select-identity}.png}

\noindent\sphinxincludegraphics[width=0.250\linewidth]{{accountidenties-040-set-identity}.png}


\section{Key Phrase Choosing for Transfers}
\label{\detokenize{account-identities:key-phrase-choosing-for-transfers}}
\sphinxAtStartPar
The app Budget Blitz for Android can create transfer transactions based upon financial institution messages. For example, when you
have an SMS

\begin{sphinxVerbatim}[commandchars=\\\{\}]
\PYG{n}{VISA1234}\PYG{p}{:} \PYG{l+m+mf}{08.08}\PYG{l+m+mf}{.13} \PYG{l+m+mi}{14}\PYG{p}{:}\PYG{l+m+mi}{05} \PYG{n}{cash} \PYG{n}{withdrawal} \PYG{l+m+mf}{200.00} \PYG{n}{USD}\PYG{o}{.} \PYG{n}{ATM} \PYG{l+m+mi}{10010001} \PYG{n}{bal} \PYG{l+m+mf}{500.00} \PYG{n}{USD}\PYG{o}{.}
\end{sphinxVerbatim}

\sphinxAtStartPar
then the app is able to create expense transaction for the VISA1234 account and revenue transaction
for a cash account. All you need is to set key phrases for the cash account. The app will
use this key phrases to find out the one. For example, key phrases above may be one of “cash withdrawal” or
“ATM”.

\begin{sphinxadmonition}{note}{Note:}
\sphinxAtStartPar
It is also necessary to ensure the app is able to identify a transaction as transfer, see {\hyperref[\detokenize{notifications:chapter-notifications}]{\sphinxcrossref{\DUrole{std,std-ref}{Advanced Import SMS and Push Notifications Setting}}}} (\autopageref*{\detokenize{notifications:chapter-notifications}}).
\end{sphinxadmonition}

\sphinxAtStartPar
Open the card of an account in order to setup key phrases. Press \DUrole{bbspinner}{Key phrases} and
select ones from a financial institution message. Also put key phrases by hands
if you want to.

\noindent\sphinxincludegraphics[width=0.250\linewidth]{{accountidenties-050-open-cash-account}.png}

\noindent\sphinxincludegraphics[width=0.250\linewidth]{{accountidenties-060-scroll-to-keywords}.png}

\noindent\sphinxincludegraphics[width=0.250\linewidth]{{accountidenties-070-set-keywords}.png}

\sphinxAtStartPar
Usually accounts having notifications have empty \DUrole{bbproperty}{Keywords} property and cash accounts
have empty \DUrole{bbproperty}{Number} property vice versa. But there are rare cases when both ones are used. See \sphinxhref{http://qa.bbmoney.biz/ru/index.php?qa=67\&qa\_1=\%D0\%BA\%D0\%B0\%D0\%BA-\%D0\%BD\%D0\%B0\%D1\%81\%D1\%82\%D1\%80\%D0\%BE\%D0\%B8\%D1\%82\%D1\%8C-\%D0\%B8\%D0\%BC\%D0\%BF\%D0\%BE\%D1\%80\%D1\%82-\%D1\%83\%D0\%B2\%D0\%B5\%D0\%B4\%D0\%BE\%D0\%BC\%D0\%BB\%D0\%B5\%D0\%BD\%D0\%B8\%D0\%B9-\%D1\%80\%D0\%BE\%D0\%BA\%D0\%B5\%D1\%82\%D0\%B1\%D0\%B0\%D0\%BD\%D0\%BA\%D0\%B0\&show=68\#a68}{Rocketbank}
notifications import setting.

\sphinxstepscope


\chapter{Advanced Import SMS and Push Notifications Setting}
\label{\detokenize{notifications:advanced-import-sms-and-push-notifications-setting}}\label{\detokenize{notifications:chapter-notifications}}\label{\detokenize{notifications::doc}}

\section{Notifications Detection Algorithm}
\label{\detokenize{notifications:notifications-detection-algorithm}}
\sphinxAtStartPar
The import tune ensures the process of SMS and push notification detection. Exactly import
tune controls type of transaction, revenue, expense or transfer, is balance required or not,
and so on.

\sphinxAtStartPar
You can see the algorithm of notifications detection at the picture below.

\noindent{\hspace*{\fill}\sphinxincludegraphics[width=0.750\linewidth]{{sms-import-algorithm-en}.png}\hspace*{\fill}}

\sphinxAtStartPar
When new notification arrived then the app tries to detect an account. It uses identities from the
\DUrole{bbmeta}{Accounts} directory. Since single account found the app loads connected import tune.

\sphinxAtStartPar
Further the app classifies transaction, revenue, expense, or transfer, based on the import tune.
While transaction is transfer the app tries to find a target account. Now it uses key phrases from the
\DUrole{bbmeta}{Accounts} directory. Since single target account found the app makes a target transaction
in order to complete transfer.

\sphinxAtStartPar
Next stage is to select dimensions. The app tries to find a category, payer or payee, project, person
based on there key phrases. Default values is used when no value found out.

\sphinxAtStartPar
Finally the app calculates amount and balance. Additional commission or correction
transaction can be made or currency rate calculated when balance from notification is not equal to the app one.
It depends on a context and transaction currency.

\sphinxAtStartPar
Sometimes notifications arrive in a wrong order not like a real transactions done. The app
creates balance correction transactions in that case until the order becomes correct.
The app will remove redundant corrections as far as possible after the order becomes correct.

\sphinxAtStartPar
Example
\begin{enumerate}
\sphinxsetlistlabels{\arabic}{enumi}{enumii}{}{.}%
\item {} 
\sphinxAtStartPar
13.04.2016, 10:00, balance = 1000 USD

\end{enumerate}

\sphinxAtStartPar
The app got messages, sequence is invalid, correct one is 4, 3, 5, 2.
\begin{enumerate}
\sphinxsetlistlabels{\arabic}{enumi}{enumii}{}{.}%
\setcounter{enumi}{1}
\item {} 
\sphinxAtStartPar
13.04.2016, 15:00, expense = \sphinxhyphen{}50 USD, balance = 500 USD, \(\rightarrow\) automatic correction = \sphinxhyphen{}450 USD

\item {} 
\sphinxAtStartPar
13.04.2016, 15:05, expense = \sphinxhyphen{}90 USD, balance = 800 USD, \(\rightarrow\) automatic correction = +390 USD

\item {} 
\sphinxAtStartPar
13.04.2016, 15:10, expense = \sphinxhyphen{}110 USD, balance = 890 USD, \(\rightarrow\) automatic correction = +200 USD

\item {} 
\sphinxAtStartPar
13.04.2016, 15:15, expense = \sphinxhyphen{}250 USD, balance = 550 USD, \(\rightarrow\) automatic correction = \sphinxhyphen{}90 USD

\end{enumerate}

\sphinxAtStartPar
The app got message that starts correct sequence.
\begin{enumerate}
\sphinxsetlistlabels{\arabic}{enumi}{enumii}{}{.}%
\setcounter{enumi}{5}
\item {} 
\sphinxAtStartPar
13.04.2016, 15:20, expense = \sphinxhyphen{}100 USD, balance = 400 USD, \(\rightarrow\) automatic correction  = 0 USD, automatic corrections  2 — 5 removed

\end{enumerate}


\section{New Custom Import Tune}
\label{\detokenize{notifications:new-custom-import-tune}}
\sphinxAtStartPar
At this moment Budget Blitz for Android has more than 160 ready to use import tune for world wide financial institutions.
It is not to match of course. But you can create an import tune by yourself with little effort.
It is very easy to do.

\noindent\sphinxincludegraphics[width=0.250\linewidth]{{notificationsimporttunes-010-select-directories}.png}

\noindent\sphinxincludegraphics[width=0.250\linewidth]{{notificationsimporttunes-020-select-import-tunes}.png}

\noindent\sphinxincludegraphics[width=0.250\linewidth]{{notificationsimporttunes-030-select-new}.png}

\sphinxAtStartPar
\DUrole{bbproperty}{Name} for new setting can be different. It would be nice to make a name th same as
financial institution.

\sphinxAtStartPar
\DUrole{bbproperty}{Restriction by sender} is only used in quite unique cases when the app is not able
to identify account. That restriction fires before the app looking for an account by identity making list
of accounts shorter.

\sphinxAtStartPar
Let the app has two accounts, for example
\begin{enumerate}
\sphinxsetlistlabels{\arabic}{enumi}{enumii}{}{.}%
\item {} 
\sphinxAtStartPar
RocketBank, the identity is ru.rocketbank.r2d2, the import tune is RocketBank;

\item {} 
\sphinxAtStartPar
VTB, the identity is ***1234, the import tune is VTB.

\end{enumerate}

\sphinxAtStartPar
RocketBank, the sender is ru.rocketbank.r2d2, sends notification about revenue as

\begin{sphinxVerbatim}[commandchars=\\\{\}]
Transaction \PYGZgt{}\PYGZgt{} +1 800 USD.
Source card is «VTB ***1234»
\end{sphinxVerbatim}

\sphinxAtStartPar
There is no identity in this notification but there is the transfer source account number. Without
restriction by sender the app can not find the RocketBank account, because both accounts
RocketBank and VTB are suitable.

\sphinxAtStartPar
Since the restriction is on, the app finds RocketBank import tune by sender ru.rocketbank.r2d2.
Only RocketBank account uses that setting, so the app selects RocketBank account correctly.

\sphinxAtStartPar
The basic options of the import are established by key phrases. An option may have one or more
comma separated key phrases.

\sphinxAtStartPar
\DUrole{bbproperty}{Revenue and expense key phrases} define transaction sign. The import is not possible when
sign is undefined.

\sphinxAtStartPar
\DUrole{bbproperty}{Transfer key phrases} indicate to the app two transactions instead one required.
The transfer direction depends on the transaction sign.

\sphinxAtStartPar
Let the setup be, for example, as:
\begin{enumerate}
\sphinxsetlistlabels{\arabic}{enumi}{enumii}{}{.}%
\item {} 
\sphinxAtStartPar
Revenues key phrases: “cash deposits,credit”

\item {} 
\sphinxAtStartPar
Transfers key phrases: “cash deposits”

\item {} 
\sphinxAtStartPar
Account Card identity: Visa2900

\item {} 
\sphinxAtStartPar
Account Cash key phrases: “ATM”

\end{enumerate}

\sphinxAtStartPar
Bank sends notification:

\begin{sphinxVerbatim}[commandchars=\\\{\}]
\PYG{n}{Card} \PYG{n}{Visa2900}\PYG{o}{.} \PYG{n}{Cash} \PYG{n}{deposits} \PYG{l+m+mf}{200.00} \PYG{n}{USD} \PYG{n}{ATM}\PYG{o}{.} \PYG{n}{Balance}\PYG{p}{:} \PYG{l+m+mf}{2740.26} \PYG{n}{USD}\PYG{o}{.} \PYG{l+m+mi}{25}\PYG{o}{/}\PYG{l+m+mi}{03}\PYG{o}{/}\PYG{l+m+mi}{14}\PYG{p}{,}\PYG{l+m+mi}{15}\PYG{p}{:}\PYG{l+m+mi}{00}\PYG{p}{:}\PYG{l+m+mf}{00.}
\end{sphinxVerbatim}

\sphinxAtStartPar
As a result, the app will create two transactions:
\begin{enumerate}
\sphinxsetlistlabels{\arabic}{enumi}{enumii}{}{.}%
\item {} 
\sphinxAtStartPar
Revenue transaction for Card account;

\item {} 
\sphinxAtStartPar
Expense transaction for Cash account.

\end{enumerate}

\noindent\sphinxincludegraphics[width=0.250\linewidth]{{notificationsimporttunes-040-import-tune}.png}

\noindent\sphinxincludegraphics[width=0.250\linewidth]{{notificationsimporttunes-050-import-tune-2}.png}

\sphinxAtStartPar
Sometimes certain notifications have a balance and certain have not. Special
key phrases help the app to understand when is case to calculate balance and when is not.

\sphinxAtStartPar
Sometimes notification is for information only but contains revenue or expense key phrases.
\DUrole{bbproperty}{Skip transaction} key phrases makes possible to cancel import.

\sphinxAtStartPar
Example
\begin{enumerate}
\sphinxsetlistlabels{\arabic}{enumi}{enumii}{}{.}%
\item {} 
\sphinxAtStartPar
Revenues key phrases: “cash deposits,credit”

\item {} 
\sphinxAtStartPar
Skip transaction key phrases: “error”

\item {} 
\sphinxAtStartPar
Account Card identity: Visa2900

\end{enumerate}

\sphinxAtStartPar
Bank sends notification:

\begin{sphinxVerbatim}[commandchars=\\\{\}]
\PYG{n}{Card} \PYG{n}{Visa2900}\PYG{o}{.} \PYG{n}{Cash} \PYG{n}{deposits} \PYG{l+m+mf}{200.00} \PYG{n}{USD} \PYG{n}{ATM}\PYG{o}{.} \PYG{n}{Balance}\PYG{p}{:} \PYG{l+m+mf}{2740.26} \PYG{n}{USD}\PYG{o}{.} \PYG{n}{An} \PYG{n}{error} \PYG{n}{occurred}\PYG{o}{.} \PYG{l+m+mi}{25}\PYG{o}{/}\PYG{l+m+mi}{03}\PYG{o}{/}\PYG{l+m+mi}{14}\PYG{p}{,}\PYG{l+m+mi}{15}\PYG{p}{:}\PYG{l+m+mi}{00}\PYG{p}{:}\PYG{l+m+mf}{00.}
\end{sphinxVerbatim}

\sphinxAtStartPar
As result, the app will not make a transaction. And it is the case, because ATM have made a money back not
a cash deposits.

\sphinxAtStartPar
\DUrole{bbproperty}{Amount position among digital values} is the most probable place of the amount. Final
decision is up to the app.

\sphinxAtStartPar
\DUrole{bbproperty}{Balance position among digital values} is the most probable place of the balance. Final
decision is up to the app too.

\sphinxAtStartPar
Put \sphinxhyphen{}1 when where is no balance in notifications at all.

\sphinxAtStartPar
The app skips all notification without balance when \DUrole{bbproperty}{Balance position among digital values} is not equal \sphinxhyphen{}1.
But you can specify key phrase to underline when is app have to expect balance and when have not to.

\sphinxAtStartPar
Transaction amount and balance are used to calculate currency rate, commission, and correction.

\sphinxAtStartPar
Near the money amount of transaction should be placed a currency. Left of or right of, it does not matter.
Currency names and keywords are the glue for the app find it out.

\sphinxAtStartPar
Certain financial institutions not always put currency in notifications. Use
\DUrole{bbproperty}{Payment currency may be omitted sometimes} in that case and the app will use the currency of account.

\sphinxstepscope


\chapter{Data Import}
\label{\detokenize{import:data-import}}\label{\detokenize{import:chapter-import}}\label{\detokenize{import::doc}}

\section{Notification Import Tunes}
\label{\detokenize{import:notification-import-tunes}}
\sphinxAtStartPar
Notification import tunes have the main role in the notifications import. Once
financial institution changes notification structure the import tunes should be changed too.
For that case you can download update or modify tunes by yourself, see chapter {\hyperref[\detokenize{notifications:chapter-notifications}]{\sphinxcrossref{\DUrole{std,std-ref}{Advanced Import SMS and Push Notifications Setting}}}} (\autopageref*{\detokenize{notifications:chapter-notifications}}).

\noindent\sphinxincludegraphics[width=0.250\linewidth]{{updateimporttunes-010-select-actions}.png}

\noindent\sphinxincludegraphics[width=0.250\linewidth]{{updateimporttunes-020-select-import}.png}

\noindent\sphinxincludegraphics[width=0.250\linewidth]{{updateimporttunes-030-select-import-sms-tunes}.png}

\sphinxAtStartPar
Select menu item \sphinxmenuselection{Actions \(\rightarrow\) Import \(\rightarrow\) SMS and Push tunes} to get updates.

\noindent\sphinxincludegraphics[width=0.250\linewidth]{{updateimporttunes-040-select-import-tunes-updated}.png}

\noindent\sphinxincludegraphics[width=0.250\linewidth]{{updateimporttunes-050-select-import-tunes-new}.png}

\noindent\sphinxincludegraphics[width=0.250\linewidth]{{updateimporttunes-060-select-import-tunes-no-updates}.png}

\sphinxAtStartPar
The app will show available update, also it is possible to download new ones here.

\noindent\sphinxincludegraphics[width=0.250\linewidth]{{updateimporttunes-070-select-actions}.png}

\noindent\sphinxincludegraphics[width=0.250\linewidth]{{updateimporttunes-080-select-active_profile}.png}

\noindent\sphinxincludegraphics[width=0.250\linewidth]{{updateimporttunes-085-select-notifications}.png}

\noindent\sphinxincludegraphics[width=0.250\linewidth]{{updateimporttunes-090-check-use_exchange_when_wifi}.png}

\sphinxAtStartPar
But may be you will see nothing. Check the import tunes exchange is on at settings.


\section{SMS and Push Notifications}
\label{\detokenize{import:sms-and-push-notifications}}
\sphinxAtStartPar
The app Budget Blitz for Android imports SMS and push notifications by default. But it is possible to import
certain notification by hands. To do that
\begin{enumerate}
\sphinxsetlistlabels{\arabic}{enumi}{enumii}{}{.}%
\item {} 
\sphinxAtStartPar
Open the import dialog.

\item {} 
\sphinxAtStartPar
Select a required account. The account should have the identity and the import tune.

\item {} 
\sphinxAtStartPar
Select required notifications.

\item {} 
\sphinxAtStartPar
Press \DUrole{bbbutton}{Import} button.

\item {} 
\sphinxAtStartPar
Check transactions list for the result.

\item {} 
\sphinxAtStartPar
Use event log to view issues.

\end{enumerate}

\noindent\sphinxincludegraphics[width=0.250\linewidth]{{manualsmsimport-010-select-actions}.png}

\noindent\sphinxincludegraphics[width=0.250\linewidth]{{manualsmsimport-020-select-import}.png}

\noindent\sphinxincludegraphics[width=0.250\linewidth]{{manualsmsimport-030-select-import-sms}.png}

\noindent\sphinxincludegraphics[width=0.250\linewidth]{{manualsmsimport-040-select-account}.png}

\noindent\sphinxincludegraphics[width=0.250\linewidth]{{manualsmsimport-050-move-next}.png}

\noindent\sphinxincludegraphics[width=0.250\linewidth]{{manualsmsimport-060-import-sms}.png}

\noindent\sphinxincludegraphics[width=0.250\linewidth]{{manualsmsimport-070-view-transactions}.png}

\noindent\sphinxincludegraphics[width=0.250\linewidth]{{manualsmsimport-080-view-events}.png}


\section{CSV files}
\label{\detokenize{import:csv-files}}
\sphinxAtStartPar
During \sphinxhref{https://en.wikipedia.org/wiki/CSV}{CSV} file import the app can create new accounts, categories, payers or payees,
projects, and persons. It depends on your choice.

\sphinxAtStartPar
The column separator can be one of  “;”, “,”, “|”, “/”, “". File must be UTF\sphinxhyphen{}8 encoded.

\sphinxAtStartPar
The first row of the file must have column names, case does not matter.
Since column names are placed at another row they are valid for next rows.


\begin{savenotes}\sphinxattablestart
\centering
\sphinxcapstartof{table}
\sphinxthecaptionisattop
\sphinxcaption{CSV file format}\label{\detokenize{import:id1}}
\sphinxaftertopcaption
\begin{tabular}[t]{|\X{7}{42}|\X{5}{42}|\X{30}{42}|}
\hline
\sphinxstyletheadfamily 
\sphinxAtStartPar
Names
&\sphinxstyletheadfamily 
\sphinxAtStartPar
Mandatory
&\sphinxstyletheadfamily 
\sphinxAtStartPar
Comment
\\
\hline
\sphinxAtStartPar
id
&
\sphinxAtStartPar
No
&
\sphinxAtStartPar
Transaction identity, the app will search existed transaction if not empty.
\\
\hline
\sphinxAtStartPar
account,incomeAccountName, Income account
&
\sphinxAtStartPar
Yes
&
\sphinxAtStartPar
Name, number, or identity of the account
\\
\hline
\sphinxAtStartPar
date
&
\sphinxAtStartPar
No
&
\sphinxAtStartPar
Date of the transaction, supported formats: “dd’d’MM’d’yyyy” (for example, 01d01d2017), “yyyy’d’MM’d’dd” (for example, 2017d01d01), “yyyyMMddHHmmss”, “yyyyMMddHHmm”, “yyyyMMdd”, “yyyy\sphinxhyphen{}MM\sphinxhyphen{}dd HH:mm:ss”, “yyyy\sphinxhyphen{}MM\sphinxhyphen{}dd HH:mm”, “yyyy\sphinxhyphen{}MM\sphinxhyphen{}dd”, “dd\sphinxhyphen{}MM\sphinxhyphen{}yyyy HH:mm:ss”, “dd\sphinxhyphen{}MM\sphinxhyphen{}yyyy HH:mm”, “dd\sphinxhyphen{}MM\sphinxhyphen{}yyyy”, “dd.MM.yyyy HH:mm:ss”, “dd.MM.yyyy HH:mm”, “dd.MM.yyyy”
\\
\hline
\sphinxAtStartPar
time
&
\sphinxAtStartPar
No
&
\sphinxAtStartPar
Time of the transaction, supported formats: “HH:mm:ss”, “HH:mm”, “HHmmss”, “HHmm”
\\
\hline
\sphinxAtStartPar
amount,income, Income amount
&
\sphinxAtStartPar
Yes
&
\sphinxAtStartPar
Transaction amount, can have a currency and digits delimiters, fixed point should be point or comma, can be an amount in the transaction currency or an amount in the account currency
\\
\hline
\sphinxAtStartPar
original amount
&
\sphinxAtStartPar
No
&
\sphinxAtStartPar
Amount in the currency of the transaction, if specified, the rate of the transaction is calculated automatically
\\
\hline
\sphinxAtStartPar
rate, exchange rate
&
\sphinxAtStartPar
No
&
\sphinxAtStartPar
Transaction rate
\\
\hline
\sphinxAtStartPar
currency,incomeCurrencyShorttitle
&
\sphinxAtStartPar
No
&
\sphinxAtStartPar
Transaction currency or account currency, if not specified, is used in the currency of the account
\\
\hline
\sphinxAtStartPar
original currency
&
\sphinxAtStartPar
No
&
\sphinxAtStartPar
Transaction currency
\\
\hline
\sphinxAtStartPar
payer, payee, contractor
&
\sphinxAtStartPar
No
&
\sphinxAtStartPar
Name of the contractor, the app will analyze current row keywords when empty
\\
\hline
\sphinxAtStartPar
category, categoryName
&
\sphinxAtStartPar
No
&
\sphinxAtStartPar
Name of the category, the app will analyze current row keywords when empty
\\
\hline
\sphinxAtStartPar
project
&
\sphinxAtStartPar
No
&
\sphinxAtStartPar
Name of the project, the app will analyze current row keywords when empty
\\
\hline
\sphinxAtStartPar
person, unit
&
\sphinxAtStartPar
No
&
\sphinxAtStartPar
Name of the person, the app will analyze current row keywords when empty
\\
\hline
\sphinxAtStartPar
location, place
&
\sphinxAtStartPar
No
&
\sphinxAtStartPar
Name of the location, the app will analyze current row keywords when empty
\\
\hline
\sphinxAtStartPar
notes, note
&
\sphinxAtStartPar
No
&
\sphinxAtStartPar
Note
\\
\hline
\sphinxAtStartPar
planned, plan
&
\sphinxAtStartPar
No
&
\sphinxAtStartPar
Actual (0) or planned (1), default value is 0
\\
\hline
\sphinxAtStartPar
detail, split
&
\sphinxAtStartPar
No
&
\sphinxAtStartPar
Transaction (0) or detail of transaction (1). default value is 0
\\
\hline
\sphinxAtStartPar
X
&
\sphinxAtStartPar
X
&
\sphinxAtStartPar
Second transaction from the single line
\\
\hline
\sphinxAtStartPar
outcomeAccountName, Expense account
&
\sphinxAtStartPar
Yes
&
\sphinxAtStartPar
Account
\\
\hline
\sphinxAtStartPar
outcome, Expense amoun
&
\sphinxAtStartPar
Yes
&
\sphinxAtStartPar
Amount
\\
\hline
\sphinxAtStartPar
outcomeCurrencyShorttitle
&
\sphinxAtStartPar
No
&
\sphinxAtStartPar
Currency
\\
\hline
\end{tabular}
\par
\sphinxattableend\end{savenotes}

\sphinxAtStartPar
The row is canceled when mandatory columns are empty.

\sphinxAtStartPar
If the row contains not all mandatory columns, but amount column is not empty, then app creates split transaction. This is like column detail contains value 1.

\sphinxAtStartPar
To start the import
\begin{enumerate}
\sphinxsetlistlabels{\arabic}{enumi}{enumii}{}{.}%
\item {} 
\sphinxAtStartPar
Open the import dialog.

\item {} 
\sphinxAtStartPar
Select a file.

\item {} 
\sphinxAtStartPar
Press \DUrole{bbbutton}{Next} and select required rows.

\item {} 
\sphinxAtStartPar
Press \DUrole{bbbutton}{Import} button.

\item {} 
\sphinxAtStartPar
Check transactions list for the result.

\item {} 
\sphinxAtStartPar
Use event log to view issues.

\end{enumerate}

\noindent\sphinxincludegraphics[width=0.250\linewidth]{{csvimport-030-select-import-csv}.png}

\noindent\sphinxincludegraphics[width=0.250\linewidth]{{csvimport-040-select-file-and-options}.png}


\section{OFX files}
\label{\detokenize{import:ofx-files}}
\sphinxAtStartPar
Budget Blitz for Android supports import of \sphinxhref{https://en.wikipedia.org/wiki/Open\_Financial\_Exchange}{OFX} files meet specification starting from 2.1.1.
\begin{enumerate}
\sphinxsetlistlabels{\arabic}{enumi}{enumii}{}{.}%
\item {} 
\sphinxAtStartPar
Open the import dialog.

\item {} 
\sphinxAtStartPar
Select a file.

\item {} 
\sphinxAtStartPar
Press \DUrole{bbbutton}{Next} and select required rows.

\item {} 
\sphinxAtStartPar
Press \DUrole{bbbutton}{Import} button.

\item {} 
\sphinxAtStartPar
Check transactions list for the result.

\item {} 
\sphinxAtStartPar
Use event log to view issues.

\end{enumerate}

\noindent\sphinxincludegraphics[width=0.250\linewidth]{{ofximport-030-select-import-ofx}.png}

\noindent\sphinxincludegraphics[width=0.250\linewidth]{{ofximport-040-select-file-and-options}.png}


\section{Electronic receipts}
\label{\detokenize{import:electronic-receipts}}
\sphinxAtStartPar
Budget Blitz for Android supports electronic receipts import. Now available import for Russia and Ukraine. You can request more formats by email.

\sphinxAtStartPar
To import from other apps:
\begin{enumerate}
\sphinxsetlistlabels{\arabic}{enumi}{enumii}{}{.}%
\item {} 
\sphinxAtStartPar
Open specific app.

\item {} 
\sphinxAtStartPar
Select receipt, link or text data and push Share / Send / Transmit button or some thing like that.

\item {} 
\sphinxAtStartPar
Select Budget Blitz for Android as receiver.

\item {} 
\sphinxAtStartPar
Follow instructions on a screen.

\end{enumerate}

\sphinxAtStartPar
To import from the clipboard:
\begin{enumerate}
\sphinxsetlistlabels{\arabic}{enumi}{enumii}{}{.}%
\item {} 
\sphinxAtStartPar
Open import dialog.

\item {} 
\sphinxAtStartPar
Paste data from the clipboard.

\item {} 
\sphinxAtStartPar
Follow instructions on a screen.

\end{enumerate}

\sphinxAtStartPar
Use events log to identify errors.

\sphinxstepscope


\chapter{Teamwork}
\label{\detokenize{teamwork:teamwork}}\label{\detokenize{teamwork:chapter-teamwork}}\label{\detokenize{teamwork::doc}}

\section{Introduction}
\label{\detokenize{teamwork:introduction}}
\sphinxAtStartPar
Budget Blitz for Android ensures the collaborative accounting of revenues and expenses. Here are a few examples:
\begin{enumerate}
\sphinxsetlistlabels{\arabic}{enumi}{enumii}{}{.}%
\item {} 
\sphinxAtStartPar
Full synchronization between devices;

\item {} 
\sphinxAtStartPar
Collaborative financial accounting restricted by selected accounts, projects, persons, payers, payees, or even categories;

\item {} 
\sphinxAtStartPar
Collecting data on a single device, in a case of, for example, parents track children expenses.

\end{enumerate}

\sphinxAtStartPar
Any device can become an exchange node, see {\hyperref[\detokenize{glossary:term-exchange-node}]{\sphinxtermref{\DUrole{xref,std,std-term}{exchange node}}}} and receive or transmit changes.
Each exchange node can communicate with other ones.

\begin{sphinxadmonition}{note}{Note:}
\sphinxAtStartPar
Free version can transmit messages only. The Pro version has no restrictions.
\end{sphinxadmonition}

\sphinxAtStartPar
The app has flexible settings to control exchange. For example, you can allow to accept only new transactions
from one node, and forbid modified ones. Each node has own settings.

\sphinxAtStartPar
Messages between nodes are encrypted in order to improve safety. For each node you can specify
own password that will be used for encryption and decryption of a transmitted information.

\sphinxAtStartPar
Collaboration does not require Dropbox account or other ones.


\section{Getting Started}
\label{\detokenize{teamwork:getting-started}}

\subsection{Initial Database}
\label{\detokenize{teamwork:initial-database}}
\sphinxAtStartPar
Suppose that Alice and Bob want to use a collaborative financial accounting.
Then, they need to decide what is the best suited case:
\begin{enumerate}
\sphinxsetlistlabels{\arabic}{enumi}{enumii}{}{.}%
\item {} 
\sphinxAtStartPar
At the beginning Alice and Bob will have similar database.

\item {} 
\sphinxAtStartPar
Alice and Bob already use the app, they do not want to combine their databases, and plan to synchronize selected accounts only.

\end{enumerate}

\sphinxAtStartPar
For the first case Alice or Bob, let it be Alice, makes a backup. Further, Alice gives the
backup to Bob and he restores database on his device. Now Alice and Bob have similar database.
The teamwork requires databases have different identities. Hence, Bob generates new identity
for his database.

\begin{sphinxadmonition}{note}{Note:}
\sphinxAtStartPar
Ii is required to make a new database identity when database restored from backup of another teamwork member.
\end{sphinxadmonition}

\noindent\sphinxincludegraphics[width=0.250\linewidth]{{exchangenewid-005-select-actions}.png}

\noindent\sphinxincludegraphics[width=0.250\linewidth]{{exchangenewid-010-select-exchange}.png}

\noindent\sphinxincludegraphics[width=0.250\linewidth]{{exchangenewid-020-select-exchange_nodes}.png}

\noindent\sphinxincludegraphics[width=0.250\linewidth]{{exchangenewid-030-select-actions}.png}

\noindent\sphinxincludegraphics[width=0.250\linewidth]{{exchangenewid-040-select-new-id}.png}

\sphinxAtStartPar
Now Alice and Bob are ready for next steps.

\sphinxAtStartPar
In the second case there is no need to preliminary actions. Alice and Bob are ready for next steps at once.


\subsection{Identities Interchange}
\label{\detokenize{teamwork:identities-interchange}}
\sphinxAtStartPar
The most important step under preparing teamwork is an identities interchange. To do that
Alice opens \DUrole{bbmeta}{Exchange nodes} directory using \sphinxmenuselection{Actions \(\rightarrow\) Data synchronization \(\rightarrow\)  Exchange nodes} menu. Then,
Alice presses \sphinxmenuselection{Share ID} and sends identity to Bob by email.

\noindent\sphinxincludegraphics[width=0.250\linewidth]{{exchangesendid-005-select-actions}.png}

\noindent\sphinxincludegraphics[width=0.250\linewidth]{{exchangesendid-010-select-exchange}.png}

\noindent\sphinxincludegraphics[width=0.250\linewidth]{{exchangesendid-020-select-exchange_nodes}.png}

\noindent\sphinxincludegraphics[width=0.250\linewidth]{{exchangesendid-030-select-actions}.png}

\noindent\sphinxincludegraphics[width=0.250\linewidth]{{exchangesendid-040-select-share-id}.png}

\noindent\sphinxincludegraphics[width=0.250\linewidth]{{exchangesendid-050-mail-id}.png}

\sphinxAtStartPar
Bob receives the message, creates new exchange node putting the name and identity from the message.
After that he sends his own identity to Alice.

\noindent\sphinxincludegraphics[width=0.250\linewidth]{{exchangenewnode-005-select-actions}.png}

\noindent\sphinxincludegraphics[width=0.250\linewidth]{{exchangenewnode-010-select-exchange}.png}

\noindent\sphinxincludegraphics[width=0.250\linewidth]{{exchangenewnode-020-select-exchange_nodes}.png}

\noindent\sphinxincludegraphics[width=0.250\linewidth]{{exchangenewnode-030-click-fab}.png}

\noindent\sphinxincludegraphics[width=0.250\linewidth]{{exchangenewnode-040-setup_node}.png}

\sphinxAtStartPar
Now, there is a turn of Alice to receive Bob message. She creates new node and puts identity
from the message of Bob.


\section{Setting Teamwork On}
\label{\detokenize{teamwork:setting-teamwork-on}}
\sphinxAtStartPar
When identities interchange completed, Alice and Bob set synchronization on at the app settings.

\noindent\sphinxincludegraphics[width=0.250\linewidth]{{exchangeenable-005-select-actions}.png}

\noindent\sphinxincludegraphics[width=0.250\linewidth]{{exchangeenable-020-click-on-actions-menu}.png}

\noindent\sphinxincludegraphics[width=0.250\linewidth]{{exchangeenable-025-select-exchange}.png}

\noindent\sphinxincludegraphics[width=0.250\linewidth]{{exchangeenable-030-check-use-exchange}.png}

\sphinxAtStartPar
Now the app sends all changes from the database of Alice to the database of Bob and vice versa.
The app synchronizes changes every five minutes when Wi\sphinxhyphen{}Fi or mobile network is on. Synchronization
is off when device falls asleep or network is off. This ensures to save network traffic and
battery power.

\sphinxAtStartPar
This is how exchange works:
\begin{enumerate}
\sphinxsetlistlabels{\arabic}{enumi}{enumii}{}{.}%
\item {} 
\sphinxAtStartPar
Since exchange started the app checks whether screen is on or not.
\begin{enumerate}
\sphinxsetlistlabels{\arabic}{enumii}{enumiii}{}{.}%
\item {} 
\sphinxAtStartPar
Next time to start synchronization is 5 minutes after, if screen is on.

\item {} 
\sphinxAtStartPar
Next time to start synchronization is 60 minutes after, if screen is off.

\item {} 
\sphinxAtStartPar
Synchronization is canceled when phone falls asleep.

\end{enumerate}

\item {} 
\sphinxAtStartPar
Since main screen started the app checks next time of synchronization.
\begin{enumerate}
\sphinxsetlistlabels{\arabic}{enumii}{enumiii}{}{.}%
\item {} 
\sphinxAtStartPar
Nothing happen if next time is within 10 minutes.

\item {} 
\sphinxAtStartPar
Synchronization starts if next time is greater then 10 minutes.

\end{enumerate}

\item {} 
\sphinxAtStartPar
Synchronization is off when where is no network until network is on.

\end{enumerate}

\sphinxAtStartPar
You can run synchronization by hands if you want to.


\section{How Synchronization Works}
\label{\detokenize{teamwork:how-synchronization-works}}
\sphinxAtStartPar
The app Budget Blitz for Android stores every directory items and transactions changes. Node sends changes
that occur starting from the time last message sent or node created itself. The sequence
of exchange matters. Alice sends changes to Bob. Then Bob sends changes to Alice and so on.
The node of Alice will await response from the node of Bob. Thus, the node of Alice will send
no anymore messages until response from the Bob coming.

\sphinxAtStartPar
The app synchronizes directories using rules below:
\begin{enumerate}
\sphinxsetlistlabels{\arabic}{enumi}{enumii}{}{.}%
\item {} 
\sphinxAtStartPar
syncing by the unique id;

\item {} 
\sphinxAtStartPar
syncing by key phrases;

\item {} 
\sphinxAtStartPar
syncing by the name.

\end{enumerate}

\sphinxAtStartPar
When syncing fails the app goes to the next step. The app will create new item or use default value
if all steps fail. You can edit default values at the node card.

\sphinxAtStartPar
Transactions sync by the unique id only.


\section{Advanced Settings}
\label{\detokenize{teamwork:advanced-settings}}
\sphinxAtStartPar
Alice and Bob can restrict the amount of information transmitted. There are two types of constraints:
\begin{enumerate}
\sphinxsetlistlabels{\arabic}{enumi}{enumii}{}{.}%
\item {} 
\sphinxAtStartPar
permitted data scope;

\item {} 
\sphinxAtStartPar
forbidden data scope.

\end{enumerate}

\sphinxAtStartPar
Scopes specified in the \DUrole{bbmeta}{Data scopes} directory. It is possible to specify any combination of accounts,
categories, payers, payees, projects, and persons.

\sphinxAtStartPar
Forbidden scope has a higher priority, when the permitted and forbidden data scopes contain
same item simultaneously.

\sphinxAtStartPar
Transactions, recurring transactions, and directory items to transmit are based upon data scopes.

\sphinxAtStartPar
Alice and Bob can restrict items to receive. For example, Alice can refuse all new, modified
or removed items. Another case is to specify certain type of directory to refuse.


\section{Data Transfer Settings}
\label{\detokenize{teamwork:data-transfer-settings}}
\sphinxAtStartPar
To improve the security of data transmission Alice should specify a password that will
be used to encrypt messages between exchange nodes. Alice’s password must match the Bob’s one.

\sphinxAtStartPar
Alice also should indicate what type of communications is used for messaging with Bob.
Available types are Wi\sphinxhyphen{}Fi and mobile network.


\section{Default Values}
\label{\detokenize{teamwork:default-values}}
\sphinxAtStartPar
Alice and Bob can have a different app content. For example, Bob has a long time using Budget Blitz for Android,
and Alice has just installed the app. Bob can create transaction and specify, for example,
a project that Alice does not have. When a message from Bob arrives, Alice’s node will
create a transaction, but could not find a proper project. In that case the app will
use a default value Alice set to the node of Bob.


\section{Moving Database to a New Device}
\label{\detokenize{teamwork:moving-database-to-a-new-device}}
\sphinxAtStartPar
Suppose that Alice decides to move onto new phone. Then, Alice should follow steps:
\begin{enumerate}
\sphinxsetlistlabels{\arabic}{enumi}{enumii}{}{.}%
\item {} 
\sphinxAtStartPar
Set synchronization off for an old device.

\item {} 
\sphinxAtStartPar
Make backup.

\item {} 
\sphinxAtStartPar
Restore backup on a new device.

\item {} 
\sphinxAtStartPar
Set synchronization on for a new device.

\end{enumerate}

\sphinxstepscope


\chapter{Reports}
\label{\detokenize{reports:reports}}\label{\detokenize{reports:chapter-reports}}\label{\detokenize{reports::doc}}
\sphinxAtStartPar
The top menu is the place where reports are available. Each report has the options to filter,
grouping and saving of settings. Use the bottom sheet to manage a report.
When you open a report from the transactions list or another report filter inherits.

\noindent\sphinxincludegraphics[width=0.250\linewidth]{{reports-010-select-reports}.png}

\noindent\sphinxincludegraphics[width=0.250\linewidth]{{reports-020-menu-reports}.png}

\noindent\sphinxincludegraphics[width=0.250\linewidth]{{reports-025-report-bottom-sheet}.png}

\noindent\sphinxincludegraphics[width=0.250\linewidth]{{reports-026-report-bottom-sheet-open}.png}

\noindent\sphinxincludegraphics[width=0.250\linewidth]{{reports-027-report-select-group}.png}

\sphinxAtStartPar
You can always drill down from row of report to look at source transactions.

\sphinxAtStartPar
Also it is possible to make shortcuts for having fast access to a report with predefined settings.
Once you have created shortcuts, they are available from the Android launcher screens.


\section{Payment Schedule}
\label{\detokenize{reports:payment-schedule}}
\sphinxAtStartPar
The report is intended to display upcoming payments planning. The schedule contains planned and actual transactions
of the current time range. Hence, you can see not only planned but remunerated transactions as well.

\noindent\sphinxincludegraphics[width=0.250\linewidth]{{reports-030-payments-schedule}.png}

\noindent\sphinxincludegraphics[width=0.250\linewidth]{{reports-040-plan-vs-fact}.png}

\noindent\sphinxincludegraphics[width=0.250\linewidth]{{reports-050-turnovers}.png}


\section{Plan vs. Actual}
\label{\detokenize{reports:plan-vs-actual}}
\sphinxAtStartPar
The report is intended to show deviations between planned and actual transactions of the current time range.
Top row is an actuals and next row is a plan. For example, you can see that there is unplanned credit
transaction under \DUrole{bbitem}{Loans} category, and planned amount
under \DUrole{bbitem}{Clothes, footwear and accessories} category is remunerated.

\sphinxAtStartPar
You can get the report grouped by dimensions and periods as well.


\section{Turnovers}
\label{\detokenize{reports:turnovers}}
\sphinxAtStartPar
The report is intended to analyze aggregated turnovers of the current time range.
For example, you can see that there are a credit transaction under \DUrole{bbitem}{Loans} category,
debit transaction under \DUrole{bbitem}{Pocket expenses} one, and so on.

\sphinxAtStartPar
The report is able to show actual and planned transaction as well. The report displays actual transactions
by default.


\section{Totals and Turnovers}
\label{\detokenize{reports:totals-and-turnovers}}
\sphinxAtStartPar
The report is intended to analyze opening, closing balances and aggregated turnovers of the current time range.
The report is based on actual transactions only.

\noindent\sphinxincludegraphics[width=0.250\linewidth]{{reports-060-totals-turnovers}.png}

\noindent\sphinxincludegraphics[width=0.250\linewidth]{{reports-070-planned-totals-turnovers}.png}

\noindent\sphinxincludegraphics[width=0.250\linewidth]{{reports-080-debts}.png}


\section{Planned Totals and Turnovers}
\label{\detokenize{reports:planned-totals-and-turnovers}}
\sphinxAtStartPar
The report is intended to analyze opening, closing balances and aggregated turnovers of the current time range.
The report is based on planned transactions only.


\section{Debts}
\label{\detokenize{reports:debts}}
\sphinxAtStartPar
The report is based on transactions that contain categories having \DUrole{bbproperty}{Totaling (summary)},
\DUrole{bbproperty}{Revenue}, and \DUrole{bbproperty}{Expense} options are on. The report
shows opening, closing balances and aggregated turnovers. Zero amounts are hidden.

\sphinxAtStartPar
For example, you can see that \DUrole{bbitem}{Loans} category has no opening balance.
During the time range there was a loan transactions. And there was no repayment because
closing balance is equal to the credit amount.


\section{Plan Implementation}
\label{\detokenize{reports:plan-implementation}}
\sphinxAtStartPar
The report is based on planned and actual transactions that contain categories having
\DUrole{bbproperty}{Totaling (summary)} option is on, and one of \DUrole{bbproperty}{Revenue}
and \DUrole{bbproperty}{Expense} options only is on. The report evaluates a total
amount of planned transactions and deduct a total amount of actual transactions.
\DUrole{bbmeta}{Plan implementation} shows opening,
closing balances and aggregated turnovers. Zero amounts are hidden.

\sphinxAtStartPar
For example, you can see that \DUrole{bbitem}{Salary, wages} category has the opening balance.
Hence, it is not completed, i.e. actual amount is less than planned.
Also it is expected an actual credit transaction. But there is no
actual transaction yet.

\noindent\sphinxincludegraphics[width=0.250\linewidth]{{reports-090-plan-implementation}.png}

\noindent\sphinxincludegraphics[width=0.250\linewidth]{{reports-100-donut}.png}

\noindent\sphinxincludegraphics[width=0.250\linewidth]{{reports-110-bars}.png}


\section{Distribution of Turnovers}
\label{\detokenize{reports:distribution-of-turnovers}}
\sphinxAtStartPar
The chart is intended to analyze turnovers distribution by dimension and time ranges. You can view
expenses or revenues separately. Rotate chart counterclockwise when names are not fully shown.


\section{Changes in Turnovers}
\label{\detokenize{reports:changes-in-turnovers}}
\sphinxAtStartPar
The chart is intended to analyze tends of turnovers. Positive part of the chart contains
credits and negative part contains debits.


\section{Changes in Totals}
\label{\detokenize{reports:changes-in-totals}}
\sphinxAtStartPar
The chart is intended to analyze how totals change within the time range. At the same time
it is possible to look at actual and planned totals.

\noindent\sphinxincludegraphics[width=0.250\linewidth]{{reports-120-lines}.png}

\sphinxstepscope


\chapter{Reminders}
\label{\detokenize{reminders:reminders}}\label{\detokenize{reminders:chapter-reminders}}\label{\detokenize{reminders::doc}}
\sphinxAtStartPar
Budget Blitz for Android can remind you about some important events based upon reports or transactions list. Reminders
may be one\sphinxhyphen{}time or recurring. Using reminders you are able to:
\begin{itemize}
\item {} 
\sphinxAtStartPar
Customize notifications about no category transactions.

\item {} 
\sphinxAtStartPar
Customize notifications about specific transactions.

\item {} 
\sphinxAtStartPar
Customize warnings about plan and actuals difference.

\item {} 
\sphinxAtStartPar
Customize warnings about any events you can find with reports.

\item {} 
\sphinxAtStartPar
Run reports on a schedule.

\end{itemize}

\begin{sphinxadmonition}{note}{Note:}
\sphinxAtStartPar
You can access to the report from notification using version Pro
\end{sphinxadmonition}

\sphinxAtStartPar
First of all you need to specify criteria for events that will produce notifications.
To achieve that edit and save report setting having required filters and grouping, see
{\hyperref[\detokenize{shortcuts:chapter-shortcuts}]{\sphinxcrossref{\DUrole{std,std-ref}{Report Settings and Shortcuts}}}} (\autopageref*{\detokenize{shortcuts:chapter-shortcuts}}).

\sphinxAtStartPar
As far as setting is ready make reminder for it using \DUrole{bbbutton}{Reminders} button or from \DUrole{bbmeta}{Reminders}
directory.

\begin{sphinxadmonition}{note}{Note:}
\sphinxAtStartPar
Beginning from Android 4.4 the accuracy of the reminders is +/\sphinxhyphen{} 15 minutes.
\end{sphinxadmonition}

\sphinxAtStartPar
Let’s take a look at example of creating a reminder about transactions with an empty category.
On the home screen move to the transactions list.

\noindent\sphinxincludegraphics[width=0.250\linewidth]{{reminders-010-main-screen}.png}

\noindent\sphinxincludegraphics[width=0.250\linewidth]{{reminders-020-main-screen-swipe-left}.png}

\noindent\sphinxincludegraphics[width=0.250\linewidth]{{reminders-030-main-screen-transactions}.png}

\sphinxAtStartPar
Edit the filter so that transaction list has only ones with an empty category.

\noindent\sphinxincludegraphics[width=0.250\linewidth]{{reminders-040-transactions-bottom-sheet-opening}.png}

\noindent\sphinxincludegraphics[width=0.250\linewidth]{{reminders-050-transactions-bottom-sheet-open}.png}

\noindent\sphinxincludegraphics[width=0.250\linewidth]{{reminders-060-report-filter}.png}

\noindent\sphinxincludegraphics[width=0.250\linewidth]{{reminders-070-report-filter-category}.png}

\noindent\sphinxincludegraphics[width=0.250\linewidth]{{reminders-080-report-filter-apply}.png}

\noindent\sphinxincludegraphics[width=0.250\linewidth]{{reminders-090-report-filter-applied}.png}

\sphinxAtStartPar
After the filter applied, transactions list has changed, it now contains only two items. Let’s
save the filter into report setting. Open bottom side for that and pressing on \DUrole{bbspinner}{Report settings}
create a new report setting. You can modify period of setting but now we leave it unchanged.

\noindent\sphinxincludegraphics[width=0.250\linewidth]{{reminders-100-report-select-new-setting}.png}

\noindent\sphinxincludegraphics[width=0.250\linewidth]{{reminders-100-report-setting-save}.png}

\noindent\sphinxincludegraphics[width=0.250\linewidth]{{reminders-110-report-view-settings-alarms}.png}

\sphinxAtStartPar
Now let’s create reminder base upon new report setting. Move to reminders list and make a new one.
Specify date of beginning, time of running, recurrence type and a name. You will see the name
in a notification that produced by reminder.

\noindent\sphinxincludegraphics[width=0.250\linewidth]{{reminders-120-alarms-new}.png}

\noindent\sphinxincludegraphics[width=0.250\linewidth]{{reminders-130-alarms-edit}.png}

\noindent\sphinxincludegraphics[width=0.250\linewidth]{{reminders-140-alarms}.png}

\sphinxAtStartPar
Reminder is ready, now we will test it. Select reminder and press \DUrole{bbbutton}{Run}.
In the status bar you see notification about transactions with an empty category.

\noindent\sphinxincludegraphics[width=0.250\linewidth]{{reminders-150-alarms-select}.png}

\noindent\sphinxincludegraphics[width=0.250\linewidth]{{reminders-160-alarms-run}.png}

\noindent\sphinxincludegraphics[width=0.250\linewidth]{{reminders-170-alarm_notification}.png}

\sphinxAtStartPar
You can press on notification to preview list of transactions.

\begin{sphinxadmonition}{note}{Note:}
\sphinxAtStartPar
Only Pro version allows moving to data of notifications.
\end{sphinxadmonition}

\sphinxAtStartPar
Now you will see notification every day at the specified time when
you have one or more transactions with an empty category.

\sphinxstepscope


\chapter{Bulk Actions}
\label{\detokenize{bulk-actions:bulk-actions}}\label{\detokenize{bulk-actions:chapter-bulk-actions}}\label{\detokenize{bulk-actions::doc}}
\sphinxAtStartPar
Budget Blitz for Android admits to make actions under the set of objects. For example you can change a category
in several transactions at once. Bulk actions are supported by any directories too.


\section{Objects Selection}
\label{\detokenize{bulk-actions:objects-selection}}
\sphinxAtStartPar
In the example below you can see multiply selection at the transactions list. Same actions can be done in
any directory.

\noindent\sphinxincludegraphics[width=0.250\linewidth]{{bulkactions-010-transactions}.png}

\noindent\sphinxincludegraphics[width=0.250\linewidth]{{bulkactions-020-transaction-check}.png}

\noindent\sphinxincludegraphics[width=0.250\linewidth]{{bulkactions-030-transactions-checked}.png}

\sphinxAtStartPar
Open transactions list first of all. Then select required transactions.
To select all transactions just select one and press \DUrole{bbbutton}{Select all} from the top menu.


\section{Editing}
\label{\detokenize{bulk-actions:editing}}
\sphinxAtStartPar
Press \DUrole{bbbutton}{Edit} button to edit selected transactions. You will see the dialog that
contains amount of objects and properties available to change. Modifications are applied
only for changed properties.

\noindent\sphinxincludegraphics[width=0.250\linewidth]{{bulkactions-040-transactions-edit}.png}

\noindent\sphinxincludegraphics[width=0.250\linewidth]{{bulkactions-050-transactions-edit-dialog}.png}


\section{Deleting}
\label{\detokenize{bulk-actions:deleting}}
\sphinxAtStartPar
Press \DUrole{bbbutton}{Delete} to delete selected transactions. Since confirmed the app will delete
selected items.

\noindent\sphinxincludegraphics[width=0.250\linewidth]{{bulkactions-060-transactions-delete}.png}

\noindent\sphinxincludegraphics[width=0.250\linewidth]{{bulkactions-070-transactions-delete-dialog}.png}


\section{Filters}
\label{\detokenize{bulk-actions:filters}}
\sphinxAtStartPar
You can make a filter based on selected items. It is convenient when you want, for example,
to see all transactions with the same dimensions as selected ones.

\sphinxAtStartPar
Press \DUrole{bbbutton}{Filter} to make a filter.

\noindent\sphinxincludegraphics[width=0.250\linewidth]{{bulkactions-080-transactions-filter}.png}

\noindent\sphinxincludegraphics[width=0.250\linewidth]{{bulkactions-090-transactions-filter-dialog}.png}


\section{Repetitive Sending Under Teamwork}
\label{\detokenize{bulk-actions:repetitive-sending-under-teamwork}}
\sphinxAtStartPar
Sometimes there would be a necessity to send transaction or directory item again when teamwork used.
Press \sphinxmenuselection{Share on data exchange} menu item to do that.

\noindent\sphinxincludegraphics[width=0.250\linewidth]{{bulkactions-100-transactions-more}.png}

\noindent\sphinxincludegraphics[width=0.250\linewidth]{{bulkactions-110-transactions-exchange-send}.png}

\noindent\sphinxincludegraphics[width=0.250\linewidth]{{bulkactions-120-transactions-exchange-send-done}.png}


\section{CSV and OFX Export}
\label{\detokenize{bulk-actions:csv-and-ofx-export}}
\sphinxAtStartPar
You can export selected transactions to CSV and OFX files. Press \sphinxmenuselection{Export CSV}
and \sphinxmenuselection{Export OFX} to do that. In contrast to transactions directory items
can be exported to a CSV file only.

\begin{sphinxadmonition}{note}{Note:}
\sphinxAtStartPar
Export transaction to an OFX file is available in the Pro version only.
\end{sphinxadmonition}

\noindent\sphinxincludegraphics[width=0.250\linewidth]{{bulkactions-130-transactions-export-csv}.png}

\noindent\sphinxincludegraphics[width=0.250\linewidth]{{bulkactions-150-transactions-export-ofx}.png}

\noindent\sphinxincludegraphics[width=0.250\linewidth]{{bulkactions-160-transactions-export-ofx-done}.png}


\section{Connecting of transactions}
\label{\detokenize{bulk-actions:connecting-of-transactions}}
\sphinxAtStartPar
Sometimes one wants to additionally connect transactions to have a real transfer. For a
example, you may want to connect two separate transactions when transfer was created by hands or
as a result of an import. To get transactions connected just check at least one transaction and
run the operation and the app will find and connect complement transactions automatic.

\noindent\sphinxincludegraphics[width=0.250\linewidth]{{bulkactions-190-transactions-connect}.png}

\begin{sphinxadmonition}{note}{Note:}
\sphinxAtStartPar
Starting from the version 6 both transactions are automatic connecting when transfer is
made by hands, thus there is no need to additionally connect these transactions. Connecting
transactions are marked by a special icon.
\end{sphinxadmonition}


\section{Sending Source Data to Developer}
\label{\detokenize{bulk-actions:sending-source-data-to-developer}}
\sphinxAtStartPar
Sometime you need a help to understand what is going on in the app. Usually in
order to get a help it is required to show source data to the developer.

\sphinxAtStartPar
Press \sphinxmenuselection{Send to developer} menu item to send select objects. You will
see a letter before sending, so you will be able to edit some data.

\noindent\sphinxincludegraphics[width=0.250\linewidth]{{bulkactions-170-transactions-developer-send}.png}

\noindent\sphinxincludegraphics[width=0.250\linewidth]{{bulkactions-180-transactions-developer-send}.png}

\sphinxstepscope


\chapter{Report Settings and Shortcuts}
\label{\detokenize{shortcuts:report-settings-and-shortcuts}}\label{\detokenize{shortcuts:chapter-shortcuts}}\label{\detokenize{shortcuts::doc}}

\section{Report Settings}
\label{\detokenize{shortcuts:report-settings}}
\sphinxAtStartPar
Budget Blitz for Android can store custom grouping and filters for reports and transactions list. Let us
look into the matter using an example of \DUrole{bbmeta}{Turnovers} report. You can save
settings for other reports and transactions list by the same way.

\sphinxAtStartPar
Since started the report has default grouping, filter and time range.

\noindent\sphinxincludegraphics[width=0.250\linewidth]{{shortcuts-010-select-reports}.png}

\noindent\sphinxincludegraphics[width=0.250\linewidth]{{shortcuts-020-report-open}.png}

\noindent\sphinxincludegraphics[width=0.250\linewidth]{{shortcuts-030-report-bottom-sheet-opening}.png}

\sphinxAtStartPar
We will try to make fast access to \DUrole{bbmeta}{Turnovers} report based on a filter having one account only.

\sphinxAtStartPar
Edit filter settings. To do that pull bottom sheet and press \DUrole{bbspinner}{Filter}. Select only one account
and save changes.

\noindent\sphinxincludegraphics[width=0.250\linewidth]{{shortcuts-040-report-bottom-sheet-open}.png}

\noindent\sphinxincludegraphics[width=0.250\linewidth]{{shortcuts-050-report-filter}.png}

\noindent\sphinxincludegraphics[width=0.250\linewidth]{{shortcuts-060-report-filter-account}.png}

\sphinxAtStartPar
You can see at the pictures that report contains data of the only one account. To create
a persistent setting press \DUrole{bbspinner}{Report settings} at the bottom sheet
and choose create new setting from the drop down list.

\noindent\sphinxincludegraphics[width=0.250\linewidth]{{shortcuts-070-report-filter-apply}.png}

\noindent\sphinxincludegraphics[width=0.250\linewidth]{{shortcuts-080-report-filter-applied}.png}

\noindent\sphinxincludegraphics[width=0.250\linewidth]{{shortcuts-090-report-select-new-setting}.png}

\sphinxAtStartPar
Put the name \DUrole{bbvalue}{One account turnovers} for the new setting and save. Now
\DUrole{bbitem}{One account turnovers} is ready to use. A report will have
grouping and filter values of the setting when you choose it from the
drop down list.

\noindent\sphinxincludegraphics[width=0.250\linewidth]{{shortcuts-100-report-setting-save}.png}

\noindent\sphinxincludegraphics[width=0.250\linewidth]{{shortcuts-110-report-view-settings}.png}


\section{Creating Shortcuts}
\label{\detokenize{shortcuts:creating-shortcuts}}
\sphinxAtStartPar
Using Budget Blitz for Android you can open reports and transactions list from the Android launcher screen. At the previous
section you have got the new persistent setting. Let us suppose that you want to create a shortcut for it.

\begin{sphinxadmonition}{note}{Note:}
\sphinxAtStartPar
Shortcuts available in the Pro version.
\end{sphinxadmonition}

\sphinxAtStartPar
Let us go back to the setting card. Please keep in mind that setting has a frequency. The time range of the
new report depends on that frequency. Current month is the default value and you can put another one.

\sphinxAtStartPar
Press \DUrole{bbbutton}{Create shortcut} to create the shortcut.

\noindent\sphinxincludegraphics[width=0.250\linewidth]{{shortcuts-120-report-setting-shortcut-create}.png}

\noindent\sphinxincludegraphics[width=0.250\linewidth]{{shortcuts-130-report-shortcut-select}.png}

\sphinxAtStartPar
New shortcut will appear at the free space of the one of Android launcher screens.

\begin{sphinxadmonition}{note}{Note:}
\sphinxAtStartPar
The shortcut is connected with list view setting. The shortcut will be broken when setting has removed.
\end{sphinxadmonition}

\noindent\sphinxincludegraphics[width=0.250\linewidth]{{shortcuts-140-report-open}.png}

\noindent\sphinxincludegraphics[width=0.250\linewidth]{{shortcuts-150-report-bottom-sheet-opening}.png}

\noindent\sphinxincludegraphics[width=0.250\linewidth]{{shortcuts-160-report-bottom-sheet-open}.png}

\sphinxAtStartPar
Let us check the shortcut. Press it and make sure that report started. As you can see at pictures the app
applied filter of the shortcut.

\sphinxAtStartPar
A shortcut is just a link to persistent setting. Modify setting properties if you want to modify shortcut.

\sphinxstepscope


\chapter{Widgets and Templates}
\label{\detokenize{widgets:widgets-and-templates}}\label{\detokenize{widgets:chapter-widgets}}\label{\detokenize{widgets::doc}}

\section{Widgets}
\label{\detokenize{widgets:widgets}}
\sphinxAtStartPar
Budget Blitz for Android has the convenient widget to show actual balance, turnovers and to create new transaction.

\noindent{\hspace*{\fill}\sphinxincludegraphics[width=0.250\linewidth]{{widget-480}.png}\hspace*{\fill}}

\sphinxAtStartPar
Available size of widget are 1x1, 1x2, and 1x4. The widget theme and the app theme are equals.

\sphinxAtStartPar
You can use widget not only like a financial highlights. Other ways of usage are short report
and template of new transaction.

\noindent\sphinxincludegraphics[width=0.250\linewidth]{{widgets-005-widget-available}.png}

\noindent\sphinxincludegraphics[width=0.250\linewidth]{{widgets-010-widget-settings-open}.png}

\noindent\sphinxincludegraphics[width=0.250\linewidth]{{widgets-020-widget-settings}.png}

\sphinxAtStartPar
A new widget contains a balance and turnovers for the current day. Press \DUrole{bbbutton}{Settings}
to modify widget.

\sphinxAtStartPar
The \DUrole{bbsection}{View} section has main widget options.

\sphinxAtStartPar
Use the \DUrole{bbproperty}{Name} when you have more than one widget. You can keep it empty if you want to.

\sphinxAtStartPar
Use the \DUrole{bbproperty}{Types of portfolios}, the \DUrole{bbproperty}{Portfolios} and the \DUrole{bbproperty}{Accounts} options
to restrict information of widget. You can put one or more values. Different widgets
can have different options. For example you can have two widgets, one for
certain account, and another for another one.

\sphinxAtStartPar
Use the \DUrole{bbproperty}{Show balance} to set balance visibility on or off. Also you can specify
whether credit limit is ignored or not when balance is calculated.
Balance is free of credit limit by default and for credit cards
you will have a negative balance.

\noindent\sphinxincludegraphics[width=0.250\linewidth]{{widgets-030-widget-settings-2}.png}

\noindent\sphinxincludegraphics[width=0.250\linewidth]{{widgets-040-widget-settings-apply}.png}


\section{Using Widgets as Transaction Templates}
\label{\detokenize{widgets:using-widgets-as-transaction-templates}}
\sphinxAtStartPar
The \DUrole{bbbutton}{New transaction} button is available in the widget since account for new transactions specified in
settings. Also you can set an amount for new transaction. That amount will be copied to a new transaction.

\sphinxAtStartPar
Values of a filter will be copied to a new transaction as well.

\sphinxAtStartPar
Thus, you can use widget like a new transaction template.

\begin{sphinxadmonition}{note}{Note:}
\sphinxAtStartPar
Templates are available in the Pro version. Free version ignores an amount and values of filter.
\end{sphinxadmonition}


\section{Using Widgets as Reports}
\label{\detokenize{widgets:using-widgets-as-reports}}
\sphinxAtStartPar
Since widget has flexible settings you can use it as a report with persistent settings. The
\DUrole{bbsection}{Filter} section is the key.

\begin{sphinxadmonition}{note}{Note:}
\sphinxAtStartPar
Widgets as reports are available in the Pro version.
\end{sphinxadmonition}


\section{Widgets as Reports Example}
\label{\detokenize{widgets:widgets-as-reports-example}}
\sphinxAtStartPar
Let us make a widget setup to show public transport expenses during current month.
Open widget settings and put the \DUrole{bbvalue}{Public transport} name.

\noindent\sphinxincludegraphics[width=0.250\linewidth]{{widgets-050-widget-example-set-name}.png}

\noindent\sphinxincludegraphics[width=0.250\linewidth]{{widgets-060-widget-example-select-period}.png}

\noindent\sphinxincludegraphics[width=0.250\linewidth]{{widgets-070-widget-example-select-period-apply}.png}

\sphinxAtStartPar
Set balance off, because we do not need to see totals. Select current month as the time range.

\noindent\sphinxincludegraphics[width=0.250\linewidth]{{widgets-080-widget-example-select-budget-item}.png}

\noindent\sphinxincludegraphics[width=0.250\linewidth]{{widgets-090-widget-example-select-budget-item-apply}.png}

\noindent\sphinxincludegraphics[width=0.250\linewidth]{{widgets-100-widget-example-settings-apply}.png}

\sphinxAtStartPar
Set the \DUrole{bbitem}{Public transport} category and save settings.

\noindent\sphinxincludegraphics[width=0.250\linewidth]{{widgets-110-widget-example}.png}

\sphinxAtStartPar
Now you can see turnovers under \DUrole{bbitem}{Public transport} category for the current month,
expenses amount, and accounts that are the source of payments.


\section{Widgets as Templates Example}
\label{\detokenize{widgets:widgets-as-templates-example}}
\sphinxAtStartPar
Now, modify settings the way you can fast create expenses. Open the setting to do that.

\noindent\sphinxincludegraphics[width=0.250\linewidth]{{widgets-120-widget-example-select-account}.png}

\noindent\sphinxincludegraphics[width=0.250\linewidth]{{widgets-140-widget-example-select-amount}.png}

\noindent\sphinxincludegraphics[width=0.250\linewidth]{{widgets-150-widget-example-select-amount-value}.png}

\sphinxAtStartPar
Set the account you will pay often for public transport. Also put the most frequent amount.

\noindent\sphinxincludegraphics[width=0.250\linewidth]{{widgets-160-widget-example-select-amount-value-2}.png}

\noindent\sphinxincludegraphics[width=0.250\linewidth]{{widgets-170-widget-example-settings-apply}.png}

\noindent\sphinxincludegraphics[width=0.250\linewidth]{{widgets-180-widget-example-transaction-new}.png}

\sphinxAtStartPar
Save settings. Now the button to create new transaction appeared.

\noindent\sphinxincludegraphics[width=0.250\linewidth]{{widgets-190-widget-example-transaction}.png}

\sphinxAtStartPar
Create new transaction and you will see one contains the account, the amount and the category already.
All you have left to do is save the new transaction.

\sphinxAtStartPar
Using same way you can put payer, payee, project, and person for a new transaction. Each new
template should have a new widget.

\sphinxstepscope


\chapter{Remote access}
\label{\detokenize{remote-access:remote-access}}\label{\detokenize{remote-access:chapter-remote-access}}\label{\detokenize{remote-access::doc}}
\sphinxAtStartPar
Since installed Budget Blitz for Android has the PC client. It supports Windows, Linux, Mac, etc. All you need is
a modern browser, Internet Explorer 8+, Google Chrome, Apple Safari, Mozilla Firefox, or Opera.

\noindent\sphinxincludegraphics[width=0.250\linewidth]{{remoteaccess-010-remote-access-enable}.png}

\noindent\sphinxincludegraphics[width=0.250\linewidth]{{remoteaccess-020-view-status-bar}.png}

\noindent\sphinxincludegraphics[width=0.250\linewidth]{{remoteaccess-030-remote-access-disable}.png}

\sphinxAtStartPar
Activate client from the main screen. After the PC access get started the app will show
the brief guide how to run client on your PC. At the same time you can see the display sign
at the Android top bar.

\sphinxAtStartPar
PC client contains ready to print reports and charts.

\sphinxstepscope


\chapter{Integration with third party applications}
\label{\detokenize{api:integration-with-third-party-applications}}\label{\detokenize{api:chapter-api}}\label{\detokenize{api::doc}}
\sphinxAtStartPar
You can integrate Budget Blitz for Android with other applications. For example, you can connect Budget Blitz for Android
with a voice assistant and create transactions by voice. Another hint is to create
transactions using \sphinxhref{https://play.google.com/store/apps/details?id=net.dinglisch.android.taskerm}{Tasker}.


\section{Making Transactions From Text}
\label{\detokenize{api:making-transactions-from-text}}
\sphinxAtStartPar
To create a new transaction you just need to send broadcast intent. Since intent received the app analyze it
and create new transaction using notifications detection algorithm.

\sphinxAtStartPar
Intent parameters are

\sphinxAtStartPar
Class = \DUrole{bbvar}{biz.interblitz.intent.CONVERT\_TEXT\_TO\_NEW\_TRANSACTION}

\sphinxAtStartPar
Extras:
\begin{enumerate}
\sphinxsetlistlabels{\arabic}{enumi}{enumii}{}{.}%
\item {} 
\sphinxAtStartPar
\DUrole{bbvar}{timestampMillis}: Type of long, date and time of a new transaction in milliseconds. Current date and time used when empty.

\item {} 
\sphinxAtStartPar
\DUrole{bbvar}{address}: Type of String, sender of the message, can be empty.

\item {} 
\sphinxAtStartPar
\DUrole{bbvar}{message}: Type of String, message like a notification to create a new transaction, required.

\end{enumerate}


\section{REST API}
\label{\detokenize{api:rest-api}}\label{\detokenize{api:sub-chapter-rest-api}}
\sphinxAtStartPar
Budget Blitz for Android supports \sphinxhref{https://github.com/interblitz/BudgetBlitz-Api}{REST API}. API allows to create new directories and transactions, and edit or delete existed ones. Using
this API you can create your own addons or applications.

\sphinxAtStartPar
You have to enable remote access to dial with \sphinxhref{https://github.com/interblitz/BudgetBlitz-Api}{REST API} and read documentation, see chart {\hyperref[\detokenize{remote-access:chapter-remote-access}]{\sphinxcrossref{\DUrole{std,std-ref}{Remote access}}}} (\autopageref*{\detokenize{remote-access:chapter-remote-access}}).
Documentation is available by \sphinxhref{https://interblitz.github.io/BudgetBlitz-Api/swagger/}{Swagger}. On the \sphinxhref{https://interblitz.github.io/BudgetBlitz-Api/swagger/}{Swagger}. page type the address  \sphinxurl{http://{[}server{]}:{[}port{]}/api/v1/docs.json}.
Server and port will be available after PC connection enabled.

\sphinxAtStartPar
You can try app examples at the \sphinxhref{https://github.com/interblitz/BudgetBlitz-Api}{github.com}. After an exmple has loaded type Budget Blitz for Android address as \sphinxurl{http://{[}server{]}:{[}port}{]}.


\section{Intents API}
\label{\detokenize{api:intents-api}}
\sphinxAtStartPar
In addition to simple API for making transactions from text Budget Blitz for Android supports extended Intents API. It consists of two parts,
events and data requests. API based on the REST API. By default Intents API is OFF. You have to enable it
selecting the part you need.


\subsection{Intents API: Part 1, Events}
\label{\detokenize{api:intents-api-part-1-events}}
\sphinxAtStartPar
When directories and transactions are saving events occurs. On the event Budget Blitz for Android sends Intent. You have to select target
packages in the settings. Intent contains:

\sphinxAtStartPar
Action = \DUrole{bbvar}{\{biz.interblitz.budget\{free/pro\}.api.event.ITEM\_ONCHANGE}

\sphinxAtStartPar
Extras:
\begin{enumerate}
\sphinxsetlistlabels{\arabic}{enumi}{enumii}{}{.}%
\item {} 
\sphinxAtStartPar
\DUrole{bbvar}{collection} \sphinxhyphen{} collection name that fires event

\item {} 
\sphinxAtStartPar
\DUrole{bbvar}{id} \sphinxhyphen{} object id that fires event

\end{enumerate}

\sphinxAtStartPar
When transaction is coming from notification import Extras contains
\begin{enumerate}
\sphinxsetlistlabels{\arabic}{enumi}{enumii}{}{.}%
\item {} 
\sphinxAtStartPar
\DUrole{bbvar}{notification} \sphinxhyphen{} notification text

\item {} 
\sphinxAtStartPar
\DUrole{bbvar}{address} \sphinxhyphen{} notification address (phone number or package name)

\item {} 
\sphinxAtStartPar
\DUrole{bbvar}{amount} \sphinxhyphen{} transaction amount

\item {} 
\sphinxAtStartPar
\DUrole{bbvar}{currency} \sphinxhyphen{} transaction currency

\end{enumerate}

\sphinxAtStartPar
To get more data you should send request Intent.


\subsection{Intents API: Part 2, Requests}
\label{\detokenize{api:intents-api-part-2-requests}}
\sphinxAtStartPar
Request intents intended to get, modify or delete some data. Intent structure is

\sphinxAtStartPar
Class = \DUrole{bbvar}{biz.interblitz.service.ApiReceiver}

\sphinxAtStartPar
Action = \DUrole{bbvar}{\{biz.interblitz.budget\{free/pro\}.api.request}

\sphinxAtStartPar
Extras:
\begin{enumerate}
\sphinxsetlistlabels{\arabic}{enumi}{enumii}{}{.}%
\item {} 
\sphinxAtStartPar
\DUrole{bbvar}{method} \sphinxhyphen{} single value from: GET, POST, DELETE

\item {} 
\sphinxAtStartPar
\DUrole{bbvar}{path} \sphinxhyphen{} path to the collection

\item {} 
\sphinxAtStartPar
\DUrole{bbvar}{body} \sphinxhyphen{} JSON data

\item {} 
\sphinxAtStartPar
\DUrole{bbvar}{package} \sphinxhyphen{} full package name to receive response, response will not return if empty

\item {} 
\sphinxAtStartPar
\DUrole{bbvar}{class} \sphinxhyphen{} package class to receive response, may be empty

\end{enumerate}

\sphinxAtStartPar
Also Extras can contain any other data. All that data will returned back in response.

\sphinxAtStartPar
Budget Blitz for Android sends Intent response witt structure

\sphinxAtStartPar
Action = \DUrole{bbvar}{\{biz.interblitz.budget\{free/pro\}.api.response}

\sphinxAtStartPar
Extras:
\begin{enumerate}
\sphinxsetlistlabels{\arabic}{enumi}{enumii}{}{.}%
\item {} 
\sphinxAtStartPar
\DUrole{bbvar}{collection} \sphinxhyphen{} collection name

\item {} 
\sphinxAtStartPar
\DUrole{bbvar}{response} \sphinxhyphen{} JSON response

\end{enumerate}

\sphinxAtStartPar
Parameters \DUrole{bbvar}{method}, \DUrole{bbvar}{path}, \DUrole{bbvar}{body}, \DUrole{bbvar}{collection}, \DUrole{bbvar}{response} matches REST API. Documentation is available from the \sphinxhref{https://interblitz.github.io/BudgetBlitz-Api/swagger/}{Swagger}.
See more {\hyperref[\detokenize{api:sub-chapter-rest-api}]{\sphinxcrossref{\DUrole{std,std-ref}{REST API}}}} (\autopageref*{\detokenize{api:sub-chapter-rest-api}}).

\sphinxstepscope


\chapter{Difference Between Versions}
\label{\detokenize{versions:difference-between-versions}}\label{\detokenize{versions:chapter-versions}}\label{\detokenize{versions::doc}}

\begin{savenotes}\sphinxatlongtablestart\begin{longtable}[c]{|\X{25}{40}|\X{10}{40}|\X{5}{40}|}
\sphinxthelongtablecaptionisattop
\caption{Difference between versions\strut}\label{\detokenize{versions:id1}}\\*[\sphinxlongtablecapskipadjust]
\hline
\sphinxstyletheadfamily &\sphinxstyletheadfamily 
\sphinxAtStartPar
Free version
&\sphinxstyletheadfamily 
\sphinxAtStartPar
Pro version
\\
\hline
\endfirsthead

\multicolumn{3}{c}%
{\makebox[0pt]{\sphinxtablecontinued{\tablename\ \thetable{} \textendash{} continued from previous page}}}\\
\hline
\sphinxstyletheadfamily &\sphinxstyletheadfamily 
\sphinxAtStartPar
Free version
&\sphinxstyletheadfamily 
\sphinxAtStartPar
Pro version
\\
\hline
\endhead

\hline
\multicolumn{3}{r}{\makebox[0pt][r]{\sphinxtablecontinued{continues on next page}}}\\
\endfoot

\endlastfoot

\sphinxAtStartPar
Accounting
&\begin{itemize}
\item {} 
\end{itemize}
&\begin{itemize}
\item {} 
\end{itemize}
\\
\hline
\sphinxAtStartPar
Planning
&\begin{itemize}
\item {} 
\end{itemize}
&\begin{itemize}
\item {} 
\end{itemize}
\\
\hline
\sphinxAtStartPar
Reports
&\begin{itemize}
\item {} 
\end{itemize}
&\begin{itemize}
\item {} 
\end{itemize}
\\
\hline
\sphinxAtStartPar
SMS, OFX, CSV import
&\begin{itemize}
\item {} 
\end{itemize}
&\begin{itemize}
\item {} 
\end{itemize}
\\
\hline
\sphinxAtStartPar
Push notifications import
&
\sphinxAtStartPar
15 per month
&\begin{itemize}
\item {} 
\end{itemize}
\\
\hline
\sphinxAtStartPar
Additional SMS that confirms transaction and contains pins, passwords, etc import
&&\begin{itemize}
\item {} 
\end{itemize}
\\
\hline
\sphinxAtStartPar
Additional SMS that contains transaction details when transaction has more than one SMS import
&&\begin{itemize}
\item {} 
\end{itemize}
\\
\hline
\sphinxAtStartPar
OFX export
&&\begin{itemize}
\item {} 
\end{itemize}
\\
\hline
\sphinxAtStartPar
Teamwork
&
\sphinxAtStartPar
Sending data only
&\begin{itemize}
\item {} 
\end{itemize}
\\
\hline
\sphinxAtStartPar
Remote access
&
\sphinxAtStartPar
50 transactions
&\begin{itemize}
\item {} 
\end{itemize}
\\
\hline
\sphinxAtStartPar
Notifications about upcoming payments
&&\begin{itemize}
\item {} 
\end{itemize}
\\
\hline
\sphinxAtStartPar
Reports driven notifications
&&\begin{itemize}
\item {} 
\end{itemize}
\\
\hline
\sphinxAtStartPar
Shortcuts
&&\begin{itemize}
\item {} 
\end{itemize}
\\
\hline
\sphinxAtStartPar
Widgets as reports
&&\begin{itemize}
\item {} 
\end{itemize}
\\
\hline
\sphinxAtStartPar
Automatic backups
&&\begin{itemize}
\item {} 
\end{itemize}
\\
\hline
\sphinxAtStartPar
Backups encryption
&&\begin{itemize}
\item {} 
\end{itemize}
\\
\hline
\sphinxAtStartPar
Support
&\begin{itemize}
\item {} 
\end{itemize}
&\begin{itemize}
\item {} 
\end{itemize}
\\
\hline
\end{longtable}\sphinxatlongtableend\end{savenotes}

\sphinxAtStartPar
Google Play:

\sphinxAtStartPar
\sphinxhref{https://play.google.com/store/apps/details?id=biz.interblitz.budgetfree}{Free version}

\sphinxAtStartPar
\sphinxhref{https://play.google.com/store/apps/details?id=biz.interblitz.budgetpro}{Pro version}

\sphinxstepscope


\chapter{Migration to Pro Version}
\label{\detokenize{migration-to-pro:migration-to-pro-version}}\label{\detokenize{migration-to-pro:chapter-migration-to-pro}}\label{\detokenize{migration-to-pro::doc}}
\sphinxAtStartPar
There are two stage of migration, prepare data in old version and loading it into a new one.
It is very easy to do.
\begin{enumerate}
\sphinxsetlistlabels{\arabic}{enumi}{enumii}{}{.}%
\item {} 
\sphinxAtStartPar
Start Free version;

\item {} 
\sphinxAtStartPar
At the main screen press \sphinxmenuselection{Actions \(\rightarrow\) Export \(\rightarrow\) Pro version upgrade};

\item {} 
\sphinxAtStartPar
Start Pro version;

\item {} 
\sphinxAtStartPar
At the main screen press \sphinxmenuselection{Actions \(\rightarrow\) Import \(\rightarrow\) Free version data}.

\end{enumerate}

\sphinxstepscope


\chapter{Service}
\label{\detokenize{service:service}}\label{\detokenize{service:chapter-service}}\label{\detokenize{service::doc}}
\sphinxAtStartPar
Generally the app Budget Blitz for Android do not need any service. But when you notice
the app becomes slower service actions can help.

\noindent\sphinxincludegraphics[width=0.250\linewidth]{{service-010-select-actions}.png}

\noindent\sphinxincludegraphics[width=0.250\linewidth]{{service-020-select-service}.png}

\noindent\sphinxincludegraphics[width=0.250\linewidth]{{service-030-select-iitems}.png}

\sphinxAtStartPar
Shrinking database frees unused space, rebuilds the database file, repacking it into a minimal
amount of disk space. This contributes to speed up the app. Shrinking database runs
\sphinxhref{https://sqlite.org/lang\_vacuum.html}{VACUUM} command.

\sphinxAtStartPar
Shrinking does not affect to files that app contains except database.

\sphinxAtStartPar
Reindexing database is useful when you notice drastic drop in of the app performance.
Reindexing runs \sphinxhref{https://sqlite.org/lang\_reindex.html}{REINDEX} command.

\begin{sphinxadmonition}{warning}{Warning:}
\sphinxAtStartPar
Do not forget making backups, especially before service. Ensure you remember encryption password if you have. Otherwise restoring data could be impossible.
\end{sphinxadmonition}

\sphinxstepscope


\chapter{Ready to use Financial Institutes}
\label{\detokenize{banks:ready-to-use-financial-institutes}}\label{\detokenize{banks:chapter-supported-banks}}\label{\detokenize{banks::doc}}

\section{Belarus}
\label{\detokenize{banks:belarus}}
\sphinxAtStartPar
BPSSberbank

\sphinxAtStartPar
BelVneshEkonomBank

\sphinxAtStartPar
Belagroprombank

\sphinxAtStartPar
Belarusbank

\sphinxAtStartPar
Belgazprombank

\sphinxAtStartPar
Belinvestbank

\sphinxAtStartPar
Belrosbank

\sphinxAtStartPar
MTBank

\sphinxAtStartPar
Priorbank

\sphinxAtStartPar
Houm Kredit Belarus


\section{Brazil}
\label{\detokenize{banks:brazil}}
\sphinxAtStartPar
Banco do Brasil

\sphinxAtStartPar
Ita Unibanco


\section{Canada}
\label{\detokenize{banks:canada}}
\sphinxAtStartPar
ICICI Bank


\section{Hungary}
\label{\detokenize{banks:hungary}}
\sphinxAtStartPar
CIB BANK

\sphinxAtStartPar
OTP Bank  Simple


\section{India}
\label{\detokenize{banks:india}}
\sphinxAtStartPar
Central Bank of India

\sphinxAtStartPar
Deutsche Bank

\sphinxAtStartPar
State Bank of India


\section{Indonesia}
\label{\detokenize{banks:indonesia}}
\sphinxAtStartPar
Commonwealth Bank


\section{Maldives}
\label{\detokenize{banks:maldives}}
\sphinxAtStartPar
BML


\section{Poland}
\label{\detokenize{banks:poland}}
\sphinxAtStartPar
Bank Millennium SA


\section{Russia}
\label{\detokenize{banks:russia}}
\sphinxAtStartPar
AnyBalance

\sphinxAtStartPar
BSGV

\sphinxAtStartPar
KARI CLUB

\sphinxAtStartPar
Modulbank

\sphinxAtStartPar
QIWI

\sphinxAtStartPar
SDMBank

\sphinxAtStartPar
AKIBANK

\sphinxAtStartPar
AMT Bank

\sphinxAtStartPar
Absolyut Bank

\sphinxAtStartPar
Avangard

\sphinxAtStartPar
AyManiBank

\sphinxAtStartPar
AkBars

\sphinxAtStartPar
AlfaBank

\sphinxAtStartPar
BKS BANK

\sphinxAtStartPar
Baltiyskiy Bank

\sphinxAtStartPar
Bank Evropeyskiy

\sphinxAtStartPar
Bank Moskvi

\sphinxAtStartPar
Bank Petrokommerts

\sphinxAtStartPar
Bank Primore

\sphinxAtStartPar
Bank SanktPeterburg

\sphinxAtStartPar
Bank Sovetskiy

\sphinxAtStartPar
Bank Tochka

\sphinxAtStartPar
Bank Transportniy

\sphinxAtStartPar
Bank URALSIB

\sphinxAtStartPar
Bank Finservis

\sphinxAtStartPar
Bankru

\sphinxAtStartPar
BarklaysBank

\sphinxAtStartPar
Belgorodsotsbank

\sphinxAtStartPar
Binbank

\sphinxAtStartPar
VTB

\sphinxAtStartPar
VUZBank

\sphinxAtStartPar
Vneshprombank

\sphinxAtStartPar
Vozrozhdenie Bank

\sphinxAtStartPar
Vostochniy ekspress

\sphinxAtStartPar
Vserossiyskiy bank razvitiya regionov

\sphinxAtStartPar
Vyatkabank

\sphinxAtStartPar
GLOBEKSBANK

\sphinxAtStartPar
GUTA Bank

\sphinxAtStartPar
Gazprombank

\sphinxAtStartPar
Gazprombank Dop karta

\sphinxAtStartPar
Dalnevostochniy Bank

\sphinxAtStartPar
Evroplan

\sphinxAtStartPar
EvrositiBank

\sphinxAtStartPar
Ekaterinburgskiy Munitsipalniy Bank

\sphinxAtStartPar
Zapsibkombank

\sphinxAtStartPar
Investbank

\sphinxAtStartPar
Interkommerts

\sphinxAtStartPar
Interprogressbank

\sphinxAtStartPar
Kedr

\sphinxAtStartPar
Koltso Urala

\sphinxAtStartPar
KreditEvropaBank

\sphinxAtStartPar
Kukuruza

\sphinxAtStartPar
Lipetskkombank

\sphinxAtStartPar
LokoBank

\sphinxAtStartPar
MDM Bank

\sphinxAtStartPar
MINBank

\sphinxAtStartPar
MTS bank

\sphinxAtStartPar
Masterbank

\sphinxAtStartPar
Metkombank

\sphinxAtStartPar
Moskovskiy kreditniy

\sphinxAtStartPar
Moskomprivatbank

\sphinxAtStartPar
NB Trast

\sphinxAtStartPar
Nefteprombank

\sphinxAtStartPar
Noviy Simvol

\sphinxAtStartPar
Nomos Bank

\sphinxAtStartPar
OTP Bank

\sphinxAtStartPar
Perviy Respublikanskiy Bank

\sphinxAtStartPar
Pochta Bank

\sphinxAtStartPar
Promsvyazbank

\sphinxAtStartPar
Rayffayzen Bank

\sphinxAtStartPar
Regionalniy bank razvitiya

\sphinxAtStartPar
Roketbank

\sphinxAtStartPar
RosEvroBank

\sphinxAtStartPar
Rosbank

\sphinxAtStartPar
RosselhozBank

\sphinxAtStartPar
Rossiya

\sphinxAtStartPar
RostFinans

\sphinxAtStartPar
Russkiy Standart

\sphinxAtStartPar
SKBBank

\sphinxAtStartPar
SMP Bank

\sphinxAtStartPar
Sberknizhka

\sphinxAtStartPar
Sberbank Rossii

\sphinxAtStartPar
SberbankMaestro Povolzhe

\sphinxAtStartPar
Svyaznoy Bank

\sphinxAtStartPar
SvyazBank

\sphinxAtStartPar
Severgazbank

\sphinxAtStartPar
Sitibank

\sphinxAtStartPar
Sobinbank

\sphinxAtStartPar
Solidarnost

\sphinxAtStartPar
Surgutneftegazbank

\sphinxAtStartPar
TAATTA

\sphinxAtStartPar
Tatfondbank

\sphinxAtStartPar
Tachbank

\sphinxAtStartPar
Tinkoff

\sphinxAtStartPar
TransKreditBank

\sphinxAtStartPar
Trastbank

\sphinxAtStartPar
Ural FD

\sphinxAtStartPar
UralPromBank

\sphinxAtStartPar
UralTransBank

\sphinxAtStartPar
Uralskiy bank rekonstruktsii i razvitiya

\sphinxAtStartPar
FK Otkritie bivsh NOMOSBank

\sphinxAtStartPar
FONDSERVISBANK

\sphinxAtStartPar
HantiMansiyskiy Bank

\sphinxAtStartPar
Houm Kredit

\sphinxAtStartPar
Tsentrinvest

\sphinxAtStartPar
Chelindbank

\sphinxAtStartPar
Chelyabinvestbank

\sphinxAtStartPar
Ekspress

\sphinxAtStartPar
EnergoMashBank

\sphinxAtStartPar
Yuniastrum Bank

\sphinxAtStartPar
Yunikredit Bank

\sphinxAtStartPar
YandeksDengi


\section{Thailand}
\label{\detokenize{banks:thailand}}
\sphinxAtStartPar
KASIKORNBANK


\section{Ukraine}
\label{\detokenize{banks:ukraine}}
\sphinxAtStartPar
VAB Bank

\sphinxAtStartPar
ABank

\sphinxAtStartPar
AlfaBank

\sphinxAtStartPar
AlfaBank Ukraina

\sphinxAtStartPar
BROKBIZNESBANK

\sphinxAtStartPar
Dongorbank

\sphinxAtStartPar
Ekspressbank

\sphinxAtStartPar
Industrial

\sphinxAtStartPar
KREDOBANK

\sphinxAtStartPar
Mihaylvskiy

\sphinxAtStartPar
OTP Bank

\sphinxAtStartPar
OschadBank

\sphinxAtStartPar
PUMB

\sphinxAtStartPar
Petrokommerts Ukraina

\sphinxAtStartPar
PrivatBank

\sphinxAtStartPar
ProKreditBank

\sphinxAtStartPar
Prominvestbank

\sphinxAtStartPar
Rayffayzenbank Aval

\sphinxAtStartPar
Sberbank Rossii v Ukraine

\sphinxAtStartPar
UkrSibBank

\sphinxAtStartPar
Ukreksmbank

\sphinxAtStartPar
Ukrsotsbank


\section{United Arab Emirates}
\label{\detokenize{banks:united-arab-emirates}}
\sphinxAtStartPar
Emirates Islamic bank

\sphinxAtStartPar
Emirates NBD


\section{United States}
\label{\detokenize{banks:united-states}}
\sphinxAtStartPar
First National Bank

\sphinxAtStartPar
Guardian Alert General

\sphinxAtStartPar
Pendleton Community Bank

\sphinxAtStartPar
Town Bank

\sphinxAtStartPar
UniBank


\section{Uzbekistan}
\label{\detokenize{banks:uzbekistan}}
\sphinxAtStartPar
Uzcard


\section{Vietnam}
\label{\detokenize{banks:vietnam}}
\sphinxAtStartPar
Australia and New Zealand Banking Group

\sphinxstepscope


\chapter{Terms and definitions}
\label{\detokenize{glossary:terms-and-definitions}}\label{\detokenize{glossary:chapter-index}}\label{\detokenize{glossary::doc}}\begin{description}
\sphinxlineitem{contractor\index{contractor@\spxentry{contractor}|spxpagem}\phantomsection\label{\detokenize{glossary:term-contractor}}}
\sphinxAtStartPar
Contractor is a payer or payee.

\sphinxlineitem{exchange node\index{exchange node@\spxentry{exchange node}|spxpagem}\phantomsection\label{\detokenize{glossary:term-exchange-node}}}
\sphinxAtStartPar
Exchange node or node is a device used by teamwork member.

\sphinxlineitem{split\index{split@\spxentry{split}|spxpagem}\phantomsection\label{\detokenize{glossary:term-split}}}
\sphinxAtStartPar
Transaction details are called split. Split has its own category, project and person for each line.

\sphinxlineitem{technical category\index{technical category@\spxentry{technical category}|spxpagem}\phantomsection\label{\detokenize{glossary:term-technical-category}}}
\sphinxAtStartPar
Technical category is category having neither \DUrole{bbproperty}{Revenue} nor \DUrole{bbproperty}{Expense}
options.

\end{description}


\chapter{Indices and tables}
\label{\detokenize{index:indices-and-tables}}\begin{itemize}
\item {} 
\sphinxAtStartPar
\DUrole{xref,std,std-ref}{genindex}

\item {} 
\sphinxAtStartPar
\DUrole{xref,std,std-ref}{search}

\end{itemize}



\renewcommand{\indexname}{Index}
\printindex
\end{document}