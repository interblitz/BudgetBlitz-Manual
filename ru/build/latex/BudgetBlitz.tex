%% Generated by Sphinx.
\def\sphinxdocclass{report}
\documentclass[a4paper,10pt,russian]{sphinxmanual}
\ifdefined\pdfpxdimen
   \let\sphinxpxdimen\pdfpxdimen\else\newdimen\sphinxpxdimen
\fi \sphinxpxdimen=.75bp\relax

\PassOptionsToPackage{warn}{textcomp}
\usepackage[utf8]{inputenc}
\ifdefined\DeclareUnicodeCharacter
% support both utf8 and utf8x syntaxes
\edef\sphinxdqmaybe{\ifdefined\DeclareUnicodeCharacterAsOptional\string"\fi}
  \DeclareUnicodeCharacter{\sphinxdqmaybe00A0}{\nobreakspace}
  \DeclareUnicodeCharacter{\sphinxdqmaybe2500}{\sphinxunichar{2500}}
  \DeclareUnicodeCharacter{\sphinxdqmaybe2502}{\sphinxunichar{2502}}
  \DeclareUnicodeCharacter{\sphinxdqmaybe2514}{\sphinxunichar{2514}}
  \DeclareUnicodeCharacter{\sphinxdqmaybe251C}{\sphinxunichar{251C}}
  \DeclareUnicodeCharacter{\sphinxdqmaybe2572}{\textbackslash}
\fi
\usepackage{cmap}
\usepackage[T1]{fontenc}
\usepackage{amsmath,amssymb,amstext}
\usepackage[russian]{babel}

\usepackage[Sonny]{fncychap}
\ChNameVar{\Large\normalfont\sffamily}
\ChTitleVar{\Large\normalfont\sffamily}
\usepackage{sphinx}

\fvset{fontsize=\small}
\usepackage{geometry}

% Include hyperref last.
\usepackage{hyperref}
% Fix anchor placement for figures with captions.
\usepackage{hypcap}% it must be loaded after hyperref.
% Set up styles of URL: it should be placed after hyperref.
\urlstyle{same}

\addto\captionsrussian{\renewcommand{\figurename}{Рис.\@ }}
\makeatletter
\def\fnum@figure{\figurename\thefigure{}}
\makeatother
\addto\captionsrussian{\renewcommand{\tablename}{Таблица }}
\makeatletter
\def\fnum@table{\tablename\thetable{}}
\makeatother
\addto\captionsrussian{\renewcommand{\literalblockname}{Список}}

\addto\captionsrussian{\renewcommand{\literalblockcontinuedname}{продолжение с предыдущей страницы}}
\addto\captionsrussian{\renewcommand{\literalblockcontinuesname}{continues on next page}}
\addto\captionsrussian{\renewcommand{\sphinxnonalphabeticalgroupname}{Non-alphabetical}}
\addto\captionsrussian{\renewcommand{\sphinxsymbolsname}{Символы}}
\addto\captionsrussian{\renewcommand{\sphinxnumbersname}{Numbers}}

\addto\extrasrussian{\def\pageautorefname{страница}}

\setcounter{tocdepth}{1}

\usepackage{bbstyle}

\title{Блиц Бюджет: Документация}
\date{авг. 19, 2021}
\release{2.7}
\author{Басин Михаил}
\newcommand{\sphinxlogo}{\vbox{}}
\renewcommand{\releasename}{Выпуск}
\makeindex
\begin{document}

\ifdefined\shorthandoff
  \ifnum\catcode`\=\string=\active\shorthandoff{=}\fi
  \ifnum\catcode`\"=\active\shorthandoff{"}\fi
\fi

\pagestyle{empty}
\sphinxmaketitle
\pagestyle{plain}
\sphinxtableofcontents
\pagestyle{normal}
\phantomsection\label{\detokenize{index::doc}}



\chapter{Используемые обозначения}
\label{\detokenize{notations:id1}}\label{\detokenize{notations::doc}}
Меню: \sphinxmenuselection{Действия \(\rightarrow\) Профили (Настройки)}

Кнопка: \DUrole{bbbutton}{Настройки импорта SMS и PUSH}

Выпадающий список: \DUrole{bbspinner}{Настройки отчета}

Наименование справочника, отчета: \DUrole{bbmeta}{Настройки импорта SMS}

Элемент справочника: \DUrole{bbitem}{Персональный}

Раздел в списке свойств карточки элемента: \DUrole{bbsection}{Вид}

Свойство элемента: \DUrole{bbproperty}{Наименование}

Значение, введенное вручную:  \DUrole{bbvalue}{Сводка по одному счету}

Переменная: \DUrole{bbvar}{biz.interblitz.intent.CONVERT\_TEXT\_TO\_NEW\_TRANSACTION}


\chapter{Предисловие}
\label{\detokenize{preface:id1}}\label{\detokenize{preface::doc}}
Хотелось бы сразу обратить Ваше внимание на основную проблему, с которой сталкиваются пользователи приложения Блиц Бюджет для Android. Это — высокий порог вхождения,
что выражается в необходимости потратить время на изучение приложения. Поэтому запланируйте время, оно не будет потеряно даром, ведь впоследствии
пользователи указывают, что о потраченном времени не жалеют.


\section{Введение}
\label{\detokenize{preface:id2}}
Руководство поможет Вам начать пользоваться приложением Блиц Бюджет для Android. Руководство не является исчерпывающим, но оно
постоянно дополняется и развивается вместе с приложением. Критика, замечания и предложения приветствуются,
см. {\hyperref[\detokenize{preface:id5}]{\sphinxcrossref{Обратная связь}}} (\autopageref*{\detokenize{preface:id5}}).


\section{Дополнительные источники информации}
\label{\detokenize{preface:id3}}
Вопросы и ответы на русском языке: \sphinxurl{http://qa.bbmoney.biz/ru/}

Вопросы и ответы на английском языке: \sphinxurl{http://qa.bbmoney.biz/en/}

Обсуждение приложения на 4PDA: \sphinxurl{http://4pda.ru/forum/index.php?showtopic=658215}

Предыдущее официальное руководство: \sphinxurl{http://interblitz.biz/projects/blitz-0035/wiki/Blitz\_Budget}

Предыдущее руководство для новичков: \sphinxurl{http://bbmoney.biz.ru/assets/budgetblitz-manual.pdf}


\section{Варианты руководства}
\label{\detokenize{preface:id4}}
HTML: \sphinxurl{http://bbmoney.biz/ru/manual/index.html}

PDF: \sphinxurl{http://bbmoney.biz/ru/assets/budgetblitz-user-manual.pdf}


\section{Обратная связь}
\label{\detokenize{preface:id5}}
Автор: Басин Михаил

Контакты: \sphinxhref{mailto:basin.michael@gmail.com}{basin.michael@gmail.com}


\chapter{О приложении}
\label{\detokenize{about:id1}}\label{\detokenize{about::doc}}
Приложение Блиц Бюджет для Android служит для автоматизации учета и планирования:
\begin{itemize}
\item {} 
персональных финансов;

\item {} 
финансов очень малого бизнеса (индивидуальные предприниматели);

\item {} 
финансов малого бизнеса.

\end{itemize}

\noindent\sphinxincludegraphics[width=0.250\linewidth]{{about-010-main-screen}.png}

\noindent\sphinxincludegraphics[width=0.250\linewidth]{{about-170-pie}.png}

\noindent\sphinxincludegraphics[width=0.250\linewidth]{{about-190-lines}.png}


\section{Ключевые особенности}
\label{\detokenize{about:id2}}
Совмещение учета личных финансов и финансов предприятия.

Тщательный учет финансов — поддержка статей, контрагентов (плательщиков и получателей), персон и проектов.

Автоматическое распознавание SMS и Push уведомлений банков — определение сумм, статей, проектов, персон,
получателей и плательщиков, выделение комиссий из суммы платежа, автоматическая корректировка баланса,
поддержка 160+ банков разных стран, см. {\hyperref[\detokenize{banks:chapter-supported-banks}]{\sphinxcrossref{\DUrole{std,std-ref}{Поддерживаемые банки}}}} (\autopageref*{\detokenize{banks:chapter-supported-banks}}).

Отображение ключевых показателей на главном экране.

«Умные» значения по умолчанию при вводе операций.

Виджет для быстрого ввода и редактирования операций, с возможностью вывода настраиваемого среза данных (виджет как отчет).

Децентрализованная коллективная работа с настраиваемыми правами доступа.

Клиент для ПК (доступ к данным программы через web браузер).

API для приема данных от других приложений.

Разнообразные аналитические отчеты.

Напоминания на базе отчетов.


\section{Интересные решения, реализованные в программе}
\label{\detokenize{about:id3}}
Подсистема распознавания уведомлений (SMS и Push) банков
\begin{itemize}
\item {} 
Автоматическое распознавание аналитик (статья, контрагент, проект, персона);

\item {} 
Удобный подбор ключевых слов непосредственно из SMS и Push-уведомлений;

\item {} 
Автоматический расчет курсов для валютных операций;

\item {} 
Автоматическое занесение переводов между счетами;

\item {} 
Возможность создания собственной настройки импорта SMS и Push-уведомлений.

\end{itemize}

Подсистема отчетности
\begin{itemize}
\item {} 
Поддержка упрощенной технологии OLAP при формировании отчетов, т.е. возможности выбирать одновременно несколько срезов данных, включая временные периоды;

\item {} 
Поддержка функции «Drilldown», т.е. возможности расшифровки и редактирования исходных данных;

\item {} 
Использование виджетов для отображения коротких сводных отчетов;

\item {} 
Использования ярлыков для быстрого запуска отчетов с сохраненными настройками.

\item {} 
Напоминания на базе отчетов с возможностью быстрого доступа к готовому отчету из напоминания.

\end{itemize}

Подсистема коллективной работы
\begin{itemize}
\item {} 
Использование механизма обмена данными для коллективной работы, регистрация на сайте разработчика не требуется. Общей базы данных не существует, у каждого участника обмена своя база данных.

\item {} 
Гибкая система настройки прав и областей данных для обмена. Можно синхронизировать операции между участниками обмена только по одному счету, проекту и т. д.

\item {} 
Неограниченное количество участников обмена данными.

\end{itemize}

Подсистема доступа с персонального компьютера
\begin{itemize}
\item {} 
Клиент работает на операционных системах Windows, Linux, Mac и пр. Все что нужно для работы - это современный браузер. Поддерживаются Internet Explorer 8+, Google Chrome, Apple Safari, Mozilla Firefox, Opera.

\end{itemize}


\chapter{Как устроено приложение}
\label{\detokenize{intro:id1}}\label{\detokenize{intro::doc}}

\section{Учет денежных средств}
\label{\detokenize{intro:id2}}
Все движения денежных средств учитываются при помощи операций (финансовых транзакций). Каждая операция содержит
четыре аналитики: статья (категория), проект, плательщик (получатель) и персона. Операции могут быть
фактическими или плановыми, разовыми или постоянными. Постоянные операции повторяются с заданной периодичностью, как правило
такие операции являются плановыми, но также могут быть и фактическими.

Любую операцию можно разбить на несколько подопераций, такая операция называется сплитом, см. {\hyperref[\detokenize{glossary:term}]{\sphinxtermref{\DUrole{xref,std,std-term}{сплит}}}}.


\section{Структура справочников}
\label{\detokenize{intro:id3}}
Операции принадлежат определенному счету. Это может быть банковский счет, счет с электронными деньгами, наличные в портмоне,
или что-то другое. Каждый счет имеет свою валюту, которая может отличаться от валюты операции.

В свою очередь, все счета разбиты по портфелям. Каждый портфель также имеет свою валюту, которая может отличаться от валюты счета.

Но это еще не все. Портфели разбиты по типам портфелям. Тип портфеля можно считать аналогом организации.
Если Вы ведете только домашнюю бухгалтерию, то у Вас будет только один тип портфеля — персональный. Если Вы
дополнительно ведете учет в организации, то два — персональный и малый бизнес. В некоторых случаях типов
портфелей может быть и больше.

Статьи, контрагенты, проекты и персоны привязаны к типу портфеля. Схематично все справочники можно представить на рисунке:

\noindent{\hspace*{\fill}\sphinxincludegraphics[width=0.750\linewidth]{{directories_structure_ru}.png}\hspace*{\fill}}

\begin{sphinxadmonition}{note}{Примечание:}
Любой справочник можно отредактировать. Например, можно добавить валюту, статью и т.п. Нет никаких ограничений!
\end{sphinxadmonition}


\section{Отличие между плательщиками (получателями) и персонами}
\label{\detokenize{intro:id4}}
Под плательщиками и получателями в программе понимается вторая сторона в денежной операции. Часто это сторону называют
контрагентом. Операции не может быть без контрагента, исключение составляет перевод между своими счетами. Если Вы даете, например,
ребенку некую сумму денег, то ребенок является контрагентом и должен быть занесен в справочник \DUrole{bbmeta}{Плательщики и получатели}.

Персоны, а также категории (статьи) и проекты являются расшифровкой операции. Так например, если Вы покупаете, одежду для
ребенка в магазине, то контрагентом является магазин, а ребенок в этой операции — персоной.

Контрагенты и персоны можно связать между собой. Для этого в карточке контрагента можно выбрать конкретную персону. Тогда,
при выборе контрагента, в операцию также будет попадать и указанная персона. Например, ребенок будет и контрагентом и персоной.

Настроив учет таким образом, можно увидеть общую сумму потраченную на содержание ребенка (аналитика по персоне)
и отдельно сумму денежных средств непосредственно переданных ребенку (аналитика по контрагенту).


\chapter{Начало работы}
\label{\detokenize{getting-started:id1}}\label{\detokenize{getting-started::doc}}
В этой главе предлагается определенная последовательность действий по настройке приложения. Следовать предлагаемому порядку совсем не обязательно. Помните, что любой из параметров всегда можно изменить позже.


\section{Настройка основных параметров}
\label{\detokenize{getting-started:id2}}
После первого запуска отредактируйте основные настройки приложения.

\noindent\sphinxincludegraphics[width=0.250\linewidth]{{gettingstarted-010-main-screen}.png}

\noindent\sphinxincludegraphics[width=0.250\linewidth]{{gettingstarted-020-click-on-actions-menu}.png}

\noindent\sphinxincludegraphics[width=0.250\linewidth]{{gettingstarted-030-settings}.png}

Здесь можно:
\begin{itemize}
\item {} 
задать графический ключ для входа в приложение (наподобие ключа, который используется для разблокировки меню смартфона);

\item {} 
включить/выключить анализ входящих SMS и push-уведомлений банков;

\item {} 
включить/выключить синхронизацию между различными устройствами;

\item {} 
установить/отключить по умолчанию отрицательную сумму для новых операций (для тех, кто расходы заносит чаще, чем доходы);

\item {} 
установить основную валюту для профиля, а также источник загрузки курсов валют;

\item {} 
настроить автоматические резервное копирование с отправкой данных в сервис Dropbox;

\item {} 
включить/отключить напоминания о предстоящих платежах;

\item {} 
задать звуковые уведомления при обработке смс от банков.

\end{itemize}

После настройки основных параметров можно приступать ко второму этапу настройки, основные пункты которого
расположены прямо на главном экране.


\section{Загрузка настроек банков}
\label{\detokenize{getting-started:id3}}
Этот раздел предназначен для тех кто планирует использовать функцию автоматического создания операций из коротких сообщений банков,
платежных систем или установленных на устройстве приложений.

Для автоматического импорта SMS уведомлений:
\begin{enumerate}
\def\theenumi{\arabic{enumi}}
\def\labelenumi{\theenumi .}
\makeatletter\def\p@enumii{\p@enumi \theenumi .}\makeatother
\item {} 
должен быть отмечен соответствующий флажок в активном профиле, меню \sphinxmenuselection{Действия \(\rightarrow\) Профили (Настройки)};

\item {} 
приложению должен быть предоставлен доступ к SMS (Android 6 и выше).

\end{enumerate}

Для автоматического импорта push-уведомлений в активном профиле, меню \sphinxmenuselection{Действия \(\rightarrow\) Профили (Настройки)}:
\begin{enumerate}
\def\theenumi{\arabic{enumi}}
\def\labelenumi{\theenumi .}
\makeatletter\def\p@enumii{\p@enumi \theenumi .}\makeatother
\item {} 
должны быть выбраны пакеты, уведомления которых Блиц Бюджет для Android будет импортировать:

\item {} 
должны быть предоставлены дополнительные разрешения.

\end{enumerate}

Чтобы загрузить настройки Вашего банка нажмите кнопку \DUrole{bbbutton}{Настройки импорта SMS и PUSH}. В дальнейшем загрузить
настройки можно будет через меню \sphinxmenuselection{Действия \(\rightarrow\) Импорт \(\rightarrow\) Настройки импорта SMS и Push} или из справочника
\DUrole{bbmeta}{Настройки импорта SMS}.

\noindent\sphinxincludegraphics[width=0.250\linewidth]{{gettingstarted-050-click-on-import-tunes}.png}

\noindent\sphinxincludegraphics[width=0.250\linewidth]{{gettingstarted-060-available-sms-tunes}.png}


\section{Настройка портфелей и счетов}
\label{\detokenize{getting-started:id4}}
После установки программа содержит три типа портфелей \DUrole{bbitem}{Персональный}, \DUrole{bbitem}{Малый бизнес} и \DUrole{bbitem}{Универсальный},
один персональный портфель \DUrole{bbitem}{Кошелек}, два счета \DUrole{bbitem}{Карта} и \DUrole{bbitem}{Наличные} в портфеле \DUrole{bbitem}{Кошелек} и предзаполненный
список категорий.

В зависимости от того, какому типу портфеля принадлежит операция, при редактировании отображаются те или иные аналитики.  Так например,
при ведении домашней бухгалтерии используется один список статей, а при учете финансов малого бизнеса — другой.
Тем не менее, есть и общие для обоих учетов статьи и другие аналитики. Они принадлежат универсальному типу портфеля и
отображаются всегда, вне зависимости от того, какому типу портфеля принадлежит операция.

Портфель можно рассматривать как группу счетов. Приложение группирует счета на главном экране по портфелям и рассчитывает
по ним сводные итоги.

Создайте нужное количество портфелей и счетов. Для использования функции автоматического создания операций из коротких
сообщений см. {\hyperref[\detokenize{account-identities:chapter-account-identities}]{\sphinxcrossref{\DUrole{std,std-ref}{Настройка счетов для импорта SMS и push-уведомлений}}}} (\autopageref*{\detokenize{account-identities:chapter-account-identities}}).

Теперь можно импортировать SMS, данные из файла в форматах \sphinxhref{https://ru.wikipedia.org/wiki/CSV}{CSV} и \sphinxhref{https://en.wikipedia.org/wiki/Open\_Financial\_Exchange}{OFX}, или просто внести начальные остатки. Обратите внимание, что как только будет занесена хотя
бы одна операция, главный экран примет вид сводки по счетам. Тем менее все функции настройки параметров будут доступны из главного меню приложения.


\section{Первоначальный импорт данных}
\label{\detokenize{getting-started:id5}}
Перед импортом ранее полученных уведомлений от банков проверьте настройки счетов согласно главе {\hyperref[\detokenize{account-identities:chapter-account-identities}]{\sphinxcrossref{\DUrole{std,std-ref}{Настройка счетов для импорта SMS и push-уведомлений}}}} (\autopageref*{\detokenize{account-identities:chapter-account-identities}}). Затем нажмите \DUrole{bbbutton}{SMS и PUSH уведомления}
в разделе \DUrole{bbsection}{Импорт} или выберите пункт меню \sphinxmenuselection{Действия \(\rightarrow\) Импорт \(\rightarrow\) SMS и PUSH уведомления}, укажите счет
и импортируйте SMS. Подробней процесс импорта разобран в главе {\hyperref[\detokenize{import:chapter-import}]{\sphinxcrossref{\DUrole{std,std-ref}{Импорт данных}}}} (\autopageref*{\detokenize{import:chapter-import}}) и \sphinxhref{http://qa.bbmoney.biz/ru/index.php?qa=13\&qa\_1=\%D0\%BA\%D0\%B0\%D0\%BA-\%D0\%B2\%D1\%80\%D1\%83\%D1\%87\%D0\%BD\%D1\%83\%D1\%8E-\%D0\%B8\%D0\%BC\%D0\%BF\%D0\%BE\%D1\%80\%D1\%82\%D0\%B8\%D1\%80\%D0\%BE\%D0\%B2\%D0\%B0\%D1\%82\%D1\%8C-sms\&show=13\#q13}{вопросах и ответах}.

Кроме того, приложение может импортировать первоначальные данные из файлов в формате \sphinxhref{https://ru.wikipedia.org/wiki/CSV}{CSV} или \sphinxhref{https://en.wikipedia.org/wiki/Open\_Financial\_Exchange}{OFX}. Перед импортом данных в формате CSV
проверьте и при необходимости отредактируйте исходный файл согласно главе {\hyperref[\detokenize{import:chapter-import}]{\sphinxcrossref{\DUrole{std,std-ref}{Импорт данных}}}} (\autopageref*{\detokenize{import:chapter-import}}). Файл в формате OFX не
требует каких-либо предварительных манипуляций.


\section{Ввод начальных остатков и кредитного лимита}
\label{\detokenize{getting-started:id7}}
Начальный остаток по счету заносится операцией. Дата операции может любой, но желательно, чтобы операция была первой в списке операций. В качестве статьи
следует указать статью \DUrole{bbitem}{Ввод начальных остатков}.

\noindent\sphinxincludegraphics[width=0.250\linewidth]{{initialbalance-010-initial-transaction}.png}

\noindent\sphinxincludegraphics[width=0.250\linewidth]{{initialbalance-020-initial-budget-item}.png}

Кредитный лимит также заносится операцией. Желательно, чтобы дата операции совпадала с датой установки лимита банком. В качестве статьи
следует указать статью \DUrole{bbitem}{Изменение кредитного лимита}. Обратите внимание, что это техническая статья,
у нее отключены признаки \DUrole{bbproperty}{Revenue} и \DUrole{bbproperty}{Expense}. Подробней причины ввода кредитного лимита операциями
рассмотрены в \sphinxhref{http://qa.bbmoney.biz/ru/index.php?qa=93\&qa\_1=\%D0\%B7\%D0\%B0\%D0\%B4\%D0\%B0\%D1\%82\%D1\%8C-\%D0\%BA\%D1\%80\%D0\%B5\%D0\%B4\%D0\%B8\%D1\%82\%D0\%BD\%D1\%8B\%D0\%B9-\%D0\%BB\%D0\%B8\%D0\%BC\%D0\%B8\%D1\%82-\%D0\%BD\%D0\%BE\%D0\%B2\%D0\%BE\%D0\%B3\%D0\%BE-\%D1\%81\%D1\%87\%D0\%B5\%D1\%82\%D0\%B0-\%D1\%81\%D1\%87\%D0\%B5\%D1\%82\%D0\%B0-\%D0\%BA\%D0\%BE\%D1\%82\%D0\%BE\%D1\%80\%D0\%BE\%D0\%BC\%D1\%83-\%D0\%BE\%D0\%BF\%D0\%B5\%D1\%80\%D0\%B0\%D1\%86\%D0\%B8\%D0\%B8}{вопросах и ответах (Как задать кредитный лимит)}.

\noindent\sphinxincludegraphics[width=0.250\linewidth]{{initialbalance-040-credit-limit-transaction}.png}

\noindent\sphinxincludegraphics[width=0.250\linewidth]{{initialbalance-050-credit-limit--budget-item}.png}

Остатки по каждому долгу лучше внести двумя операциями. Например, если Ваш долг составляет 1000 руб., то следует:
\begin{enumerate}
\def\theenumi{\arabic{enumi}}
\def\labelenumi{\theenumi .}
\makeatletter\def\p@enumii{\p@enumi \theenumi .}\makeatother
\item {} 
ввести приходную (положительную) операцию на сумму 1000 руб. по статье \DUrole{bbitem}{Кредиты (Я должен)} и указать плательщика или персону — кому должны.

\item {} 
ввести расходную (отрицательную) операцию на сумму 1000 руб. по статье \DUrole{bbitem}{00 Не указано}, либо задать статью, если известно, на что были потрачены средства.

\end{enumerate}

В итоге баланс будет 0, а отчет Долги покажет долг.


\chapter{Главный экран}
\label{\detokenize{main-screen:chapter-main-screen}}\label{\detokenize{main-screen:id1}}\label{\detokenize{main-screen::doc}}

\section{Описание}
\label{\detokenize{main-screen:id2}}
Главный экран Блиц Бюджет для Android содержит сводку по портфелям и счетам. Если в приложении используется несколько
типов портфелей, то на экране будет выведено несколько сводок. В примерах ниже рассматривается вариант с
одним типом портфеля \DUrole{bbitem}{Персональный}.

В сводке отображаются остатки по каждому счету и всему портфелю в целом. Если для счета задан кредитный лимит,
то также отображается доступная сумма.

Ниже остатков расположены фактические и планируемые суммы расходов и доходов. Также отдельно выводятся суммы переводов,
если таковые были в выбранном периоде.

\noindent\sphinxincludegraphics[width=0.250\linewidth]{{mainscreen-010-main-screen}.png}

\noindent\sphinxincludegraphics[width=0.250\linewidth]{{mainscreen-014-main-screen-swipe-left}.png}

\noindent\sphinxincludegraphics[width=0.250\linewidth]{{mainscreen-015-main-screen-transactions}.png}

Кроме того, слева от набора сводок отображается список операций, в котором отображаются операции без разделения
на типы портфелей.


\section{Выбор периода}
\label{\detokenize{main-screen:id3}}
Период можно изменить при помощи редактора периодов, который расположен в верхней части экрана. Редактор
поддерживает жесты перелистывания и выбора.

\noindent\sphinxincludegraphics[width=0.250\linewidth]{{mainscreen-089-main-screen-dates-range-swipe-left}.png}

\noindent\sphinxincludegraphics[width=0.250\linewidth]{{mainscreen-090-main-screen-dates-range-swipe}.png}

\noindent\sphinxincludegraphics[width=0.250\linewidth]{{mainscreen-100-main-screen-dates-range-spinner}.png}


\section{Настройки}
\label{\detokenize{main-screen:id4}}
Настройки сводки расположены в нижней части экрана. В настройках можно изменить группировку, заданную по умолчанию,
отредактировать фильтр и изменить период. В фильтре можно задать отбор по портфелям, счетам, валютам и отключить
отображение плана.

В следующем примере задается фильтр по счетам.

\noindent\sphinxincludegraphics[width=0.250\linewidth]{{mainscreen-020-main-screen-bottom-sheet-opening}.png}

\noindent\sphinxincludegraphics[width=0.250\linewidth]{{mainscreen-030-main-screen-bottom-sheet-open}.png}

\noindent\sphinxincludegraphics[width=0.250\linewidth]{{mainscreen-040-main-screen-filter}.png}

\noindent\sphinxincludegraphics[width=0.250\linewidth]{{mainscreen-050-main-screen-filter-account}.png}

\noindent\sphinxincludegraphics[width=0.250\linewidth]{{mainscreen-060-main-screen-filter-apply}.png}

\noindent\sphinxincludegraphics[width=0.250\linewidth]{{mainscreen-065-main-screen-filter-applied}.png}

Теперь на главном экране отображается только счет \DUrole{bbitem}{Наличные}.


\section{Сохранение настроек}
\label{\detokenize{main-screen:id5}}
Измененные настройки можно сохранить для последующего использования. Для этого следует выбрать
\DUrole{bbspinner}{Настройки отчета} и создать новую настройку. Значения фильтра будут автоматически
скопированы. Остается указать название настройки, например \DUrole{bbvalue}{Сводка по одному счету}, и
нажать \DUrole{bbbutton}{Сохранить}.

\noindent\sphinxincludegraphics[width=0.250\linewidth]{{mainscreen-070-main-screen-select-new-setting}.png}

\noindent\sphinxincludegraphics[width=0.250\linewidth]{{mainscreen-080-main-screen-setting-save}.png}

Приложение позволяет иметь одновременно несколько сохраненных настроек и
при запуске загружает для главного экрана последнюю использованную настройку.


\chapter{Справочники}
\label{\detokenize{directories:chapter-directories}}\label{\detokenize{directories:id1}}\label{\detokenize{directories::doc}}
Любой справочник можно открыть из панели быстрых кнопок или меню \sphinxmenuselection{Действия \(\rightarrow\) Справочники} в зависимости от того,
какой экран открыт в данный момент.

\noindent\sphinxincludegraphics[width=0.250\linewidth]{{directories-010-select-directories}.png}

\noindent\sphinxincludegraphics[width=0.250\linewidth]{{directories-020-menu-directories}.png}


\section{Типы портфелей}
\label{\detokenize{directories:id2}}
Типы портфелей служат для разделения аналитик между портфелями. Например для персонального портфеля
может использоваться один набор статей, а для малого бизнеса совсем другой.
Тип портфеля учитывается при подборе аналитик (статей, проектов, плательщиков, получателей, персон) в
момент редактирования операции.

\noindent\sphinxincludegraphics[width=0.250\linewidth]{{directories-030-types-of-portfolio}.png}

\noindent\sphinxincludegraphics[width=0.250\linewidth]{{directories-040-portfolios}.png}

\noindent\sphinxincludegraphics[width=0.250\linewidth]{{directories-050-accounts}.png}

Следует обратить внимание на тип портфеля \DUrole{bbitem}{Универсальный}. В старых версиях этот тип называется \DUrole{bbitem}{00 Не указано}.
Аналитики этого типа портфеля всегда доступны для использования. Например, статья \DUrole{bbitem}{Перевод}
принадлежит портфелю \DUrole{bbitem}{Универсальный} и может быть выбрана в любой операции вне зависимости от выбранного счета и
связанного с ним типа портфеля.

\begin{sphinxadmonition}{note}{Примечание:}
По умолчанию, аналитики портфеля \DUrole{bbitem}{Универсальный} доступны для всех счетов
\end{sphinxadmonition}

См. также {\hyperref[\detokenize{shared-transactions:chapter-shared-transactions}]{\sphinxcrossref{\DUrole{std,std-ref}{Совместное использование типов портфелей}}}} (\autopageref*{\detokenize{shared-transactions:chapter-shared-transactions}}).


\section{Портфели}
\label{\detokenize{directories:id3}}
Справочник Портфели служит для группировки счетов. Каждый портфель имеет свои валюту. Итоги и движения по портфелю
будут отображаться в указанной валюте исходя из курсов справочника Валюты.


\section{Счета}
\label{\detokenize{directories:id4}}
Счетом может быть банковский счет или карта, металлический счет,
наличные и пр. Каждый счет имеет свою валюту, которая может отличаться от валюты портфеля.

Идентификатор счета используется при импорте данных, см. {\hyperref[\detokenize{import:chapter-import}]{\sphinxcrossref{\DUrole{std,std-ref}{Импорт данных}}}} (\autopageref*{\detokenize{import:chapter-import}}). Можно указать несколько
идентификаторов, разделенных запятой. Идентификатором может также служить телефонный номер, короткое имя отправителя SMS
или идентификатор пакета Push уведомлений.

Ключевые слова счета также используются при импорте SMS. В случае операций перевода счет, найденный по идентификатору
является отправителем, а счет найденный по ключевым словам — получателем. Перевод может быть как положительным
(зачисление), так и отрицательным (списание). Ключевые слова используются только для переводов.

Например, от банка поступило SMS:

\begin{sphinxVerbatim}[commandchars=\\\{\}]
\PYG{n}{Karta} \PYG{n}{Visa2900}\PYG{o}{.} \PYG{n}{Proizvedeno} \PYG{n}{snyatie} \PYG{l+m+mf}{2000.00} \PYG{n}{RUR} \PYG{n}{ATM} \PYG{o}{.}\PYG{n}{Ostatok}\PYG{p}{:}\PYG{l+m+mf}{274.26} \PYG{n}{RUR}\PYG{o}{.} \PYG{l+m+mi}{25}\PYG{o}{/}\PYG{l+m+mi}{03}\PYG{o}{/}\PYG{l+m+mi}{14}\PYG{p}{,}\PYG{l+m+mi}{15}\PYG{p}{:}\PYG{l+m+mi}{00}\PYG{p}{:}\PYG{l+m+mf}{00.}
\end{sphinxVerbatim}

В этом случае Visa2900 является идентификатором счета \DUrole{bbitem}{Карта}, ATM — ключевой фразой счета \DUrole{bbitem}{Наличные}. При импорте SMS приложение создаст
две операции — операцию списания для счета Карта и операцию зачисления для счета Наличные.

Настройка импорта SMS определяет как именно будут распознаваться операции.
Подробнее о настройках импорта SMS см. {\hyperref[\detokenize{notifications:chapter-notifications}]{\sphinxcrossref{\DUrole{std,std-ref}{Расширенная настройка импорта SMS и push-уведомлений}}}} (\autopageref*{\detokenize{notifications:chapter-notifications}}).

Значения по умолчанию, заданные для проектов, контрагентов и персон, будут использоваться при создании операций. При
импорте и обмене данными приложение также использует эти значения.

\noindent\sphinxincludegraphics[width=0.250\linewidth]{{directories-055-accounts-continue}.png}

\noindent\sphinxincludegraphics[width=0.250\linewidth]{{directories-060-categories}.png}

\noindent\sphinxincludegraphics[width=0.250\linewidth]{{directories-070-contractors}.png}


\section{Статьи}
\label{\detokenize{directories:id5}}
Справочник \DUrole{bbmeta}{Статьи} является основным при классификации операций. Выбранная статья влияет на то, как будет учитываться
операция на главном экране, в отчетах и списке операций. В зависимости от признаков статья может быть доходной, расходной,
технической, переводной и архивной.

Признаки Доходная и Расходная влияют на сортировку статей при редактировании операции. Для доходный операций сначала
будут отображаться доходные операции, затем расходные и наоборот.

Статья может не быть ни доходной ни расходной. В этом случае статья считается технической. В качестве примера использования
технической статьи можно привести операцию изменения кредитного лимита на карте Сбербанка.
Движение денег для владельца карты в такой операции нет, но сумма на карте увеличивается или уменьшается.
Для такой операции следует выбрать техническую статью. Подробней  ввод кредитного лимита
рассмотрен в \sphinxhref{http://qa.bbmoney.biz/ru/index.php?qa=93\&qa\_1=\%D0\%B7\%D0\%B0\%D0\%B4\%D0\%B0\%D1\%82\%D1\%8C-\%D0\%BA\%D1\%80\%D0\%B5\%D0\%B4\%D0\%B8\%D1\%82\%D0\%BD\%D1\%8B\%D0\%B9-\%D0\%BB\%D0\%B8\%D0\%BC\%D0\%B8\%D1\%82-\%D0\%BD\%D0\%BE\%D0\%B2\%D0\%BE\%D0\%B3\%D0\%BE-\%D1\%81\%D1\%87\%D0\%B5\%D1\%82\%D0\%B0-\%D1\%81\%D1\%87\%D0\%B5\%D1\%82\%D0\%B0-\%D0\%BA\%D0\%BE\%D1\%82\%D0\%BE\%D1\%80\%D0\%BE\%D0\%BC\%D1\%83-\%D0\%BE\%D0\%BF\%D0\%B5\%D1\%80\%D0\%B0\%D1\%86\%D0\%B8\%D0\%B8}{вопросах и ответах (Как задать кредитный лимит)}.

По суммируемым статьям можно увидеть баланс в отчетах \DUrole{bbmeta}{Долги} и \DUrole{bbmeta}{Исполнение плана}.

Статья может иметь признак исключаемой из портфеля. Такие статьи обычно используются в операциях,
которые не изменяют остатка внутри портфеля. Если у статьи установлен такой признак,
то все движения по этой статье не будут влиять на сумму движений денежных средств за период.
В списке операций итоги по таким статьям выводятся отдельно.

Ключевые фразы используются для подбора при импорте данных. Можно указать несколько ключевых фраз, разделенных запятой,

В операции может быть указано несколько статей.

Справочник автоматически заполняется при установке приложения, однако Вы можете отредактировать его на свой вкус.


\section{Плательщики и получатели}
\label{\detokenize{directories:id7}}
Под плательщиками и получателями в программе понимается вторая сторона в денежной операции. Часто это сторону называют
контрагентом. Без контрагента операции не может быть (за исключением перевода между своими счетами). В операции может быть
указан только один контрагент.


\section{Проекты}
\label{\detokenize{directories:id8}}
Проектом может быть, например, отпуск, строительство дома, стартап и т.п. В операции может быть указано несколько проектов.

Ключевые фразы используются для подбора при импорте данных. Можно указать несколько ключевых фраз, разделенных запятой.


\section{Персоны}
\label{\detokenize{directories:id9}}
В справочник Персоны можно указать членов семьи или сотрудников предприятия. В операции может быть указано несколько персон.

Ключевые фразы используются для подбора при импорте данных. Можно указать несколько ключевых фраз, разделенных запятой.

\noindent\sphinxincludegraphics[width=0.250\linewidth]{{directories-080-projects}.png}

\noindent\sphinxincludegraphics[width=0.250\linewidth]{{directories-090-persons}.png}

\noindent\sphinxincludegraphics[width=0.250\linewidth]{{directories-100-currencies}.png}


\section{Валюты}
\label{\detokenize{directories:id10}}
Сразу после установки приложение содержит практически все мировые валюты. При необходимости, Вы можете добавить в справочник новую валюту.

Итоговые значения в разрезе портфелей рассчитываются согласно курсам валют. Курсы валют можно указывать вручную или
загружать из интернет-источников. В зависимости от настроек курсы валют загружаются из следующих источников:
Центральный Банк РФ (валюты и драг. металлы), Центральный Европейский Банк, банк Канады, Национальный Банк Республики Беларусь,
Национальный Банк Республики Казахстан, банк Израиля, BitPay (котировки валют относительно BTC), Poloniex (биржа крипто-валют).


\chapter{Финансовые операции}
\label{\detokenize{transactions:chapter-transactions}}\label{\detokenize{transactions:id1}}\label{\detokenize{transactions::doc}}

\section{Введение}
\label{\detokenize{transactions:id2}}
Любые изменения денежных средств учитываются при помощи операций. Ввод начальных остатков,
изменение кредитного лимита, списание или зачисление средств, снятие наличных в банкомате или
что-то другое — все это отражается при помощи операций. Такой
подход является наиболее гибким и позволяет хранить историю всех движений.

\noindent\sphinxincludegraphics[width=0.250\linewidth]{{transactions-010-transactions}.png}

\noindent\sphinxincludegraphics[width=0.250\linewidth]{{transactions-015-transactions-bottom-sheet-opening}.png}

\noindent\sphinxincludegraphics[width=0.250\linewidth]{{transactions-016-transactions-bottom-sheet-open}.png}

В списке операций можно использовать фильтр, который находится в подвале. Также доступен быстрый выбор периода.

\noindent\sphinxincludegraphics[width=0.250\linewidth]{{transactions-017-transactions-dates-range-swipe}.png}

\noindent\sphinxincludegraphics[width=0.250\linewidth]{{transactions-018-transactions-dates-range-spinner}.png}

\noindent\sphinxincludegraphics[width=0.250\linewidth]{{transactions-020-transaction}.png}

Операция может быть доходной или расходной. Специального признака для вида операции нет, достаточно указать
положительную или отрицательную сумму. Если операция является переводом, то в ней необходимо выбрать статью
с признаком \DUrole{bbproperty}{Исключаемая из портфеля}. Сразу после установки приложение содержит статью \DUrole{bbitem}{Перевод},
которую можно использовать при переводах.

Если операция выполнена в валюте, то следует указать валюту и курс валюты операции. Этот курс может отличаться
от курса валюты в справочнике валют. Если операция создается в результате импорта SMS или push-уведомлений, то
курс и валюта определяются автоматически.

Для подробного финансового учета следует правильно указывать статьи, проекты, плательщиков, получателей и персон.


\section{Сплиты}
\label{\detokenize{transactions:id3}}
Любую операцию можно разбить на несколько, такая операция называется {\hyperref[\detokenize{glossary:term}]{\sphinxtermref{\DUrole{xref,std,std-term}{сплит}}}}. Использовать сплиты удобно, например,
для классификации покупок в супермаркете, когда часть затрат, предположим, ушла на питание дома, часть — на
хозяйственные товары. Конечно, это далеко не единственный пример использования сплитов.

При редактировании сумма первой части сплита автоматически пересчитывается с учетом новых частей так, чтобы
общая сумма операции оставалась неизменной. Для удаления части сплита, достаточно указать сумму равную 0.

\noindent\sphinxincludegraphics[width=0.250\linewidth]{{transactionsplit-010-select-transaction}.png}

\noindent\sphinxincludegraphics[width=0.250\linewidth]{{transactionsplit-020-transaction-details}.png}

\noindent\sphinxincludegraphics[width=0.250\linewidth]{{transactionsplit-030-transaction-edit-detail}.png}

\noindent\sphinxincludegraphics[width=0.250\linewidth]{{transactionsplit-040-transaction-details-row-second}.png}

\noindent\sphinxincludegraphics[width=0.250\linewidth]{{transactionsplit-050-transaction-details-row-first}.png}


\section{Планируемые операции}
\label{\detokenize{transactions:id4}}
Операции могут быть фактическими или планируемыми. Планируемая операция отмечается флажком \DUrole{bbproperty}{План}. Такие операции
учитываются в планируемом движении денежных средств до того момента пока не перестанут быть актуальными.
Актуальность определяется по дате и времени операции. Планировать можно любые операции: расходы, доходы,
возврат долгов, накопления и др. В дальнейшем можно сравнить фактические и планируемые операции при помощи отчетов.

\noindent\sphinxincludegraphics[width=0.250\linewidth]{{transactionplan-010-transaction-set-plan}.png}


\section{Ручные переводы}
\label{\detokenize{transactions:id5}}
Переводы отражаются в приложении двумя операциями. В карточке операции предусмотрен быстрый и удобный способ
создания перевода вручную. Чтобы создать новый перевод:
\begin{enumerate}
\def\theenumi{\arabic{enumi}}
\def\labelenumi{\theenumi .}
\makeatletter\def\p@enumii{\p@enumi \theenumi .}\makeatother
\item {} 
Создайте операцию, укажите сумму.

\item {} 
Нажмите \DUrole{bbbutton}{Перевод} рядом со счетом-источником.

\item {} 
Укажите счет-получатель.

\item {} 
При необходимости отредактируйте остальные реквизиты операции.

\item {} 
Сохраните операцию.

\item {} 
Приложение автоматически обновит главный экран, сумма переводов отразится в соответствующей строке.

\end{enumerate}

\noindent\sphinxincludegraphics[width=0.250\linewidth]{{transactionstransfer-010-create-transaction}.png}

\noindent\sphinxincludegraphics[width=0.250\linewidth]{{transactionstransfer-020-transaction-edit}.png}

\noindent\sphinxincludegraphics[width=0.250\linewidth]{{transactionstransfer-030-transaction-select-transfer}.png}

\noindent\sphinxincludegraphics[width=0.250\linewidth]{{transactionstransfer-040-transaction-select-transfer-account}.png}

\noindent\sphinxincludegraphics[width=0.250\linewidth]{{transactionstransfer-045-transaction-select-transfer-account}.png}

\noindent\sphinxincludegraphics[width=0.250\linewidth]{{transactionstransfer-050-transaction-edit-save}.png}

\noindent\sphinxincludegraphics[width=0.250\linewidth]{{transactionstransfer-070-transfer-result}.png}

Переводы сохраняются в виде двух связанных между с собой операций. При удалении одной
операции автоматически удаляется вторая.


\section{Постоянные операции}
\label{\detokenize{transactions:id6}}
Многие операции могут повторяться с некоторой периодичностью. Такие операции называются постоянными. Обычно постоянные
операции используют при планировании, однако такие операции могут быть также и фактическими.

\noindent\sphinxincludegraphics[width=0.250\linewidth]{{recurringtransactions-010-select-directories}.png}

\noindent\sphinxincludegraphics[width=0.250\linewidth]{{recurringtransactions-020-select-recurring-transactions}.png}

\noindent\sphinxincludegraphics[width=0.250\linewidth]{{recurringtransactions-030-select-transaction}.png}

\noindent\sphinxincludegraphics[width=0.250\linewidth]{{recurringtransactions-040-reccuring-transaction}.png}


\chapter{Совместное использование типов портфелей}
\label{\detokenize{shared-transactions:chapter-shared-transactions}}\label{\detokenize{shared-transactions:id1}}\label{\detokenize{shared-transactions::doc}}
Совместное использование типов портфелей предназначено в основном для предпринимателей
и решает проблему «агентских» платежей. Например, пользователь является
индивидуальным предпринимателем и имеет два типа портфеля, персональный и малый бизнес.
Часто бывает так, что часть расходов за малый бизнес оплачивается с личной
карты, которая принадлежит персональному типу портфеля. Т.е. персональный тип портфеля
выступает в роли агента для типа портфеля Малый бизнес, т.к. действует в его интересах.
Но по умолчанию такие расходы отражаются в персональном типе портфеля, что, конечно,
не удобно для формирования отчетности. Более того для таких расходов необходимо
использовать аналитики малого бизнеса, которые недоступны для личной карты.

Для решения этой проблемы в приложении предусмотрен режим совместного использования типов
портфелей.

Чтобы включить режим совместного использования отредактируйте тип портфеля.
Флажок \DUrole{bbproperty}{Совместное использование} предназначен для того, чтобы аналитики данного типа портфеля
были доступны для всех типов портфелей.
Флажок \DUrole{bbproperty}{Выделение операций при совместном использовании как агентские} предназначен для того, чтобы:
\begin{enumerate}
\def\theenumi{\arabic{enumi}}
\def\labelenumi{\theenumi .}
\makeatletter\def\p@enumii{\p@enumi \theenumi .}\makeatother
\item {} 
В списке операций выделялись как агентские операции те операции, у которых не совпадает тип портфеля счета и статьи;

\item {} 
В отчетах тип портфеля определялся не по типу портфеля счета, а по типу портфеля статьи.

\end{enumerate}


\section{Пример совместного использования типов портфелей}
\label{\detokenize{shared-transactions:id2}}
Включим режим совместного использования для типа портфеля \DUrole{bbitem}{Малый бизнес}.

\noindent\sphinxincludegraphics[width=0.250\linewidth]{{transactionsshared-010-select-directories}.png}

\noindent\sphinxincludegraphics[width=0.250\linewidth]{{transactionsshared-020-menu-directories}.png}

\noindent\sphinxincludegraphics[width=0.250\linewidth]{{transactionsshared-025-types-of-portfolio}.png}

\noindent\sphinxincludegraphics[width=0.250\linewidth]{{transactionsshared-030-types-of-portfolio-business}.png}

Занесем новую операцию выплаты зарплаты работникам со счета \DUrole{bbitem}{Наличные}, который принадлежит типу портфеля \DUrole{bbitem}{Персональный},
в интересах типа портфеля \DUrole{bbitem}{Малый бизнес}.

\noindent\sphinxincludegraphics[width=0.250\linewidth]{{transactionsshared-040-create-transaction}.png}

\noindent\sphinxincludegraphics[width=0.250\linewidth]{{transactionsshared-050-transaction-edit}.png}

\noindent\sphinxincludegraphics[width=0.250\linewidth]{{transactionsshared-050-transaction-select-salary-start}.png}

Видно, что теперь доступна статья \DUrole{bbitem}{Выплата зарплаты}, а после ее выбора ниже статьи отображается
расшифровка, какому типу портфеля принадлежит аналитика.

\noindent\sphinxincludegraphics[width=0.250\linewidth]{{transactionsshared-055-transaction-select-salary}.png}

\noindent\sphinxincludegraphics[width=0.250\linewidth]{{transactionsshared-057-transaction-select-salary-end}.png}

Теперь проверим список операций, сводку и отчеты. В списке операций новая операция выделена другим цветом,
кроме того отдельно рассчитаны итоги. Точно также в сводке итоги рассчитаны отдельно.

\noindent\sphinxincludegraphics[width=0.250\linewidth]{{transactionsshared-060-transactions}.png}

\noindent\sphinxincludegraphics[width=0.250\linewidth]{{transactionsshared-070-summary-personal}.png}

\noindent\sphinxincludegraphics[width=0.250\linewidth]{{transactionsshared-080-summary-small-business}.png}

В отчетах мы видим, что такая операция попала в тип портфеля \DUrole{bbitem}{Малый бизнес},
благодаря чему легче рассчитать баланс.

\noindent\sphinxincludegraphics[width=0.250\linewidth]{{transactionsshared-090-turnovers}.png}

\begin{sphinxadmonition}{note}{Примечание:}
Чтобы не дублировались расходы, перечисление денежных средств на сумму агентских операций из типа портфеля \DUrole{bbitem}{Малый бизнес} в тип портфеля \DUrole{bbitem}{Персональный} необходимо отражать переводом
\end{sphinxadmonition}


\chapter{Настройка счетов для импорта SMS и push-уведомлений}
\label{\detokenize{account-identities:sms-push}}\label{\detokenize{account-identities:chapter-account-identities}}\label{\detokenize{account-identities::doc}}

\section{Выбор идентификатора}
\label{\detokenize{account-identities:id1}}
Перед импортом SMS и Push-уведомлений в карточке счета необходимо указать идентификатор. Это нужно для того,
чтобы приложение смогло определить, какому счету принадлежит импортируемая операция. Обычно банки
указывают  в сообщениях последние четыре цифры карты. Именно их лучше всего указать в качестве идентификатора счета.

Например, в сообщении вида

\begin{sphinxVerbatim}[commandchars=\\\{\}]
\PYG{n}{VISA1234}\PYG{p}{:} \PYG{l+m+mf}{08.08}\PYG{o}{.}\PYG{l+m+mi}{13} \PYG{l+m+mi}{14}\PYG{p}{:}\PYG{l+m+mi}{05} \PYG{n}{oplata} \PYG{n}{uslug} \PYG{l+m+mf}{5000.00} \PYG{n}{rub}\PYG{o}{.} \PYG{n}{dostupno} \PYG{l+m+mf}{1000.00} \PYG{n}{rub}\PYG{o}{.}
\end{sphinxVerbatim}

в качестве идентификатора следует выбрать VISA1234. Бывает так, что в сообщении банка номер карты или счета не указан.
Например в сообщении вида

\begin{sphinxVerbatim}[commandchars=\\\{\}]
\PYG{n}{Операция} \PYG{o}{\PYGZgt{}\PYGZgt{}} \PYG{o}{\PYGZhy{}}\PYG{l+m+mi}{6000} \PYG{n}{руб}\PYG{o}{.} \PYG{n}{Atm}\PYG{o}{\PYGZhy{}}\PYG{n}{msk}\PYG{o}{\PYGZhy{}}\PYG{l+m+mi}{001}
\end{sphinxVerbatim}

невозможно выбрать идентификатор. В этом случае следует указать отправителя сообщения. Для SMS это будет номер или
имя. Например, Сбербанк отправляет все сообщения с номера 900. Для Push-уведомлений отправителем является идентификатор пакета.
Например, для РокетБанка это ru.rocketbank.r2d2.

Чтобы задать идентификатор, откройте карточку счета. Перейдите к полю \DUrole{bbspinner}{Идентификатор счета или карты} счета или карты и выберите его из любого
сообщения банка. Если Вы хотите указать отправителя, то отредактируйте идентификатор вручную.
Также укажите настройку для Вашего банка.

\noindent\sphinxincludegraphics[width=0.250\linewidth]{{accountidenties-005-select-references}.png}

\noindent\sphinxincludegraphics[width=0.250\linewidth]{{accountidenties-010-select-accounts}.png}

\noindent\sphinxincludegraphics[width=0.250\linewidth]{{accountidenties-020-open-card-account}.png}

\noindent\sphinxincludegraphics[width=0.250\linewidth]{{accountidenties-030-scroll-to-identity}.png}

\noindent\sphinxincludegraphics[width=0.250\linewidth]{{accountidenties-035-select-identity}.png}

\noindent\sphinxincludegraphics[width=0.250\linewidth]{{accountidenties-040-set-identity}.png}


\section{Выбор ключевой фразы для перевода}
\label{\detokenize{account-identities:id2}}
Приложение может автоматически создавать переводы на основании сообщения банка. Например, при получении сообщения вида

\begin{sphinxVerbatim}[commandchars=\\\{\}]
\PYG{n}{VISA1234}\PYG{p}{:} \PYG{l+m+mf}{08.08}\PYG{o}{.}\PYG{l+m+mi}{13} \PYG{l+m+mi}{14}\PYG{p}{:}\PYG{l+m+mi}{05} \PYG{n}{выдача} \PYG{n}{наличных} \PYG{l+m+mf}{2000.00}\PYG{n}{р}\PYG{o}{.} \PYG{n}{ATM} \PYG{l+m+mi}{10010001} \PYG{n}{Баланс} \PYG{l+m+mf}{500.00} \PYG{n}{rub}\PYG{o}{.}
\end{sphinxVerbatim}

приложение может создать не только списание на 2000.00 руб. со счета VISA1234, но и поступление на счет Наличные. Для этого в карточке
счета Наличные следует задать ключевые фразы, по которым приложение будет идентифицировать этот счет. Для приведенного примера
это может быть \sphinxtitleref{выдача наличных} или \sphinxtitleref{ATM}.

\begin{sphinxadmonition}{note}{Примечание:}
Для автоматического создания переводов необходимо также, чтобы приложение смогло правильно идентифицировать операцию как перевод, см. {\hyperref[\detokenize{notifications:chapter-notifications}]{\sphinxcrossref{\DUrole{std,std-ref}{Расширенная настройка импорта SMS и push-уведомлений}}}} (\autopageref*{\detokenize{notifications:chapter-notifications}}).
\end{sphinxadmonition}

Чтобы задать ключевую фразу, откройте карточку счета. Перейдите к полю Ключевые слова и выберите его их сообщения банка. Также, при необходимости, можно отредактировать фразы вручную.

\noindent\sphinxincludegraphics[width=0.250\linewidth]{{accountidenties-050-open-cash-account}.png}

\noindent\sphinxincludegraphics[width=0.250\linewidth]{{accountidenties-060-scroll-to-keywords}.png}

\noindent\sphinxincludegraphics[width=0.250\linewidth]{{accountidenties-070-set-keywords}.png}

Обычно для счетов, по которым приходят уведомления, поле \DUrole{bbproperty}{Ключевые слова} остается пустым и наоборот, в наличных счетах
остается пустым поле \DUrole{bbproperty}{Номер}. Однако есть случаи, когда для счетов используются оба поля. В качестве примера можно привести настройку
импорта сообщений \sphinxhref{http://qa.bbmoney.biz/ru/index.php?qa=67\&qa\_1=\%D0\%BA\%D0\%B0\%D0\%BA-\%D0\%BD\%D0\%B0\%D1\%81\%D1\%82\%D1\%80\%D0\%BE\%D0\%B8\%D1\%82\%D1\%8C-\%D0\%B8\%D0\%BC\%D0\%BF\%D0\%BE\%D1\%80\%D1\%82-\%D1\%83\%D0\%B2\%D0\%B5\%D0\%B4\%D0\%BE\%D0\%BC\%D0\%BB\%D0\%B5\%D0\%BD\%D0\%B8\%D0\%B9-\%D1\%80\%D0\%BE\%D0\%BA\%D0\%B5\%D1\%82\%D0\%B1\%D0\%B0\%D0\%BD\%D0\%BA\%D0\%B0\&show=68\#a68}{РокетБанка}.


\chapter{Расширенная настройка импорта SMS и push-уведомлений}
\label{\detokenize{notifications:sms-push}}\label{\detokenize{notifications:chapter-notifications}}\label{\detokenize{notifications::doc}}

\section{Алгоритм распознавания уведомлений}
\label{\detokenize{notifications:id1}}
Основную роль при импорте SMS и push-уведомлений играет настройка импорта. Именно от нее зависит
как приложение распознает операцию, будет ли операция доходной, расходной или переводом, нужно ли
рассчитать баланс и курс операции и т.д.

Алгоритм распознавания операции показан на рис ниже.

\noindent{\hspace*{\fill}\sphinxincludegraphics[width=0.750\linewidth]{{sms-import-algorithm-ru}.png}\hspace*{\fill}}

При поступлении нового уведомление приложение пытается определить счет исходя из идентификаторов,
которые указаны в справочнике \DUrole{bbmeta}{Счета}. Если счет найден и он является единственным, то приложение
загружает связанную со счетом настройку импорта.

Далее, на основании настройки выполняется классификация типа операции — доходная, расходная или перевод.
Для переводов приложение пытается подобрать счет-получатель исходя из ключевых фраз,
которые указаны в справочнике счетов. Если корреспондирующий счет найден и он является единственным, то
приложение создаст вторую операцию и перевод будет завершенным.

Следующий этап — определение аналитик. По соответствующим ключевым фразам приложение пытается подобрать
контрагента, статью, проект и персону. Если какую-либо из
аналитик не удалось подобрать, то используются значения по умолчанию.

Наконец, приложение вычисляет сумму и баланс после операции. Если баланс по данным приложения не совпадает
с указанным в тексте уведомления, то могут быть созданы дополнительные операции комиссии, корректировки баланса
или же будет рассчитан курс операции. Это зависит от контекста и валюты операции.

Иногда бывает так, что уведомления приходят не в том порядке, как были совершены операции. В этом случае приложение
будет создавать автоматические корректировки баланса до тех пор, пока порядок не восстановится. После восстановления
правильного порядка приложение по возможности удалит лишние корректировки.

Например:
\begin{enumerate}
\def\theenumi{\arabic{enumi}}
\def\labelenumi{\theenumi .}
\makeatletter\def\p@enumii{\p@enumi \theenumi .}\makeatother
\item {} 
13.04.2016, 10:00, остаток = 1000 руб.

\end{enumerate}

Пришли SMS в неправильном порядке (правильный порядок: 4, 3, 5, 2)
\begin{enumerate}
\def\theenumi{\arabic{enumi}}
\def\labelenumi{\theenumi .}
\makeatletter\def\p@enumii{\p@enumi \theenumi .}\makeatother
\setcounter{enumi}{1}
\item {} 
13.04.2016, 15:00, списание = -50 руб., баланс = 500 руб., \(\rightarrow\) автокорректировка = -450 руб.

\item {} 
13.04.2016, 15:05, списание = -90 руб., баланс = 800 руб., \(\rightarrow\) автокорректировка = +390 руб.

\item {} 
13.04.2016, 15:10, списание = -110 руб., баланс = 890 руб., \(\rightarrow\) автокорректировка = +200 руб.

\item {} 
13.04.2016, 15:15, списание = -250 руб., баланс = 550 руб., \(\rightarrow\) автокорректировка = -90 руб.

\end{enumerate}

Поступила SMS в правильном порядке
\begin{enumerate}
\def\theenumi{\arabic{enumi}}
\def\labelenumi{\theenumi .}
\makeatletter\def\p@enumii{\p@enumi \theenumi .}\makeatother
\setcounter{enumi}{5}
\item {} 
13.04.2016, 15:20, списание = -100 руб., баланс = 400 руб., \(\rightarrow\) автокорректировка = 0 руб., автокорректировки 2 - 5 удалены

\end{enumerate}


\section{Создание новой настройки импорта}
\label{\detokenize{notifications:id2}}
На момент написания руководства приложение имеет более 160 готовых настроек импорта для банков различных стран.
Конечно, это не очень много, однако Вы с легкостью можете добавить настройку импорта для своего банка.
Поверьте, это совсем не сложно.

\noindent\sphinxincludegraphics[width=0.250\linewidth]{{notificationsimporttunes-010-select-directories}.png}

\noindent\sphinxincludegraphics[width=0.250\linewidth]{{notificationsimporttunes-020-select-import-tunes}.png}

\noindent\sphinxincludegraphics[width=0.250\linewidth]{{notificationsimporttunes-030-select-new}.png}

\DUrole{bbproperty}{Наименование} новой настройки может быть любым. Конечно, лучше чтобы название совпадало с
названием банка или платежной системы.

\DUrole{bbproperty}{Ограничение по отправителю} (номерами или именами отправителей SMS, идентификаторами пакетов push-уведомлений)
используется в редких случаях, когда приложение не может корректно определить счет. Оно отрабатывает раньше
подбора счета, ограничивая выбор счета только среди счетов с подходящей настройкой импорта.
\begin{description}
\item[{Например, пусть в приложении занесено два счета}] \leavevmode\begin{enumerate}
\def\theenumi{\arabic{enumi}}
\def\labelenumi{\theenumi .}
\makeatletter\def\p@enumii{\p@enumi \theenumi .}\makeatother
\item {} 
РокетБанк, идентификатор ru.rocketbank.r2d2, настройка импорта РокетБанк;

\item {} 
ВТБ, идентификатор ***1234, настройка импорта ВТБ.

\end{enumerate}

\end{description}

РокетБанк, отправитель ru.rocketbank.r2d2, присылает уведомления о зачислении средств в виде

\begin{sphinxVerbatim}[commandchars=\\\{\}]
Операция \PYGZgt{}\PYGZgt{} +18 000 руб.
Пополнение с карты «ВТБ\PYGZhy{}24 ***1234»
\end{sphinxVerbatim}

В этом уведомлении нет идентификатора счета, зато указан счет-источник перевода. Если ограничение не указано,
то приложение не может корректно выбрать счет, т.к. подходят оба счета.

Если задано ограничение ru.rocketbank.r2d2, то приложение по совпадению отправителя и заданного ограничения
находит настройку импорта РокетБанк. Эта настройка указана только в одном счете, поэтому приложение правильно выбирает
счет РокетБанк.

Основные параметры импорта задаются ключевыми фразами. Каждый параметр может содержать несколько ключевых фраз.
Ключевая фраза может содержать пробелы, между собой ключевые фразы должны быть разделены запятыми.

\DUrole{bbproperty}{Ключевые фразы для доходов и расходов} определяют знак операции. Если знак операции не определен, то импорт такой операции невозможен.

\DUrole{bbproperty}{Ключевые фразы для перевода} сигнализируют приложению о том, что нужно создать не одну, а две операции. Направление
перевода зависит от знака операции.
\begin{description}
\item[{Например, пусть:}] \leavevmode\begin{enumerate}
\def\theenumi{\arabic{enumi}}
\def\labelenumi{\theenumi .}
\makeatletter\def\p@enumii{\p@enumi \theenumi .}\makeatother
\item {} 
Ключевые фразы для доходов: «пополнение наличными,кредит,поступление»

\item {} 
Ключевые фразы для переводов: «пополнение наличными»

\item {} 
Идентификатор счета Карта: Visa2900

\item {} 
Ключевые фразы счета Наличные: ATM

\end{enumerate}

\end{description}

От банка поступило SMS:

\begin{sphinxVerbatim}[commandchars=\\\{\}]
\PYG{n}{Karta} \PYG{n}{Visa2900}\PYG{o}{.} \PYG{n}{Пополнение} \PYG{n}{наличными} \PYG{l+m+mf}{2000.00} \PYG{n}{RUR} \PYG{n}{ATM} \PYG{o}{.}\PYG{n}{Ostatok}\PYG{p}{:}\PYG{l+m+mf}{2740.26} \PYG{n}{RUR}\PYG{o}{.} \PYG{l+m+mi}{25}\PYG{o}{/}\PYG{l+m+mi}{03}\PYG{o}{/}\PYG{l+m+mi}{14}\PYG{p}{,}\PYG{l+m+mi}{15}\PYG{p}{:}\PYG{l+m+mi}{00}\PYG{p}{:}\PYG{l+m+mf}{00.}
\end{sphinxVerbatim}
\begin{description}
\item[{В результате программа создаст две операции:}] \leavevmode\begin{enumerate}
\def\theenumi{\arabic{enumi}}
\def\labelenumi{\theenumi .}
\makeatletter\def\p@enumii{\p@enumi \theenumi .}\makeatother
\item {} 
Операцию зачисления на счет Карта;

\item {} 
Операцию списания со счета Наличные.

\end{enumerate}

\end{description}

\noindent\sphinxincludegraphics[width=0.250\linewidth]{{notificationsimporttunes-040-import-tune}.png}

\noindent\sphinxincludegraphics[width=0.250\linewidth]{{notificationsimporttunes-050-import-tune-2}.png}

Иногда бывает так, что некоторые уведомления содержат баланс, некоторые — нет. Соответствующие ключевые фразы
подскажут приложению, когда нужно определять баланс, а когда — нет.

Банковские сообщения о том, что не может быть выполнена та или иная операция носят информационный характер,
однако содержат ключевые фразы для доходов или расходов. Ключевые фразы в параметре \DUrole{bbproperty}{Пропустить операцию} позволяют
прервать обработку импорта уведомления.
\begin{description}
\item[{Например:}] \leavevmode\begin{enumerate}
\def\theenumi{\arabic{enumi}}
\def\labelenumi{\theenumi .}
\makeatletter\def\p@enumii{\p@enumi \theenumi .}\makeatother
\item {} 
Ключевые фразы для зачисления: «пополнение наличными,кредит,поступление»

\item {} 
Ключевые фразы для неудачной операции: «ошибка»

\item {} 
Идентификатор счета Карта: Visa2900

\end{enumerate}

\end{description}

От банка поступило SMS:

\begin{sphinxVerbatim}[commandchars=\\\{\}]
\PYG{n}{Karta} \PYG{n}{Visa2900}\PYG{o}{.} \PYG{n}{Пополнение} \PYG{n}{наличными} \PYG{l+m+mf}{2000.00} \PYG{n}{RUR} \PYG{n}{ATM} \PYG{o}{.}\PYG{n}{Ostatok}\PYG{p}{:}\PYG{l+m+mf}{740.26} \PYG{n}{RUR}\PYG{o}{.} \PYG{n}{Произошла} \PYG{n}{ошибка}\PYG{o}{.} \PYG{l+m+mi}{25}\PYG{o}{/}\PYG{l+m+mi}{03}\PYG{o}{/}\PYG{l+m+mi}{14}\PYG{p}{,}\PYG{l+m+mi}{15}\PYG{p}{:}\PYG{l+m+mi}{00}\PYG{p}{:}\PYG{l+m+mf}{00.}
\end{sphinxVerbatim}

В результате программа не создаст операцию зачисления средств на счет Карта. И это будет соответствовать действительности,
т.к. по какой-то причине банкомат вернул деньги, вместо зачисления на счет.

\DUrole{bbproperty}{Позиция суммы операции среди числовых значений} указывает программе на наиболее вероятное
расположение суммы. В процессе разбора уведомления приложение примет окончательное решение.

\DUrole{bbproperty}{Позиция остатка операции среди числовых значений} указывает приложению на наиболее вероятное
расположение баланса. Также как и в случае с суммой операции, в процессе разбора уведомления
приложение самостоятельно примет окончательное решение.

Если все уведомления банка не содержат информацию о балансе, то следует указать «-1».

Если задана позиция баланса, отличная от «-1», то приложение будет игнорировать все сообщения, в которых нет баланса.
Тем не менее, при помощи ключевых фраз Вы можете уточнить, в каких случаях надо искать баланс, а когда — нет.

Сумма операции и значение баланса используются для расчета курса операции, комиссий и автоматических корректировок.

Для правильной работы программы необходимо, чтобы рядом с суммой была указана валюта. Валюта может быть указана как
слева от суммы так и справа. Для подбора валюты используются название и ключевые слова, указанные для каждой валюты
в справочнике Валюты.

Однако некоторые банки не всегда указывают валюту, например, Росбанк. В этом случае отметьте флажок
\DUrole{bbproperty}{Иногда валюта платежа может быть не указана}. В этом случае программа будет использовать валюту счета.


\chapter{Импорт данных}
\label{\detokenize{import:chapter-import}}\label{\detokenize{import:id1}}\label{\detokenize{import::doc}}

\section{Настройки импорта уведомлений}
\label{\detokenize{import:id2}}
Настройки импорта уведомлений играют важную роль в процессе импорта уведомлений. Если банк меняет
структуру уведомлений, то вместе со структурой меняются и настройки импорта. В этом случае Вы можете
загрузить обновление настроек или отредактировать настройки самостоятельно, см. главу {\hyperref[\detokenize{notifications:chapter-notifications}]{\sphinxcrossref{\DUrole{std,std-ref}{Расширенная настройка импорта SMS и push-уведомлений}}}} (\autopageref*{\detokenize{notifications:chapter-notifications}}).

\noindent\sphinxincludegraphics[width=0.250\linewidth]{{updateimporttunes-010-select-actions}.png}

\noindent\sphinxincludegraphics[width=0.250\linewidth]{{updateimporttunes-020-select-import}.png}

\noindent\sphinxincludegraphics[width=0.250\linewidth]{{updateimporttunes-030-select-import-sms-tunes}.png}

Для загрузки обновлений выберите пункт меню \sphinxmenuselection{Действия \(\rightarrow\) Импорт \(\rightarrow\) Настройки импорта SMS и Push}.

\noindent\sphinxincludegraphics[width=0.250\linewidth]{{updateimporttunes-040-select-import-tunes-updated}.png}

\noindent\sphinxincludegraphics[width=0.250\linewidth]{{updateimporttunes-050-select-import-tunes-new}.png}

\noindent\sphinxincludegraphics[width=0.250\linewidth]{{updateimporttunes-060-select-import-tunes-no-updates}.png}

Приложение покажет доступные обновления. Здесь же можно загрузить настройки для новых банков.

\noindent\sphinxincludegraphics[width=0.250\linewidth]{{updateimporttunes-070-select-actions}.png}

\noindent\sphinxincludegraphics[width=0.250\linewidth]{{updateimporttunes-080-select-active_profile}.png}

\noindent\sphinxincludegraphics[width=0.250\linewidth]{{updateimporttunes-085-select-notifications}.png}

\noindent\sphinxincludegraphics[width=0.250\linewidth]{{updateimporttunes-090-check-use_exchange_when_wifi}.png}

Возможно, что приложение не покажет доступные настройки импорта уведомлений. В этом случае проверьте, что
в настройках включен обмен настройками импорта SMS.


\section{SMS и Push уведомления}
\label{\detokenize{import:sms-push}}
По умолчанию Блиц Бюджет для Android автоматически импортирует SMS и Push уведомления. Тем не менее, в приложении есть возможность
в любой момент импортировать вручную SMS и push-уведомления. Для этого:
\begin{enumerate}
\def\theenumi{\arabic{enumi}}
\def\labelenumi{\theenumi .}
\makeatletter\def\p@enumii{\p@enumi \theenumi .}\makeatother
\item {} 
Откройте диалог импорта.

\item {} 
Выберите счет, для которого нужно импортировать SMS. В счете должны быть указаны идентификатор и настройка импорта SMS.

\item {} 
Отметьте галочками SMS для импорта.

\item {} 
Нажмите \DUrole{bbbutton}{Импорт}. Кнопка будет доступна, если есть отмеченные SMS.

\item {} 
Проверьте результат в списке операций.

\item {} 
Проблемы, возникшие при импорте, можно увидеть в журнале событий.

\end{enumerate}

\noindent\sphinxincludegraphics[width=0.250\linewidth]{{manualsmsimport-010-select-actions}.png}

\noindent\sphinxincludegraphics[width=0.250\linewidth]{{manualsmsimport-020-select-import}.png}

\noindent\sphinxincludegraphics[width=0.250\linewidth]{{manualsmsimport-030-select-import-sms}.png}

\noindent\sphinxincludegraphics[width=0.250\linewidth]{{manualsmsimport-040-select-account}.png}

\noindent\sphinxincludegraphics[width=0.250\linewidth]{{manualsmsimport-050-move-next}.png}

\noindent\sphinxincludegraphics[width=0.250\linewidth]{{manualsmsimport-060-import-sms}.png}

\noindent\sphinxincludegraphics[width=0.250\linewidth]{{manualsmsimport-070-view-transactions}.png}

\noindent\sphinxincludegraphics[width=0.250\linewidth]{{manualsmsimport-080-view-events}.png}


\section{CSV файлы}
\label{\detokenize{import:id3}}
Во время импорта данных из файлов в формате \sphinxhref{https://ru.wikipedia.org/wiki/CSV}{CSV}, помимо операций, могут быть созданы новые
счета и элементы справочников статей, плательщиков / получателей, проектов и персон.

Программа автоматически определяет разделитель колонок, который может быть одним из символов «;», «,», «\textbar{}», «/», «».
Файл должен быть в кодировке UTF-8.

Первая строка файла должна содержать имена колонок на английском языке, регистр не имеет значения. Помимо этого, имена
колонок могут быть заданы в любой другой строке, тогда они будут иметь силу для последующих строк. Поддерживаются имена:


\begin{savenotes}\sphinxattablestart
\centering
\sphinxcapstartof{table}
\sphinxthecaptionisattop
\sphinxcaption{Формат CSV файла}\label{\detokenize{import:id5}}
\sphinxaftertopcaption
\begin{tabular}[t]{|\X{7}{42}|\X{5}{42}|\X{30}{42}|}
\hline
\sphinxstyletheadfamily 
Имена
&\sphinxstyletheadfamily 
Обязательный
&\sphinxstyletheadfamily 
Комментарий
\\
\hline
id
&
Нет
&
Идентификатор операции, если указан, то будет выполнен поиск существующей операции
\\
\hline
account
&
Да
&
Наименование или номер счета
\\
\hline
date
&
Нет
&
Дата в одном из форматов: «dd’d’MM’d’yyyy» (например, 01d01d2017), «yyyy’d’MM’d’dd» (например, 2017d01d01), «yyyyMMddHHmmss», «yyyyMMddHHmm», «yyyyMMdd», «yyyy-MM-dd HH:mm:ss», «yyyy-MM-dd HH:mm», «yyyy-MM-dd», «dd-MM-yyyy HH:mm:ss», «dd-MM-yyyy HH:mm», «dd-MM-yyyy», «dd.MM.yyyy HH:mm:ss», «dd.MM.yyyy HH:mm», «dd.MM.yyyy»
\\
\hline
time
&
Нет
&
Время в одном из форматов: «HH:mm:ss», «HH:mm», «HHmmss», «HHmm»
\\
\hline
amount
&
Да
&
Сумма операции (может содержать валюту и разделители групп разрядов), десятичный разделитель может быть точкой или запятой
\\
\hline
rate, exchange rate
&
Нет
&
Курс операции
\\
\hline
currency
&
Нет
&
Валюта операции, если не указано, то используется в валюта счета
\\
\hline
payer, payee, contractor
&
Нет
&
Наименование плательщика или получателя платежа
\\
\hline
category
&
Нет
&
Наименование категории, если не указано, то программа дополнительно выполняет поиск по ключевым словам текущей строки
\\
\hline
project
&
Нет
&
Наименование проекта, если не указано, то программа дополнительно выполняет поиск по ключевым словам текущей строки
\\
\hline
person, unit
&
Нет
&
Наименование персоны/подразделения, если не указано, то программа дополнительно выполняет поиск по ключевым словам текущей строки
\\
\hline
notes, note
&
Нет
&
Примечание
\\
\hline
planned, plan
&
Нет
&
Фактическая (0), или плановая (1) операция. Если колонка не задана, создается фактическая операция
\\
\hline
detail, split
&
Нет
&
Операция (0), или детализация операции (1). По умолчанию используется значение 0.
\\
\hline
\end{tabular}
\par
\sphinxattableend\end{savenotes}

Если строка не содержит обязательных колонок, или значение обязательной колонки не задано, то такая
строка будет пропущена.

Если строка содержит не все обязательные колонки, но при этом задано значение обязательной колонки amount, то такая
строка считается детализацией операции и приложение создает сплит.

Для импорта:
\begin{enumerate}
\def\theenumi{\arabic{enumi}}
\def\labelenumi{\theenumi .}
\makeatletter\def\p@enumii{\p@enumi \theenumi .}\makeatother
\item {} 
Откройте диалог импорта.

\item {} 
Выберите файл для импорта.

\item {} 
Нажмите \DUrole{bbbutton}{Далее} и отметьте галочками строки для импорта.

\item {} 
Нажмите \DUrole{bbbutton}{Импорт}. Кнопка будет доступна, если есть отмеченные строки.

\item {} 
Проверьте результат в списке операций.

\item {} 
Проблемы, возникшие при импорте, можно увидеть в журнале событий.

\end{enumerate}

\noindent\sphinxincludegraphics[width=0.250\linewidth]{{csvimport-030-select-import-csv}.png}

\noindent\sphinxincludegraphics[width=0.250\linewidth]{{csvimport-040-select-file-and-options}.png}


\section{OFX файлы}
\label{\detokenize{import:id4}}
Блиц Бюджет для Android поддерживает импорт \sphinxhref{https://en.wikipedia.org/wiki/Open\_Financial\_Exchange}{OFX} файлов. Поддерживается спецификация начиная с версии 2.1.1. Для импорта:
\begin{enumerate}
\def\theenumi{\arabic{enumi}}
\def\labelenumi{\theenumi .}
\makeatletter\def\p@enumii{\p@enumi \theenumi .}\makeatother
\item {} 
Откройте диалог импорта.

\item {} 
Выберите файл для импорта.

\item {} 
Нажмите \DUrole{bbbutton}{Далее} и отметьте галочками строки для импорта.

\item {} 
Нажмите \DUrole{bbbutton}{Импорт}. Кнопка будет доступна, если есть отмеченные строки.

\item {} 
Проверьте результат в списке операций.

\item {} 
Проблемы, возникшие при импорте, можно увидеть в журнале событий.

\end{enumerate}

\noindent\sphinxincludegraphics[width=0.250\linewidth]{{ofximport-030-select-import-ofx}.png}

\noindent\sphinxincludegraphics[width=0.250\linewidth]{{ofximport-040-select-file-and-options}.png}


\chapter{Коллективная работа}
\label{\detokenize{teamwork:chapter-teamwork}}\label{\detokenize{teamwork:id1}}\label{\detokenize{teamwork::doc}}

\section{Введение}
\label{\detokenize{teamwork:id2}}
Блиц Бюджет для Android позволяет вести совместный учет доходов и расходов. Вот несколько примеров:
\begin{enumerate}
\def\theenumi{\arabic{enumi}}
\def\labelenumi{\theenumi .}
\makeatletter\def\p@enumii{\p@enumi \theenumi .}\makeatother
\item {} 
Полная синхронизация между устройствами;

\item {} 
Совместный финансовый учет только по выбранным счетам, проектам, персонам, контрагентам или даже статьям;

\item {} 
Сбор данных на одном устройстве, в случае, когда, скажем, родители отслеживают расходы детей.

\end{enumerate}

Любое устройство, на котором установлена программа, может стать узлом обмена (см. {\hyperref[\detokenize{glossary:term-1}]{\sphinxtermref{\DUrole{xref,std,std-term}{узел обмена}}}}) и получать или передавать изменения. Каждый узел обмена может обмениваться информацией с другими узлами.

\begin{sphinxadmonition}{note}{Примечание:}
Версия Free может передавать, но не может принимать сообщения. Версия Pro не содержит ограничений.
\end{sphinxadmonition}

Программа имеет гибкие настройки, регулирующие процесс обмена. Так например, можно разрешить принимать только новые операции от одного узла, и запретить принимать измененные. Для каждого узла обмена действуют свои настройки.

В целях повышения безопасности все сообщения между узлами шифруются, для каждого узла можно задать свой пароль, который будет использоваться для шифрования / дешифрования передаваемой информации.

Для совместной работы не требуется учетная запись Dropbox или других сервисов.


\section{Начало работы}
\label{\detokenize{teamwork:id3}}

\subsection{Выбор исходных данных}
\label{\detokenize{teamwork:id4}}
Предположим, что Алиса и Боб хотят вести совместный финансовый учет. Предварительно, им нужно определиться какая ситуация для них ближе:
\begin{enumerate}
\def\theenumi{\arabic{enumi}}
\def\labelenumi{\theenumi .}
\makeatletter\def\p@enumii{\p@enumi \theenumi .}\makeatother
\item {} 
В начале работы у Алисы и Боба будут одинаковые данные.

\item {} 
Алиса и/или Боб уже давно ведут учет и не хотят объединять все данные, а планируют синхронизировать лишь отдельные счета.

\end{enumerate}

В первом случае Алиса или Боб (для определенности пусть это будет Алиса) делает резервную копию данных. Затем Алиса передает резервную
копию данных Бобу и тот восстанавливает ее у себя на устройстве. Теперь у Алисы и Боба идентичные базы данных. Для корректной работы обмена
необходимо, что идентификаторы баз данных различались, поэтому Боб выполняет дополнительную сервисную операцию, формирует новый идентификатор на своем устройстве.

\begin{sphinxadmonition}{note}{Примечание:}
После восстановления данных из резервной копии для нового узла обмена необходимо сформировать новый идентификатор.
\end{sphinxadmonition}

\noindent\sphinxincludegraphics[width=0.250\linewidth]{{exchangenewid-005-select-actions}.png}

\noindent\sphinxincludegraphics[width=0.250\linewidth]{{exchangenewid-010-select-exchange}.png}

\noindent\sphinxincludegraphics[width=0.250\linewidth]{{exchangenewid-020-select-exchange_nodes}.png}

\noindent\sphinxincludegraphics[width=0.250\linewidth]{{exchangenewid-030-select-actions}.png}

\noindent\sphinxincludegraphics[width=0.250\linewidth]{{exchangenewid-040-select-new-id}.png}

Теперь Алиса и Боб готовы к настройке обмена.

Во втором случае никаких предварительных действий совершать не нужно. Алиса и Боб сразу готовы к настройке обмена.


\subsection{Обмен идентификаторами}
\label{\detokenize{teamwork:id5}}
Для работы обмена Алисе и Бобу необходимо обменяться идентификаторами узлов обмена и сохранить их в справочнике узлов.
Для этого Алиса открывает справочник \DUrole{bbmeta}{Узлы обмена} используя меню \sphinxmenuselection{Действия \(\rightarrow\) Обмен данными \(\rightarrow\)  Узлы обмена}. В справочнике
узлов обмена Алиса выбирает пункт меню \sphinxmenuselection{Отправить идентификатор} и отправляет идентификатор своего узла
по электронной почте Бобу.

\noindent\sphinxincludegraphics[width=0.250\linewidth]{{exchangesendid-005-select-actions}.png}

\noindent\sphinxincludegraphics[width=0.250\linewidth]{{exchangesendid-010-select-exchange}.png}

\noindent\sphinxincludegraphics[width=0.250\linewidth]{{exchangesendid-020-select-exchange_nodes}.png}

\noindent\sphinxincludegraphics[width=0.250\linewidth]{{exchangesendid-030-select-actions}.png}

\noindent\sphinxincludegraphics[width=0.250\linewidth]{{exchangesendid-040-select-share-id}.png}

\noindent\sphinxincludegraphics[width=0.250\linewidth]{{exchangesendid-050-mail-id}.png}

Боб принимает сообщение, создает новый узел обмена, указывает название и копирует полученный идентификатор. После этого отправляет свой идентификатор Алисе.

\noindent\sphinxincludegraphics[width=0.250\linewidth]{{exchangenewnode-005-select-actions}.png}

\noindent\sphinxincludegraphics[width=0.250\linewidth]{{exchangenewnode-010-select-exchange}.png}

\noindent\sphinxincludegraphics[width=0.250\linewidth]{{exchangenewnode-020-select-exchange_nodes}.png}

\noindent\sphinxincludegraphics[width=0.250\linewidth]{{exchangenewnode-030-click-fab}.png}

\noindent\sphinxincludegraphics[width=0.250\linewidth]{{exchangenewnode-040-setup_node}.png}

Алиса, в свою очередь, принимает сообщение Боба и создает новый узел обмена с идентификатором, который указан в сообщении Боба.


\section{Включение обмена}
\label{\detokenize{teamwork:id6}}
После обмена идентификаторами Алиса и Боб включают в настройках синхронизацию данных между узлами обмена.

\noindent\sphinxincludegraphics[width=0.250\linewidth]{{exchangeenable-005-select-actions}.png}

\noindent\sphinxincludegraphics[width=0.250\linewidth]{{exchangeenable-020-click-on-actions-menu}.png}

\noindent\sphinxincludegraphics[width=0.250\linewidth]{{exchangeenable-025-select-exchange}.png}

\noindent\sphinxincludegraphics[width=0.250\linewidth]{{exchangeenable-030-check-use-exchange}.png}

Теперь все изменения, которые делает Алиса отправляются Бобу и наоборот. Приложение синхронизирует изменения
автоматически примерно один раз в пять минут при наличии Wi-Fi или мобильного интернета. Функция синхронизации
автоматически отключается, если нет интернет-соединения или во время сна устройства. Благодаря этому
экономится трафик и электроэнергия аккумулятора.

Вот точный алгоритм запуска обмена:
\begin{enumerate}
\def\theenumi{\arabic{enumi}}
\def\labelenumi{\theenumi .}
\makeatletter\def\p@enumii{\p@enumi \theenumi .}\makeatother
\item {} 
После запуска обмена программа проверяет включен экран телефона или нет.
\begin{enumerate}
\def\theenumii{\arabic{enumii}}
\def\labelenumii{\theenumii .}
\makeatletter\def\p@enumiii{\p@enumii \theenumii .}\makeatother
\item {} 
Если экран включен, то следующее время срабатывания - через 5 мин. от текущего.

\item {} 
Если экран выключен, то следующее время срабатывания - через 60 мин. от текущего.

\item {} 
Оба будильника не имеют права будить телефон.

\end{enumerate}

\item {} 
При открытии главного экрана программы выполняется проверка на следующее время срабатывания будильника.
\begin{enumerate}
\def\theenumii{\arabic{enumii}}
\def\labelenumii{\theenumii .}
\makeatletter\def\p@enumiii{\p@enumii \theenumii .}\makeatother
\item {} 
Если следующее время срабатывания находится в пределах 10 мин. от текущего, то ничего не происходит.

\item {} 
Если следующее время срабатывания находится в пределах более 10 мин. от текущего, то запускается обмен в фоновом режиме и далее программа определяет следующее время срабатывания по п. 1

\end{enumerate}

\item {} 
Если сети нет, то обмен отключается полностью до следующего появления сети.

\end{enumerate}

При необходимости всегда можно вызвать синхронизацию вручную.


\section{Как работает обмен данными}
\label{\detokenize{teamwork:id7}}
Приложение Блиц Бюджет для Android ведет журнал изменений справочников и операций. Каждому узлу обмена отправляются изменения,
которые произошли либо с момента получения предыдущего пакета изменений, либо с момента создания узла.
Важна последовательность обмена: Алиса отправляет сообщение Бобу, Боб - Алисе и т.д. Если узел обмена Алисы не
получит ответ Боба, то не будет отправлять следующий пакет изменений до тех пор, пока не придет ответ.

Синхронизация всех элементов справочников выполняется в несколько этапов:
\begin{enumerate}
\def\theenumi{\arabic{enumi}}
\def\labelenumi{\theenumi .}
\makeatletter\def\p@enumii{\p@enumi \theenumi .}\makeatother
\item {} 
синхронизация по уникальному идентификатору;

\item {} 
синхронизация по ключевым фразам или коду;

\item {} 
синхронизация по наименованию.

\end{enumerate}

Каждый следующий шаг синхронизации выполняется в случае, если предыдущий закончился неудачей. Если элемент
не удалось найти, то программа создает новый, используя при этом значения по умолчанию, указанные в настройке узла.

Синхронизация операций выполняется только по уникальному идентификатору.


\section{Расширенная настройка}
\label{\detokenize{teamwork:id8}}
Алиса и Боб могут ограничить объем передаваемой информации. Есть два варианта задания ограничений:
\begin{enumerate}
\def\theenumi{\arabic{enumi}}
\def\labelenumi{\theenumi .}
\makeatletter\def\p@enumii{\p@enumi \theenumi .}\makeatother
\item {} 
разрешенная область данных;

\item {} 
запрещенная область данных.

\end{enumerate}

Области задаются в справочнике \DUrole{bbmeta}{Области данных}. Можно указать любую комбинацию счетов, статей, плательщиков и получателей,
проектов и персон.

В случае, если один и тот же элемент справочника одновременно попадает в разрешенную и запрещенную область, то более
высокий приоритет имеет запрещенная область.

На основании областей данных формируется список операций, постоянных операций и справочников для передачи узлу обмена.

Алиса и Боб могут ограничить объем принимаемой информации. Можно полностью отказаться принимать новые, измененные или удаленные объекты.
Или можно конкретизировать какой тип объектов не принимать в случае создания, изменения или удаления.


\section{Настройка передачи данных}
\label{\detokenize{teamwork:id9}}
Для повышения безопасности передачи данных следует указать пароль, которым будут зашифрованы сообщения между узлами обмена. Пароль Алисы должен совпадать с паролем Боба.

Также Алисе и Бобу следует указать какой вид коммуникаций использовать для обмена сообщениями: Wi-Fi и/или мобильный интернет.


\section{Значения по умолчанию}
\label{\detokenize{teamwork:id10}}
Справочники на устройствах Алисы и Боба могут не совпадать между собой. Например, Боб уже давно
ведет финансовый учет, а Алиса только что установила приложение. Боб может создать операцию и указать в ней, например,
проект, которого нет в узле обмена Алисы. При поступлении сообщения Боба, приложение на устройстве Алисы
создаст операцию, однако не сможет найти указанный Бобом проект. В этом случае приложение будет использовать
значение проекта по умолчанию, которое Алиса задала для узла обмена Боба.


\section{Перенос данных на новый телефон с сохранением настроек обмена}
\label{\detokenize{teamwork:id11}}
В случае работающего обмена, переход на новый телефон следует выполнять в порядке:
\begin{enumerate}
\def\theenumi{\arabic{enumi}}
\def\labelenumi{\theenumi .}
\makeatletter\def\p@enumii{\p@enumi \theenumi .}\makeatother
\item {} 
На старом телефоне в настройках выключить синхронизацию.

\item {} 
Сделать резервную копию.

\item {} 
Восстановить резервную копию на новом телефоне.

\item {} 
Включить синхронизацию на новом телефоне.

\end{enumerate}


\chapter{Отчеты}
\label{\detokenize{reports:chapter-reports}}\label{\detokenize{reports:id1}}\label{\detokenize{reports::doc}}
Любой отчет можно открыть из панели быстрых кнопок. Каждый отчет имеет возможность фильтрации,
группировки исходных данных и сохранения выбранных параметров. Функции
управления отчетом располагаются в нижней части экрана. При открытии отчета из списка или
другого отчета наследуются установленные фильтры.

\noindent\sphinxincludegraphics[width=0.250\linewidth]{{reports-010-select-reports}.png}

\noindent\sphinxincludegraphics[width=0.250\linewidth]{{reports-020-menu-reports}.png}

\noindent\sphinxincludegraphics[width=0.250\linewidth]{{reports-025-report-bottom-sheet}.png}

\noindent\sphinxincludegraphics[width=0.250\linewidth]{{reports-026-report-bottom-sheet-open}.png}

\noindent\sphinxincludegraphics[width=0.250\linewidth]{{reports-027-report-select-group}.png}

Из отчета всегда можно открыть исходные операции, чтобы понять из чего состоит та или иная цифра.

Также для отчетов можно создавать ярлыки для быстрого доступа к отчетам с заранее подготовленными параметрами.
Ярлыки запускаются из окна оболочки Android.


\section{Расписание платежей}
\label{\detokenize{reports:id2}}
Отчет предназначен для планирования предстоящих операций. Расписание отображает не только плановые, но и фактические
операции, если они находятся в выбранном периоде. Например, можно увидеть не только сколько запланировано расходов
на текущую неделю, но и сколько расходов уже оплачено.

\noindent\sphinxincludegraphics[width=0.250\linewidth]{{reports-030-payments-schedule}.png}

\noindent\sphinxincludegraphics[width=0.250\linewidth]{{reports-040-plan-vs-fact}.png}

\noindent\sphinxincludegraphics[width=0.250\linewidth]{{reports-050-turnovers}.png}


\section{План-факт}
\label{\detokenize{reports:id3}}
Отчет служит для выявления отклонений между запланированными и фактическими движениями в заданном периоде. Так, например,
из отчета видно, что по статье \DUrole{bbitem}{Доходы от аренды} прошло внеплановое поступление денежных средств,
а по статье \DUrole{bbitem}{Зарплата}
запланированные денежные средства еще не поступили.

Отчет можно сформировать как в разрезе аналитик, так и в разрезе периодов.


\section{Обороты}
\label{\detokenize{reports:id4}}
Отчет служит для просмотра агрегированных движений в заданном периоде. Так, например,
из отчета видно, что по статье \DUrole{bbitem}{Кредиты (Я должен)} было поступление денежных средств, по статье \DUrole{bbitem}{Карманные расходы}
было списание и т.д.

Отчет можно формировать как по фактическим, так и по планируемым операциям. По умолчанию отчет формируется
по фактическим операциям.


\section{Остатки и обороты}
\label{\detokenize{reports:id5}}
Отчет служит для просмотра начальных, конечных остатков и агрегированных движений в заданном периоде. Отчет формируется
только по фактическим операциям.

\noindent\sphinxincludegraphics[width=0.250\linewidth]{{reports-060-totals-turnovers}.png}

\noindent\sphinxincludegraphics[width=0.250\linewidth]{{reports-070-planned-totals-turnovers}.png}

\noindent\sphinxincludegraphics[width=0.250\linewidth]{{reports-080-debts}.png}


\section{Планируемые остатки и обороты}
\label{\detokenize{reports:id6}}
Отчет служит для просмотра начальных, конечных остатков и агрегированных планируемых движений в заданном периоде. Отчет формируется
только по планируемым операциям.


\section{Долги}
\label{\detokenize{reports:id7}}
Отчет формируется по операциями, которые содержат статьи с признаком \DUrole{bbproperty}{Суммируемая} и при этом являются
одновременно и доходными и расходными. Суммы таких операций складываются, отчет показывает начальный, конечный
остатки и движения за выбранный период. Нулевые суммы скрываются.

Так, например, из отчета видно, что по статье \DUrole{bbitem}{Кредиты (Я должен)} на начало периода не было остатка. Затем
в течение периода было зачисление, т.е. был получен кредит. Погашения кредита не было, поэтому конечный
остаток совпадает с суммой зачисления.


\section{Исполнение плана}
\label{\detokenize{reports:id8}}
Отчет Исполнение плана формируется по плановым и фактическим операциями, которые содержат статьи с
признаком \DUrole{bbproperty}{Суммируемая} и являются либо доходными, либо расходными. Из суммы плановых операций вычитается
сумма фактических, отчет также показывает начальный, конечный остатки и движения за выбранный период.
Нулевые суммы скрываются.

Так, например, из отчета видно, что по статье \DUrole{bbitem}{Зарплата} на начало периода есть остаток плана, т.е. фактическая сумма
движений по статье \DUrole{bbitem}{Зарплата} меньше запланированной. По этой статье также в течение выбранного периода запланировано
поступление денежных средств. Однако фактического движения не было.

\noindent\sphinxincludegraphics[width=0.250\linewidth]{{reports-090-plan-implementation}.png}

\noindent\sphinxincludegraphics[width=0.250\linewidth]{{reports-100-donut}.png}

\noindent\sphinxincludegraphics[width=0.250\linewidth]{{reports-110-bars}.png}


\section{Распределение оборотов}
\label{\detokenize{reports:id9}}
Диаграмма служит для анализа распределения движений денежных средств. Отчет имеет два режима формирования —
по расходам и по доходам. Иногда не все названия помещаются на экране или накладываются друг на друга. Чтобы
увидеть такие значения, просто вращайте график против часовой стрелки.


\section{Изменение оборотов}
\label{\detokenize{reports:id10}}
График служит для анализа и выявления тенденций движений денежных средств. В положительной части графика отображаются доходы,
в отрицательной — расходы.


\section{Изменение остатков}
\label{\detokenize{reports:id11}}
График служит для анализа и выявления тенденций в изменениях остатков денежных средств. Возможно одновременное отображение
фактических и планируемых остатков.

\noindent\sphinxincludegraphics[width=0.250\linewidth]{{reports-120-lines}.png}


\chapter{Напоминания}
\label{\detokenize{reminders:chapter-reminders}}\label{\detokenize{reminders:id1}}\label{\detokenize{reminders::doc}}
Блиц Бюджет для Android позволяет создавать напоминания на основании отчетов или списка операций. Напоминание может быть разовым
или иметь заданную периодичность. Механизм напоминаний позволяет:
\begin{itemize}
\item {} 
Настроить напоминания о незаполненных операциях.

\item {} 
Настроить напоминания об особенных операциях.

\item {} 
Настроить предупреждения о расхождении плана и факта.

\item {} 
Настроить предупреждения о любом событии, которое можно выявить с помощью отчета.

\item {} 
Автоматически формировать отчеты по расписанию.

\end{itemize}

\begin{sphinxadmonition}{note}{Примечание:}
В версии Pro из напоминания можно сразу перейти в отчет
\end{sphinxadmonition}

Перед созданием напоминания необходимо сформулировать условия, при наступлении которых Вы хотите увидеть
уведомление о возникновении того или иного события. Для этого создайте и сохраните настройку
отчета, в которой укажите нужные значения фильтра и группировку отчета, см.
{\hyperref[\detokenize{shortcuts:chapter-shortcuts}]{\sphinxcrossref{\DUrole{std,std-ref}{Настройки отчетов и ярлыки}}}} (\autopageref*{\detokenize{shortcuts:chapter-shortcuts}}).

Как только настройка будет готова, создайте для нее напоминание кнопкой \DUrole{bbbutton}{Напоминания} или из справочника \DUrole{bbmeta}{Напоминания}.

\begin{sphinxadmonition}{note}{Примечание:}
Начиная с версии Android 4.4 точность срабатывания напоминаний составляет +/- 15 мин.
\end{sphinxadmonition}

Разберем создание напоминания на примере напоминания об операциях с незаполненной категорией. На главном экране
перейдем в список операций.

\noindent\sphinxincludegraphics[width=0.250\linewidth]{{reminders-010-main-screen}.png}

\noindent\sphinxincludegraphics[width=0.250\linewidth]{{reminders-020-main-screen-swipe-left}.png}

\noindent\sphinxincludegraphics[width=0.250\linewidth]{{reminders-030-main-screen-transactions}.png}

Отредактируем фильтр так, что бы в список попадали только операции с незаполненной категорией.

\noindent\sphinxincludegraphics[width=0.250\linewidth]{{reminders-040-transactions-bottom-sheet-opening}.png}

\noindent\sphinxincludegraphics[width=0.250\linewidth]{{reminders-050-transactions-bottom-sheet-open}.png}

\noindent\sphinxincludegraphics[width=0.250\linewidth]{{reminders-060-report-filter}.png}

\noindent\sphinxincludegraphics[width=0.250\linewidth]{{reminders-070-report-filter-category}.png}

\noindent\sphinxincludegraphics[width=0.250\linewidth]{{reminders-080-report-filter-apply}.png}

\noindent\sphinxincludegraphics[width=0.250\linewidth]{{reminders-090-report-filter-applied}.png}

После установки фильтра список операций изменился, теперь он содержит только две операции. Сохраним фильтр
в настройке отчета. Для этого откроем подвал и щелкнув по \DUrole{bbspinner}{Настройки отчета} создадим
новую настройку. Оставим период настройки без изменений, однако Вы можете отредактировать его нужным образом.

\noindent\sphinxincludegraphics[width=0.250\linewidth]{{reminders-100-report-select-new-setting}.png}

\noindent\sphinxincludegraphics[width=0.250\linewidth]{{reminders-100-report-setting-save}.png}

\noindent\sphinxincludegraphics[width=0.250\linewidth]{{reminders-110-report-view-settings-alarms}.png}

Теперь на основании новой настройки создадим напоминание. Для этого перейдем в список напоминаний и
откроем карточку нового напоминания. Зададим дату начала и время выполнения напоминания, периодичность и название. Название
отображается в уведомлении, которое будет сформировано по напоминанию.

\noindent\sphinxincludegraphics[width=0.250\linewidth]{{reminders-120-alarms-new}.png}

\noindent\sphinxincludegraphics[width=0.250\linewidth]{{reminders-130-alarms-edit}.png}

\noindent\sphinxincludegraphics[width=0.250\linewidth]{{reminders-140-alarms}.png}

Напоминание готово, осталось только проверить, как оно работает. Для этого отмечаем напоминание и выбираем
\DUrole{bbbutton}{Выполнить}. В строке состояния мы видим уведомление о наличии операций с незаполненной категорией.

\noindent\sphinxincludegraphics[width=0.250\linewidth]{{reminders-150-alarms-select}.png}

\noindent\sphinxincludegraphics[width=0.250\linewidth]{{reminders-160-alarms-run}.png}

\noindent\sphinxincludegraphics[width=0.250\linewidth]{{reminders-170-alarm_notification}.png}

Для просмотра списка операций достаточно щелкнуть по уведомлению.

\begin{sphinxadmonition}{note}{Примечание:}
Переход к данным уведомления доступен только в версии Pro
\end{sphinxadmonition}

Теперь при появлении операций с незаполненной категорией мы каждый день в назначенное время
будем видеть уведомление о необходимости заполнить категорию.


\chapter{Действия с группами объектов}
\label{\detokenize{bulk-actions:chapter-bulk-actions}}\label{\detokenize{bulk-actions:id1}}\label{\detokenize{bulk-actions::doc}}
Блиц Бюджет для Android позволяет выполнять действия сразу со множеством объектов. В качестве примера можно привести замену
статьи сразу для нескольких операций. Групповые действия можно выполнять не только с операциями, но и с
любыми справочниками.


\section{Выбор объектов}
\label{\detokenize{bulk-actions:id2}}
Разберем выбор нескольких объектов на примере списка операций. Точно такие же действия можно выполнить в
любом справочнике.

\noindent\sphinxincludegraphics[width=0.250\linewidth]{{bulkactions-010-transactions}.png}

\noindent\sphinxincludegraphics[width=0.250\linewidth]{{bulkactions-020-transaction-check}.png}

\noindent\sphinxincludegraphics[width=0.250\linewidth]{{bulkactions-030-transactions-checked}.png}

Первом делом следует открыть список операций. Затем отметим галочками нужные операции. Если необходимо выбрать все
операции, то достаточно отметить любую операцию из списка, а затем в панели действий выбрать \DUrole{bbbutton}{Отметить все}.


\section{Редактирование}
\label{\detokenize{bulk-actions:id3}}
Для редактирования выбранных операций следует нажать \DUrole{bbbutton}{Редактировать}. Приложение откроет диалог, в котором
указано количество выбранных элементов и поля, которые в этих элементах можно изменить. Так, для операций можно изменить
дату и время, примечаний, аналитики и пр. поля. Изменения применяются только для модифицированных в диалоге редактирования
полей.

\noindent\sphinxincludegraphics[width=0.250\linewidth]{{bulkactions-040-transactions-edit}.png}

\noindent\sphinxincludegraphics[width=0.250\linewidth]{{bulkactions-050-transactions-edit-dialog}.png}


\section{Удаление}
\label{\detokenize{bulk-actions:id4}}
Для удаления выбранных операций следует нажать \DUrole{bbbutton}{Удалить}. После подтверждения приложение удалит выбранные элементы.

\noindent\sphinxincludegraphics[width=0.250\linewidth]{{bulkactions-060-transactions-delete}.png}

\noindent\sphinxincludegraphics[width=0.250\linewidth]{{bulkactions-070-transactions-delete-dialog}.png}


\section{Фильтр}
\label{\detokenize{bulk-actions:id5}}
На основании нескольких элементов можно создать фильтр. Это удобно, когда например, необходимо увидеть все операции с
такими же как и отмеченных операций, статьями, контрагентами, проектами или персонами. В поля фильтра сразу будут скопированы
значения из выбранных операций, останется лишь отметить нужные поля отбора галочками.

Для создания фильтра на основании выбранных элементов следует нажать \DUrole{bbbutton}{Фильтр}.

\noindent\sphinxincludegraphics[width=0.250\linewidth]{{bulkactions-080-transactions-filter}.png}

\noindent\sphinxincludegraphics[width=0.250\linewidth]{{bulkactions-090-transactions-filter-dialog}.png}


\section{Повторная отправка объектов при обмене}
\label{\detokenize{bulk-actions:id6}}
Иногда в случае коллективной работы необходимо повторить отправку операции или элементов справочника. Для этого служит
пункт меню \sphinxmenuselection{Отправить при обмене}.

\noindent\sphinxincludegraphics[width=0.250\linewidth]{{bulkactions-100-transactions-more}.png}

\noindent\sphinxincludegraphics[width=0.250\linewidth]{{bulkactions-110-transactions-exchange-send}.png}

\noindent\sphinxincludegraphics[width=0.250\linewidth]{{bulkactions-120-transactions-exchange-send-done}.png}


\section{Экспорт в CSV и OFX}
\label{\detokenize{bulk-actions:csv-ofx}}
Выделенные операции можно экспортировать в файлы формата CSV и OFX, используя пункты меню \sphinxmenuselection{Экспорт CSV}
и \sphinxmenuselection{Экспорт OFX}. В отличие от операций, элементы справочников можно экспортировать только в файлы формата
CSV.

\begin{sphinxadmonition}{note}{Примечание:}
Экспорт операций в OFX доступен только в версии Pro.
\end{sphinxadmonition}

\noindent\sphinxincludegraphics[width=0.250\linewidth]{{bulkactions-130-transactions-export-csv}.png}

\noindent\sphinxincludegraphics[width=0.250\linewidth]{{bulkactions-150-transactions-export-ofx}.png}

\noindent\sphinxincludegraphics[width=0.250\linewidth]{{bulkactions-160-transactions-export-ofx-done}.png}


\section{Автоматическое связывание операций}
\label{\detokenize{bulk-actions:id7}}
Для точного учета переводов иногда требуется дополнительно связать операции. Например,
такая операция может потребоваться, если перевод занесен вручную или в результате импорта
в виде двух отдельных несвязанных операций. Для связывания операций отметьте хотя бы
одну операцию и приложение автоматически определит завершающую операцию в переводе.

\noindent\sphinxincludegraphics[width=0.250\linewidth]{{bulkactions-190-transactions-connect}.png}

\begin{sphinxadmonition}{note}{Примечание:}
Начиная с версии 6 при вводе перевода вручную обе операции автоматически связываются, поэтому нет
необходимости дополнительно связывать такие операции. Связанные операции помечаются специальным значком.
\end{sphinxadmonition}


\section{Отправка исходных данных разработчику}
\label{\detokenize{bulk-actions:id8}}
Иногда требуется помощь разработчика для выяснения причин того или иного поведения приложения. В этих
случаях для анализа обычно требуются исходные данные.

Отправить исходные данные можно через пункт меню \sphinxmenuselection{Отправить разработчику}. Перед отправкой
приложение откроет предварительный просмотр письма и Вы можете увидеть и отредактировать содержание
отправляемых данных. Таким образом можно избежать передачи конфиденциальной информации.

\noindent\sphinxincludegraphics[width=0.250\linewidth]{{bulkactions-170-transactions-developer-send}.png}

\noindent\sphinxincludegraphics[width=0.250\linewidth]{{bulkactions-180-transactions-developer-send}.png}


\chapter{Настройки отчетов и ярлыки}
\label{\detokenize{shortcuts:chapter-shortcuts}}\label{\detokenize{shortcuts:id1}}\label{\detokenize{shortcuts::doc}}

\section{Настройки отчетов}
\label{\detokenize{shortcuts:id2}}
Блиц Бюджет для Android позволяет сохранять значения группировок и фильтров для отчетов и списка операций. Разберем
сохранение настройки на примере отчета \DUrole{bbmeta}{Обороты}. Аналогичным образом сохраняются настройки
для других отчетов и списка операций.

Итак, после открытия в отчете по умолчанию установлены текущий месяц, группировки и значения фильтра.

\noindent\sphinxincludegraphics[width=0.250\linewidth]{{shortcuts-010-select-reports}.png}

\noindent\sphinxincludegraphics[width=0.250\linewidth]{{shortcuts-020-report-open}.png}

\noindent\sphinxincludegraphics[width=0.250\linewidth]{{shortcuts-030-report-bottom-sheet-opening}.png}

Наша цель — добиться того, чтобы был быстрый доступ к формированию отчета \DUrole{bbmeta}{Обороты} с отбором
сразу только по одному счету.

Отредактируем настройки фильтр. Для этого следует вытянуть подвал и нажать на \DUrole{bbspinner}{Фильтр}.
В редакторе фильтра зададим отбор только по одному счету и применим изменения.

\noindent\sphinxincludegraphics[width=0.250\linewidth]{{shortcuts-040-report-bottom-sheet-open}.png}

\noindent\sphinxincludegraphics[width=0.250\linewidth]{{shortcuts-050-report-filter}.png}

\noindent\sphinxincludegraphics[width=0.250\linewidth]{{shortcuts-060-report-filter-account}.png}

На рисунке видно, что теперь в отчете отображаются данные только одного счета. Теперь создадим и
сохраним настройку. Для этого в подвале следует нажать на \DUrole{bbspinner}{Настройки отчета} и в выпадающем списке
выбрать создание новой настройки.

\noindent\sphinxincludegraphics[width=0.250\linewidth]{{shortcuts-070-report-filter-apply}.png}

\noindent\sphinxincludegraphics[width=0.250\linewidth]{{shortcuts-080-report-filter-applied}.png}

\noindent\sphinxincludegraphics[width=0.250\linewidth]{{shortcuts-090-report-select-new-setting}.png}

Укажем название новой настройки \DUrole{bbvalue}{Обороты по одному счету} и сохраним ее. Теперь в списке настроек доступна
готовая настройка \DUrole{bbitem}{Обороты по одному счету}, при ее выборе в отчете сразу будут установлены нужные группировки
и значения фильтра.

\noindent\sphinxincludegraphics[width=0.250\linewidth]{{shortcuts-100-report-setting-save}.png}

\noindent\sphinxincludegraphics[width=0.250\linewidth]{{shortcuts-110-report-view-settings}.png}


\section{Создание ярлыка}
\label{\detokenize{shortcuts:id3}}
Блиц Бюджет для Android позволяет открывать отчеты и список операций прямо из окна оболочки Android. В предыдущей части мы рассмотрели
создание настройки. Предположим, что мы хотим не только создать настройку но и сделать для нее ярлык.

\begin{sphinxadmonition}{note}{Примечание:}
Создание ярлыков доступно в версии Pro.
\end{sphinxadmonition}

Вернемся к карточке настройки. Обратите внимание, что в карточке можно задать вид периодичности. От этого зависит,
какой период будет установлен при открытии отчета по ярлыку. По умолчанию установлен текущий месяц, но при
необходимости можно задать другой вид периодичности, например, текущий квартал или полугодие и т.п.

Для создания ярлыка следует нажать \DUrole{bbbutton}{Создать ярлык}.

\noindent\sphinxincludegraphics[width=0.250\linewidth]{{shortcuts-120-report-setting-shortcut-create}.png}

\noindent\sphinxincludegraphics[width=0.250\linewidth]{{shortcuts-130-report-shortcut-select}.png}

При создании ярлыка приложения автоматически размещает новый ярлык на свободном месте на одном из окон оболочки.

\begin{sphinxadmonition}{note}{Примечание:}
Ярлык связан с настройкой списка. Если удалить настройку, то ярлык перестанет работать.
\end{sphinxadmonition}

\noindent\sphinxincludegraphics[width=0.250\linewidth]{{shortcuts-140-report-open}.png}

\noindent\sphinxincludegraphics[width=0.250\linewidth]{{shortcuts-150-report-bottom-sheet-opening}.png}

\noindent\sphinxincludegraphics[width=0.250\linewidth]{{shortcuts-160-report-bottom-sheet-open}.png}

Проверим работу ярлыка. По нажатию открывается отчет, на рисунках видно, что приложение сразу применило заданные
настройки фильтра.

Ярлык — это только ссылка на настройку, поэтому если в дальнейшем требуется изменить параметры отчета,
то достаточно просто отредактировать сохраненную настройку.


\chapter{Виджеты и шаблоны}
\label{\detokenize{widgets:chapter-widgets}}\label{\detokenize{widgets:id1}}\label{\detokenize{widgets::doc}}

\section{Виджеты}
\label{\detokenize{widgets:id2}}
Блиц Бюджет для Android содержит удобный виджет для отображения фактических остатков, оборотов и быстрого создания новой операции.

\noindent{\hspace*{\fill}\sphinxincludegraphics[width=0.250\linewidth]{{widget-480}.png}\hspace*{\fill}}

Виджет доступен в различных размерах, 1x1, 1x2 и 1x4 ячейки. Оформление виджета совпадает с темой приложения.

Благодаря гибким настройкам виджет можно использовать не только как сводку, но и как краткий отчет или шаблон
новой операции. Примеры будут рассмотрены ниже.

\noindent\sphinxincludegraphics[width=0.250\linewidth]{{widgets-005-widget-available}.png}

\noindent\sphinxincludegraphics[width=0.250\linewidth]{{widgets-010-widget-settings-open}.png}

\noindent\sphinxincludegraphics[width=0.250\linewidth]{{widgets-020-widget-settings}.png}

После создания в виджете отображаются текущий остаток и движения денежных средств в основной валюте за текущий день.
Для изменения этих настроек служит кнопка \DUrole{bbbutton}{Настройка}.

В разделе \DUrole{bbsection}{Вид} можно задать основные параметры виджета.

\DUrole{bbproperty}{Наименование} удобно использовать, если на экран выведено несколько виджетов. При желании можно оставить это поле
пустым.

\DUrole{bbproperty}{Типы портфелей}, \DUrole{bbproperty}{Портфели} и \DUrole{bbproperty}{Счета} служат для базового ограничения отображаемой в
виджете информации. Можно указать один из параметров или их комбинацию. В разных экземплярах виджета могут использоваться
разные ограничения. Так, например, на экран можно вывести два виджета, один будет показывать информацию по одному счету,
другой — по другому.

\DUrole{bbproperty}{Отображать баланс}  служит для включения и отключения расчета баланса, также можно конкретизировать как именно
рассчитывать баланс, с учетом кредитного лимита или без. По умолчанию кредитный лимит не учитывается в балансе, т.е. для
кредитных карт отображается отрицательный остаток.

\noindent\sphinxincludegraphics[width=0.250\linewidth]{{widgets-030-widget-settings-2}.png}

\noindent\sphinxincludegraphics[width=0.250\linewidth]{{widgets-040-widget-settings-apply}.png}


\section{Использование виджетов в качестве шаблонов операций}
\label{\detokenize{widgets:id3}}
Виджет содержит кнопку \DUrole{bbbutton}{Новая операция}. Эта кнопка доступна после того, как в настройках будет указан счет для
создания новых операций. При желании можно указать сумму новой операции, которая будет автоматически установлена
при открытии карточки новой операции.

Также в новую операцию будут скопированы значения фильтров.

Таким образом, задав счет, сумму и фильтры, возможно использовать виджет для создания новой операции по шаблону. Все
поля новой операции будут сразу заполнены.

\begin{sphinxadmonition}{note}{Примечание:}
Использование виджетов для создания новых операций по шаблону доступно в версии Pro. В версии Free доступно создание новых операций без заполнения по шаблону.
\end{sphinxadmonition}


\section{Использование виджетов в качестве отчетов}
\label{\detokenize{widgets:id4}}
Гибкие настройки позволяют использовать виджеты в качестве отчетов с заранее сохраненными настройками. Для этого
служат параметры расположенные в разделе \DUrole{bbsection}{Фильтр}.

\begin{sphinxadmonition}{note}{Примечание:}
Использование виджетов в качестве отчетов доступно в версии Pro
\end{sphinxadmonition}


\section{Пример использования виджета в качестве отчета}
\label{\detokenize{widgets:id5}}
Рассмотрим в качестве примера настройку виджета для отображения расходов на общественный транспорт в течение текущего месяца.
Откроем настройки виджета и зададим название \DUrole{bbvalue}{Общественный транспорт}.

\noindent\sphinxincludegraphics[width=0.250\linewidth]{{widgets-050-widget-example-set-name}.png}

\noindent\sphinxincludegraphics[width=0.250\linewidth]{{widgets-060-widget-example-select-period}.png}

\noindent\sphinxincludegraphics[width=0.250\linewidth]{{widgets-070-widget-example-select-period-apply}.png}

Общее количество потраченных денежных средств на общественный транспорт с начала ведения учета нас не интересует,
поэтому отключим отображение баланса.

В качестве периода выберем текущий месяц.

\noindent\sphinxincludegraphics[width=0.250\linewidth]{{widgets-080-widget-example-select-budget-item}.png}

\noindent\sphinxincludegraphics[width=0.250\linewidth]{{widgets-090-widget-example-select-budget-item-apply}.png}

\noindent\sphinxincludegraphics[width=0.250\linewidth]{{widgets-100-widget-example-settings-apply}.png}

В настройках фильтра зададим статью \DUrole{bbitem}{Общественный транспорт} и сохраним настройку.

\noindent\sphinxincludegraphics[width=0.250\linewidth]{{widgets-110-widget-example}.png}

Теперь виджет отображает движения только по статье \DUrole{bbitem}{Общественный транспорт} за текущий месяц, мы видим
конкретную сумму расходов и два счета, с которых оплачивался транспорт.


\section{Пример использования виджета в качестве шаблона}
\label{\detokenize{widgets:id6}}
Теперь изменим настройки виджета так, чтобы мы могли не только видеть расходы, но и быстро их создавать. Для
этого снова откроем настройки.

\noindent\sphinxincludegraphics[width=0.250\linewidth]{{widgets-120-widget-example-select-account}.png}

\noindent\sphinxincludegraphics[width=0.250\linewidth]{{widgets-140-widget-example-select-amount}.png}

\noindent\sphinxincludegraphics[width=0.250\linewidth]{{widgets-150-widget-example-select-amount-value}.png}

Зададим счет, с которого чаще всего будет происходить оплата транспорта. После этого укажем сумму, которая будет
подставлена в новую операцию.

\noindent\sphinxincludegraphics[width=0.250\linewidth]{{widgets-160-widget-example-select-amount-value-2}.png}

\noindent\sphinxincludegraphics[width=0.250\linewidth]{{widgets-170-widget-example-settings-apply}.png}

\noindent\sphinxincludegraphics[width=0.250\linewidth]{{widgets-180-widget-example-transaction-new}.png}

Сохраним настройки. Теперь в виджете появилась кнопка добавления новой операции.

\noindent\sphinxincludegraphics[width=0.250\linewidth]{{widgets-190-widget-example-transaction}.png}

Добавляем новую операцию из виджета. Видно, что в новой операции автоматически заполнились счета, сумма и статья. Осталось
только сохранить новую операцию.

Аналогичным образом можно задать контрагентов, проекты и персоны, которые будут подставлены в новую операцию. Для каждого
шаблона следует создать свой виджет.


\chapter{Удаленный доступ}
\label{\detokenize{remote-access:chapter-remote-access}}\label{\detokenize{remote-access:id1}}\label{\detokenize{remote-access::doc}}
Блиц Бюджет для Android включает в себя клиента для персональных компьютеров. Клиент работает на операционных системах
Windows, Linux, Mac и пр. Все что нужно для работы — это современный браузер. Поддерживаются Internet
Explorer 8+, Google Chrome, Apple Safari, Mozilla Firefox, Opera.

\noindent\sphinxincludegraphics[width=0.250\linewidth]{{remoteaccess-010-remote-access-enable}.png}

\noindent\sphinxincludegraphics[width=0.250\linewidth]{{remoteaccess-020-view-status-bar}.png}

\noindent\sphinxincludegraphics[width=0.250\linewidth]{{remoteaccess-030-remote-access-disable}.png}

Активация клиента выполняется из главного экрана. После включения программа выводит сообщение с инструкцией
по доступу к данным с персонального компьютера. Одновременно появляется значок дисплей в строке всплывающих
сообщение, напоминающий о том что связь с ПК включена.

Клиент для ПК имеет широкие возможности по формированию отчетов и графиков, позволяет выводить их на печать.


\chapter{Интеграция со сторонними приложениями}
\label{\detokenize{api:chapter-api}}\label{\detokenize{api:id1}}\label{\detokenize{api::doc}}
Блиц Бюджет для Android можно интегрировать с другими приложениями. Например, можно совместить приложение с голосовым
ассистентом и просто надиктовывать операции или автоматизировать создание или заполнение новых операций при помощи
приложения \sphinxhref{https://play.google.com/store/apps/details?id=net.dinglisch.android.taskerm}{Tasker}.


\section{Создание операций из текстовой строки}
\label{\detokenize{api:id2}}
Чтобы создать новую операцию в Блиц Бюджет для Android достаточно отправить широковещательный интент. Получив такой
интент, приложение проанализирует сообщение и создаст новую операцию по алгоритму, аналогичному распознавания SMS
и push-уведомлений.

Параметры интента:

Class = \DUrole{bbvar}{biz.interblitz.intent.CONVERT\_TEXT\_TO\_NEW\_TRANSACTION}

Extras:
\begin{enumerate}
\def\theenumi{\arabic{enumi}}
\def\labelenumi{\theenumi .}
\makeatletter\def\p@enumii{\p@enumi \theenumi .}\makeatother
\item {} 
\DUrole{bbvar}{timestampMillis}: Тип long, дата и время новой операции в миллисекундах. Значение может быть не указано, тогда используются текущие дата и время.

\item {} 
\DUrole{bbvar}{address}: Тип String, адресат сообщения, может быть не указан.

\item {} 
\DUrole{bbvar}{message}: Тип String, текстовое сообщение, из которого приложение создаст новую операцию. Сообщение по своей структуре должно быть аналогично SMS. Обязательный параметр.

\end{enumerate}


\section{REST API}
\label{\detokenize{api:rest-api}}\label{\detokenize{api:sub-chapter-rest-api}}
Блиц Бюджет для Android поддерживает \sphinxhref{https://github.com/interblitz/BudgetBlitz-Api}{REST API}. API позволяет создавать новые объекты, справочники и операции, а также редактировать и удалять существующие. Используя API
можно создавать свои дополнения и приложения.

Для работы с \sphinxhref{https://github.com/interblitz/BudgetBlitz-Api}{REST API} и просмотра документации необходимо включить удаленный доступ, см. {\hyperref[\detokenize{remote-access:chapter-remote-access}]{\sphinxcrossref{\DUrole{std,std-ref}{Удаленный доступ}}}} (\autopageref*{\detokenize{remote-access:chapter-remote-access}}).
Документация к API доступна через \sphinxhref{https://interblitz.github.io/BudgetBlitz-Api/swagger/}{Swagger}. После загрузки страницы в адресной строке \sphinxhref{https://interblitz.github.io/BudgetBlitz-Api/swagger/}{Swagger} укажите \sphinxurl{http://{[}server{]}:{[}port{]}/api/v1/docs.json}.
В качестве сервера (server) и порта (port) необходимо указать значения, которые выводятся при включении удаленного доступа.

Примеры приложений можно посмотреть на \sphinxhref{https://github.com/interblitz/BudgetBlitz-Api}{github.com}. В примерах также нужно указать адрес к Блиц Бюджет для Android в виде \sphinxurl{http://{[}server{]}:{[}port}{]}.


\section{Intents API}
\label{\detokenize{api:intents-api}}
Помимо простого API для создания операций из текста Блиц Бюджет для Android предоставляет расширенный Intents API.
Intents API  состоит из двух частей, событий и запросов данных,.и построен на базе REST API.
По умолчанию Intents API выключен. В настройках можно указать, какая часть API должна быть включена.


\subsection{Intents API: Часть 1, События}
\label{\detokenize{api:intents-api-1}}
События возникают при записи элементов справочников и операций. При возникновении события Блиц Бюджет для Android отправляет интент тем пакетам,
которые указаны в настройках. В интенте передается:

Action = \DUrole{bbvar}{\{biz.interblitz.budget\{free/pro\}.api.event.ITEM\_ONCHANGE}

Extras:
\begin{enumerate}
\def\theenumi{\arabic{enumi}}
\def\labelenumi{\theenumi .}
\makeatletter\def\p@enumii{\p@enumi \theenumi .}\makeatother
\item {} 
\DUrole{bbvar}{collection} - наименование коллекции, для которой произошло событие

\item {} 
\DUrole{bbvar}{id} - идентификатор объекта, для которого произошло событие

\end{enumerate}

При импорте уведомлений в Extras дополнительно передаются
\begin{enumerate}
\def\theenumi{\arabic{enumi}}
\def\labelenumi{\theenumi .}
\makeatletter\def\p@enumii{\p@enumi \theenumi .}\makeatother
\item {} 
\DUrole{bbvar}{notification} - текст уведомления

\item {} 
\DUrole{bbvar}{address} - адрес уведомления (номер телефона или наименование пакета)

\item {} 
\DUrole{bbvar}{amount} - сумма операции

\item {} 
\DUrole{bbvar}{currency} - валюта операции

\end{enumerate}

Для получения дополнительных данных, которых нет в интенте, следует отправить запрос.


\subsection{Intents API: Часть 2, Запросы}
\label{\detokenize{api:intents-api-2}}
Для запроса данных следует отправить Intent:

Class = \DUrole{bbvar}{biz.interblitz.service.ApiReceiver}

Action = \DUrole{bbvar}{\{biz.interblitz.budget\{free/pro\}.api.request}

Extras:
\begin{enumerate}
\def\theenumi{\arabic{enumi}}
\def\labelenumi{\theenumi .}
\makeatletter\def\p@enumii{\p@enumi \theenumi .}\makeatother
\item {} 
\DUrole{bbvar}{method} - одно из значений: GET, POST, DELETE

\item {} 
\DUrole{bbvar}{path} - путь к данным

\item {} 
\DUrole{bbvar}{body} - данные в формате JSON

\item {} 
\DUrole{bbvar}{package} - полное имя пакета, которому следует вернуть ответ, если не указано - ответ не будет возвращен

\item {} 
\DUrole{bbvar}{class} - класс пакета, которые примет ответ, можно не указывать.

\end{enumerate}

Также в Extras можно указать любые дополнительные данные. Все заданные в запросе данные Extras вернутся обратно  в ответе.

В ответ на запрос отправляется Intent:

Action = \DUrole{bbvar}{\{biz.interblitz.budget\{free/pro\}.api.response}

Extras:
\begin{enumerate}
\def\theenumi{\arabic{enumi}}
\def\labelenumi{\theenumi .}
\makeatletter\def\p@enumii{\p@enumi \theenumi .}\makeatother
\item {} 
\DUrole{bbvar}{collection} - наименование коллекции, для которой отправляется ответ

\item {} 
\DUrole{bbvar}{response} - ответ в формате JSON

\end{enumerate}

Параметры \DUrole{bbvar}{method}, \DUrole{bbvar}{path}, \DUrole{bbvar}{body}, \DUrole{bbvar}{collection}, \DUrole{bbvar}{response} соответствуют REST API. Документация к ним доступна через \sphinxhref{https://interblitz.github.io/BudgetBlitz-Api/swagger/}{Swagger}.
Подробней см. {\hyperref[\detokenize{api:sub-chapter-rest-api}]{\sphinxcrossref{\DUrole{std,std-ref}{REST API}}}} (\autopageref*{\detokenize{api:sub-chapter-rest-api}}).


\chapter{Отличия между версиями}
\label{\detokenize{versions:chapter-versions}}\label{\detokenize{versions:id1}}\label{\detokenize{versions::doc}}

\begin{savenotes}\sphinxatlongtablestart\begin{longtable}{|\X{25}{40}|\X{10}{40}|\X{5}{40}|}
\sphinxthelongtablecaptionisattop
\caption{Отличия между версиями\strut}\label{\detokenize{versions:id2}}\\*[\sphinxlongtablecapskipadjust]
\hline
\sphinxstyletheadfamily &\sphinxstyletheadfamily 
Версия Free
&\sphinxstyletheadfamily 
Версия Pro
\\
\hline
\endfirsthead

\multicolumn{3}{c}%
{\makebox[0pt]{\sphinxtablecontinued{\tablename\ \thetable{} -- продолжение с предыдущей страницы}}}\\
\hline
\sphinxstyletheadfamily &\sphinxstyletheadfamily 
Версия Free
&\sphinxstyletheadfamily 
Версия Pro
\\
\hline
\endhead

\hline
\multicolumn{3}{r}{\makebox[0pt][r]{\sphinxtablecontinued{Продолжается на следующей странице}}}\\
\endfoot

\endlastfoot

Финансовый учет
&\begin{itemize}
\item {} 
\end{itemize}
&\begin{itemize}
\item {} 
\end{itemize}
\\
\hline
Планирование
&\begin{itemize}
\item {} 
\end{itemize}
&\begin{itemize}
\item {} 
\end{itemize}
\\
\hline
Отчеты
&\begin{itemize}
\item {} 
\end{itemize}
&\begin{itemize}
\item {} 
\end{itemize}
\\
\hline
Импорт SMS, OFX, CSV
&\begin{itemize}
\item {} 
\end{itemize}
&\begin{itemize}
\item {} 
\end{itemize}
\\
\hline
Импорт Push уведомлений
&
15 в месяц
&\begin{itemize}
\item {} 
\end{itemize}
\\
\hline
Дополнительный анализ SMS с паролями подтверждения операции
&&\begin{itemize}
\item {} 
\end{itemize}
\\
\hline
Дополнительный импорт данных из SMS с детализацией платежа (две SMS для одной операции)
&&\begin{itemize}
\item {} 
\end{itemize}
\\
\hline
Экспорт OFX
&&\begin{itemize}
\item {} 
\end{itemize}
\\
\hline
Коллективная работа
&
Только передача данных
&\begin{itemize}
\item {} 
\end{itemize}
\\
\hline
Удаленный доступ
&
Только 50 операций
&\begin{itemize}
\item {} 
\end{itemize}
\\
\hline
Напоминания о предстоящих платежах
&&\begin{itemize}
\item {} 
\end{itemize}
\\
\hline
Напоминания на базе отчетов
&&\begin{itemize}
\item {} 
\end{itemize}
\\
\hline
Создание ярлыков с настройками для быстрого открытия отчетов
&&\begin{itemize}
\item {} 
\end{itemize}
\\
\hline
Использование фильтров в виджетах (виджет как отчет)
&&\begin{itemize}
\item {} 
\end{itemize}
\\
\hline
Автоматическое резервное копирование по расписанию
&&\begin{itemize}
\item {} 
\end{itemize}
\\
\hline
Шифрование резервных копий
&&\begin{itemize}
\item {} 
\end{itemize}
\\
\hline
Отчетность для программы 1С:Предприятие 8.2, 8.3
&&\begin{itemize}
\item {} 
\end{itemize}
\\
\hline
Техническая поддержка
&\begin{itemize}
\item {} 
\end{itemize}
&\begin{itemize}
\item {} 
\end{itemize}
\\
\hline
\end{longtable}\sphinxatlongtableend\end{savenotes}

Перейти на Google Play:

\sphinxhref{https://play.google.com/store/apps/details?id=biz.interblitz.budgetfree}{Версия Free}

\sphinxhref{https://play.google.com/store/apps/details?id=biz.interblitz.budgetpro}{Версия Pro}


\chapter{Переход на версию Pro}
\label{\detokenize{migration-to-pro:pro}}\label{\detokenize{migration-to-pro:chapter-migration-to-pro}}\label{\detokenize{migration-to-pro::doc}}
Переход выполняется в два этапа. Сначала нужно подготовить данные вашей версии, а затем загрузить их в новую версию программы. Предварительно убедитесь что SD карта подключена к мобильному устройству и на ней достаточно свободного места.
\begin{enumerate}
\def\theenumi{\arabic{enumi}}
\def\labelenumi{\theenumi .}
\makeatletter\def\p@enumii{\p@enumi \theenumi .}\makeatother
\item {} 
Загрузите версию Free;

\item {} 
В главном меню выберите \sphinxmenuselection{Действия \(\rightarrow\) Экспорт \(\rightarrow\) В версию Pro};

\item {} 
Загрузите версию Pro;

\item {} 
В главном меню выберите \sphinxmenuselection{Действия \(\rightarrow\) Импорт \(\rightarrow\) Из версии Free}.

\end{enumerate}


\chapter{Сервисное обслуживание}
\label{\detokenize{service:chapter-service}}\label{\detokenize{service:id1}}\label{\detokenize{service::doc}}
Как правило приложение Блиц Бюджет для Android не требует сервисного обслуживания. Тем не менее, если Вы замечаете
падение скорости работы программы, то запуск сервисных операций может решить проблему.

\noindent\sphinxincludegraphics[width=0.250\linewidth]{{service-010-select-actions}.png}

\noindent\sphinxincludegraphics[width=0.250\linewidth]{{service-020-select-service}.png}

\noindent\sphinxincludegraphics[width=0.250\linewidth]{{service-030-select-iitems}.png}

Сжатие данных освобождает захваченное и неиспользуемое место в памяти, дефрагментирует таблицы и индексы.
Это способствует увеличению производительности работы приложения. Технически, сжатие данных вызывает команду \sphinxhref{https://sqlite.org/lang\_vacuum.html}{VACUUM}.

Сжатие данных работает только с базой данных и не удаляет файлы, которые также могут располагаться в данных приложения.

Переиндексация может помочь в случае резкого падения производительности. Технически, переиндексация данных вызывает
команду \sphinxhref{https://sqlite.org/lang\_reindex.html}{REINDEX}.

\begin{sphinxadmonition}{warning}{Предупреждение:}
Не забывайте делать резервные копии, особенно перед выполнением сервисных операций. Если установлен пароль шифрования, то обязательно убедитесь, что Вы его помните. Иначе восстановить данные из резервной копии будет невозможно.
\end{sphinxadmonition}


\chapter{Поддерживаемые банки}
\label{\detokenize{banks:chapter-supported-banks}}\label{\detokenize{banks:id1}}\label{\detokenize{banks::doc}}

\section{Maldives}
\label{\detokenize{banks:maldives}}
BML


\section{Беларусь}
\label{\detokenize{banks:id2}}
БПС-Сбербанк

БелВнешЭкономБанк

Белагропромбанк

Беларусбанк

Белгазпромбанк

Белинвестбанк

Белросбанк

МТБанк

Приорбанк

Хоум Кредит Беларусь


\section{Бразилия}
\label{\detokenize{banks:id3}}
Banco do Brasil

Itaú Unibanco


\section{Венгрия}
\label{\detokenize{banks:id4}}
CIB BANK

OTP Bank - Simple


\section{Вьетнам}
\label{\detokenize{banks:id5}}
Australia and New Zealand Banking Group


\section{Индия}
\label{\detokenize{banks:id6}}
Central Bank of India

Deutsche Bank

State Bank of India


\section{Индонезия}
\label{\detokenize{banks:id7}}
Commonwealth Bank


\section{Канада}
\label{\detokenize{banks:id8}}
ICICI Bank


\section{Объединенные Арабские Эмираты}
\label{\detokenize{banks:id9}}
Emirates Islamic bank

Emirates NBD


\section{Польша}
\label{\detokenize{banks:id10}}
Bank Millennium SA


\section{Россия}
\label{\detokenize{banks:id11}}
AnyBalance

BSGV

KARI CLUB

Modulbank

QIWI

SDM-Bank

АКИБАНК

АМТ Банк

Абсолют Банк

Авангард

АйМаниБанк

АкБарс

Альфа-Банк

БКС БАНК

Балтийский Банк

Банк Европейский

Банк Москвы

Банк Петрокоммерц

Банк Приморье

Банк Санкт-Петербург

Банк Советский

Банк Точка

Банк Транспортный

Банк УРАЛСИБ

Банк Финсервис

Банк24.ru

Барклайс-Банк

Белгородсоцбанк

Бинбанк

ВТБ 24

ВУЗ-Банк

Внешпромбанк

Возрождение Банк

Восточный экспресс

Всероссийский банк развития регионов

Вятка-банк

ГЛОБЭКСБАНК

ГУТА Банк

Газпромбанк

Газпромбанк Доп карта

Дальневосточный Банк

Европлан

ЕвроситиБанк

Екатеринбургский Муниципальный Банк

Запсибкомбанк

Инвестбанк

Интеркоммерц

Интерпрогрессбанк

Кедр

Кольцо Урала

КредитЕвропаБанк

Кукуруза

Липецккомбанк

ЛокоБанк

МДМ Банк

МИНБанк

МТС банк

Мастербанк

Меткомбанк

Московский кредитный

Москомприватбанк

НБ Траст

Нефтепромбанк

Новый Символ

Номос Банк

ОТП Банк

Первый Республиканский Банк

Почта Банк

Промсвязьбанк

Райффайзен Банк

Региональный банк развития

Рокетбанк

РосЕвроБанк

Росбанк

РоссельхозБанк

Россия

РостФинанс

Русский Стандарт

СКБ-Банк

СМП Банк

Сбер.книжка

Сбербанк России

Сбербанк-Maestro Поволжье

Связной Банк

Связь-Банк

Севергазбанк

Ситибанк

Собинбанк

Солидарность

Сургутнефтегазбанк

ТААТТА

Татфондбанк

Тачбанк

Тинькофф

ТрансКредитБанк

Трастбанк

Урал ФД

УралПромБанк

УралТрансБанк

Уральский банк реконструкции и развития

ФК Открытие (бывш. НОМОС-Банк)

ФОНДСЕРВИСБАНК

Ханты-Мансийский Банк

Хоум Кредит

Центр-инвест

Челиндбанк

Челябинвестбанк

Экспресс

ЭнергоМашБанк

Юниаструм Банк

Юникредит Банк

Яндекс.Деньги


\section{Соединенные Штаты}
\label{\detokenize{banks:id12}}
First National Bank

Guardian Alert General

Pendleton Community Bank

Town Bank

UniBank


\section{Таиланд}
\label{\detokenize{banks:id13}}
KASIKORNBANK


\section{Узбекистан}
\label{\detokenize{banks:id14}}
Uzcard


\section{Украина}
\label{\detokenize{banks:id15}}
VAB Банк

АБанк

Альфа-Банк

Альфа-Банк Украина

БРОКБИЗНЕСБАНК

Донгорбанк

Експресс-банк

Индустриал

КРЕДОБАНК

Михайлівський

ОТП Банк

ОщадБанк

ПУМБ

Петрокоммерц Украина

ПриватБанк

ПроКредитБанк

Проминвестбанк

Райффайзенбанк Аваль

Сбербанк России в Украине

УкрСибБанк

Укрексімбанк

Укрсоцбанк


\chapter{Определения и термины}
\label{\detokenize{glossary:chapter-index}}\label{\detokenize{glossary:id1}}\label{\detokenize{glossary::doc}}\begin{description}
\item[{контрагент\index{контрагент@\spxentry{контрагент}|spxpagem}\phantomsection\label{\detokenize{glossary:term-2}}}] \leavevmode
Контрагентом называется противоположная сторона в операции.

\item[{сплит\index{сплит@\spxentry{сплит}|spxpagem}\phantomsection\label{\detokenize{glossary:term}}}] \leavevmode
Расшифровка операции называется сплитом. В сплите можно указать статью, проект, персону и комментарий
для каждой строки расшифровки.

\item[{техническая статья\index{техническая статья@\spxentry{техническая статья}|spxpagem}\phantomsection\label{\detokenize{glossary:term-3}}}] \leavevmode
Технической называется статья, в которой выключены признаки \DUrole{bbproperty}{Revenue} и \DUrole{bbproperty}{Expense}.

\item[{узел обмена\index{узел обмена@\spxentry{узел обмена}|spxpagem}\phantomsection\label{\detokenize{glossary:term-1}}}] \leavevmode
Узлом обмена или просто узлом называется устройство, участвующее в обмене данными при коллективной работе.

\end{description}


\chapter{Indices and tables}
\label{\detokenize{index:indices-and-tables}}\begin{itemize}
\item {} 
\DUrole{xref,std,std-ref}{genindex}

\item {} 
\DUrole{xref,std,std-ref}{search}

\end{itemize}



\renewcommand{\indexname}{Алфавитный указатель}
\printindex
\end{document}